\chapter{EVALUATION PARADIGMS}

\section{Introduction}

\paragraph{Evaluation.} The system described in this proposal is an unsupervised learning system and is thus inherently difficult to evaluate, as one goal of unsupervised learning is to discover previously unknown categories \citep{parsons:2004}.
The previously unknown categories in this case are going to be morphological units of some kind.
However, these units will not be conventional morphemes or morphosyntactic categories. 

Instead, MCMM-generated clusters will correspond roughly to Aronoff's 
\emph{morphomes} \citep{aronoff:1994}, which can be described as \emph{pre-morphosyntactic} units, i.e.,
units that have been assembled from phonemes, but have not yet been assigned 
a syntactic or semantic meaning. I will use the term \emph{morph} instead of 
\emph{morphome}, however, since MCMM-generated clusters may not correspond 
precisely to morphomes in every case (see section %~\ref{sec:targets}).

Thus, the evaluation itself presents an important research question, namely the question of how to evaluate the output of an unsupervised morphological clustering algorithm, particularly one that considers only features of \emph{word-internal form}, having no access to word-external morphosyntactic features, e.g., the person, number, and gender of  surrounding words.

As explained in section (ref), my system is intrinsically difficult to evaluate, and thus no single evaluation method is likely to be perfect. I will therefore use two complementary methods, %I would thus rather not rely on a single method.
%not only because is it an unsupervised learning system, which are notoriously challenging to evaluate,  it learns unconventional morphological units, namely morphs. Indeed, it would be very difficult to come up with a gold standard for the morphs because we do not know what the morphs are supposed to look like; i.e., the ``right answers" are not obvious at the outset. 
%In particular, I will use 
one \emph{extrinsic} method and one \emph{intrinsic} method.

\section{Intrinsic Evaluation}
An \emph{intrinsic} evaluation, by contrast, considers a system as independent, examining its output directly. That is, the system is evaluated as a stand-alone application, not as one embedded in a larger system.

The intrinsic component of the evaluation will compare an MCMM-generated clustering to a 
\emph{gold-standard categorization}.\footnote{By convention, 
gold-standard clusters are not actually called clusters, 
but (gold-standard) \emph{categories}.} 
An MCMM's output consists of 
%Given the final valuations in the MCMM matrices $\mathbf{M}$ and $\mathbf{C}$,
the final values in the matrices $\mathbf{M}$ and $\mathbf{C}$.
Given these values, one can derive a list of word-to-cluster mappings; each mapping, or item, 
in this list
consists of a word followed by a list of cluster labels, i.e., labels pointing to
the clusters in which the word has membership, 
as in ``runs: run, -s,"  where \textit{runs} is the word, and \textit{run} and \textit{-s} 
are the labels of the clusters to which \textit{runs} belongs. These labels also represent morphs. That is, e.g., \emph{-s} represents a cluster whose words end with the morph \emph{-s}.

To evaluate a list of mappings like ``runs: run, -s," we need an analogous gold-standard list, 
one that supplies, for each word, a list of
gold-standard \emph{categories}.
%(i.e., the categories to which the word belongs). 
This is to say that we need a gold-standard morphological analysis for each word. 
We actually need different gold standard lists for the orthographic and transcribed datasets, 
since these datasets come from different sources, and their respective gold standards must 
be obtained through separate means.

% (Notice that a lot of properties are packed into the suffix \textit{-s}).
% Let us consider  
% briefly the nature of an MCMM's output. One can represent a 
% clustering (or categorization) as a table in which the rows 
% correspond to the clustered (or categorized) items, and 
% the columns to the clusters (or categories) themselves. Now, 
% in the proposed dissertation, the items will be feature-vector 
% representations of words, and the clusters will be morphs. Each 
% $\text{cell}_{i,k}$ contains either 1 or 0: 1 if $\text{word}_i$ has 
% morph $k$, and 0 if it does not. 
%%What we have just described is essentially the MCMM's $\mathb{M}$ matrix. 
% Each 1 thus represents the presence of a particular morph. 
 
% If we wanted to get away from the matrix format, i.e., the grid 
% of 1's and 0's, we could replace each 1 with its morph and eliminate the 0's altogether. 
% Suppose, for example, 
% that $\text{word}_{1253}$ is \textit{runs}, which has two morphs, namely the stem 
% \textit{run} and the suffix \textit{-s} 3rd-person (\textsc{3p}), present-tense (\textsc{Pres}) singular (\textsc{Sg}) s.
% (Notice that a lot of properties are packed into the suffix \textit{-s}). 
% Suppose further that \textit{run} is cluster 87 and \textit{-s} is cluster 6, and the overall clustering 
% has a total of $K=500$ clusters, in which case $\mathbf{M}$ has 500 columns, and
% row 1253 has 498 zeros and only two 1's, one at column 6 and the other at column 87.
%We can replace the 
%1 in column 6 with \textit{-s} and the 1 in column 87 with \textit{run} 
%and discard the zeros, yielding ``runs: run, -s," which is tantamount a 
%morphological segmentation. 
%---
%Notice that the suffix \textit{-s} maps to three 
%``atomic" morphosyntactic categories, 
%namely 3rd-person (\textsc{3p}), present-tense (\textsc{pres}), 
%and singular (\textsc{sg}). 
%My system cannot learn abstract morphosyntactic labels like \textsc{3p} 
%and \textsc{sg}. Rather, it learns \emph{morphs}, 
%%the pre-morphosyntactic units of form. We will call these units \emph{morphs}, 
%which may or may not correspond to morphosyntactic categories (see the discussion in section~\ref{sec:targets}). 
%When there \emph{is} a 
%correspondence between morphs and morphosyntactic categories, it is often a 
%one-to-many mapping because the same morph can be
%requisitioned by more than one morphosyntactic category.
%---
 %can lay claim to the same morph. 
%In ``runs: run, -s," for example, the suffix \textit{-s} 
%represents the union of three ``atomic" morphosyntactic categories, namely \textsc{3p}, \textsc{pres}, and \textsc{sg}. 

%Thus, the output of an MCMM, after a little post-processing, can look like ``runs: run, -s." It is essentially a list of morphological segmentations. But more accurately, it is a list of word-to-cluster mappings; for each word, it will specify the cluster(s) to which it belongs. Note that the morphs \emph{run} and  \emph{-s} are essentially the labels of particular morphological clusters.
%The output of an MCMM is thus essentially a list of word-to-cluster mappings. Each item in this list is a word followed by a list of the clusters in which it has membership.
%%After little post-processing, it can look like ``runs: run, -s." 
%Morphs like \emph{run} and \emph{-s} are essentially cluster labels. 
%That is, \emph{-s} represents a cluster whose words predominantly end in  \emph{-s}.

\paragraph{Orthographic data.}

To obtain morphological analyses for the wordlist O, I will use the MILA Morphological
Analysis tool (\textsc{mila-ma}) \citep{hebrew-resources:2008}.
Because \textsc{mila-ma} requires that input words be spelled 
according to Modern Hebrew standard orthography, it can only be
used to create a gold standard for orthographic wordlist O. The gold-standard morphological analyses
for the transcribed wordlists TS and TR must come from a different source (see below).
% \textsc{mila-ma} is in essence a finite-state transducer, a variation
% of the finite-state automaton. 
\textsc{mila-ma} is essentially a finite-state transducer. Because its morphological knowledge has been manually coded by humans and its output is
deterministic, it provides a good approximation to human
annotation. 

However, many of the original \textsc{mila-ma} categories are
ill-suited to the purpose of evaluating an MCMM's clustering. The \textsc{mila-ma} categories are
often atomic and abstract, e.g., \texttt{feminine} and \texttt{masculine}. Such categories are purely morphosyntactic; they are meaningless at the word-internal level because they can only be observed in agreement phenomena. Moreover, there is no morphological unit in Hebrew that means strictly `feminine,' %(i.e., nothing more than `feminine' and nothing less). 
nor is there one that means strictly ``masculine." Hebrew inflectional affixes 
tend to be fusional, having meanings like ``feminine plural" and ``masculine plural."

For this and similar reasons, \textsc{mila-ma}'s categories need to be mapped to a modified set 
of gold-standard categories, i.e., categories that correspond more closely to actual differences in form.
The MCMM's clustering will then be quantitatively compared to the modified \textsc{mila-ma}-based gold-standard categorization. 
In particular, I will use the measures \emph{average cluster-wise purity}, \emph{BCubed precision} and \emph{BCubed recall}. The latter two are important because they are specifically designed for cases of overlapping clusters \citep{amigo-et-al:2009}.

\paragraph{Transcribed data.} I will obtain gold-standard category mappings for the transcribed words by
extracting morphological analyses from the Berman longitudinal corpus.
Recall that for each utterance in the Berman longitudinal corpus, 
there is a transcription tier and a morphological-analysis tier. 
The latter provides a morphological 
analysis for each word in the utterance,
including roots for the words that have roots. I will extract the 
morphological analyses and use them to create a list of word-to-categories mappings like one created
from \textsc{mila-ma}'s analyses.


\section{Extrinsic Evaluation} \label{sec:eval-extrinsic}
An \emph{extrinsic evaluation} views a system as a component of a larger, or outer, system. Its purpose is to evaluate the embedded system, but it does so by evaluating the outer system. If the outer system scores highly,
the embedded system, i.e., the system under evaluation, scores highly.
An extrinsic evaluation makes sense for my system for two reasons:
\begin{enumerate}
\item My system already functions as an embedded system: It learns morphs, which are intermediate units, intended to facilitate the learning of morphemes. %My system is thus meant to be an embedded component of a larger process. 
\item While it would be very difficult to come up with gold-standard morphs, gold-standard morphemes are relatively easy to produce, since morphemes, in contrast to morphs, are already well-defined. 
%As intermediate units, the value of morphs lies in their utility, i.e., in their capacity for yielding correct morphemes. What they look like is not important as long as they are effective. is evaluate \emph{morphemes} that have been induced from morphs. 
%On the other hand, it is relatively easy to come up with gold-standard morphemes, since morphemes, in contrast to morphs, are well-defined. 
\end{enumerate}

The extrinsic evaluation will consist of the four stages described below. 
Stages 1 to 3 prepare the output of an MCMM to be fed to Morfessor in Stage 4. 
To obtain the gold-standard datasets for Morfessor, I will manually segment 
$\frac{1}{10}$ of the original, unprocessed wordlists, i.e., both the transcribed and orthographic wordlists. 
Note that this four-stage extrinsic evaluation only considers stem-external, 
concatenative morphology. This is because Stage 3 effectively removes interdigitation.
Morfessor is not even capable of handling interdigitation, since it is a sequential algorithm.

\begin{description}
\item[Stage 1: Extract.] Derive morphs from cluster centroids. That is, for each cluster centroid vector,  map the \emph{active} features to a particular sequence of alphabetic characters. The mappings will be governed by a set of mapping rules. The resulting sequence of alphabetic characters is the morph. Repeat this process for each cluster (i.e., cluster centroid).

\textbf{Example:} Suppose that \texttt{z<k} and \texttt{k<r} are the active features in a given cluster's centroid. These features would map to the (potentially discontinuous) character sequence \textit{zkr}. The morph would thus be the root \textit{z.k.r}.

\item[Stage 2: Match.] Map morph characters to word characters. That is, given a cluster and its morph (obtained in Stage 1), go through the cluster's words, and in each word, determine which characters are the morph's characters. Label these characters as components of the morph in question. Repeat this process for each cluster/morph.
%Each morph is a (possibly discontinuous) sequence of alphabetic characters. 
%Given a cluster and its newly extracted morph, identity the morph's characters in each of the clusters words. 
%That is, for each word, match the morph's characters to the \emph{correct} word characters. Note that there is potential for ambiguity here. Suppose, for example, that the morph in question is the \textit{-wt}. It's easy enough to find a single \texttt{t} in a string of letters, but \texttt{t} is a frequently occurring letter, and it could easily occur elsewhere in the word. I have to make sure my matching algorithm selects the right character in cases like this. Repeat this process for each cluster. 
%For each word $w$ in a given cluster, identify the characters in $w$ that correspond to the morph's characters. counterparts of each the morph characters to their counterpart character in the word in question. with their matching characters in the word in 
%the characters of the morph to the corresponding characters in the cluster's member words.

\textbf{Example:}  
Consider a cluster whose member words are \textit{mazkir}, \textit{hizkir}, \textit{zoker}, \textit{zokrim}, and \textit{zikron}. The morph in this case is the root \textit{z.k.r}. Stage 2 identifies the root consonants in each word and labels them as components of the morph \textit{z.k.r}. Here, the morph's characters are ``labeled" via boldface type:
%\footnote{In reality, of course, a larger and more sophisticated labeling/indexing system will be necessary, as every morph will require a distinct label/index.}:  
\textit{ma\textbf{zk}i\textbf{r}},
\textit{hi\textbf{zk}i\textbf{r}}, \textit{{z}o\textbf{k}e\textbf{r}}, \textit{\textbf{z}o\textbf{kr}im},
and \textit{\textbf{z}i\textbf{kr}on}.
The process is repeated for each cluster/morph.

\item[Stage 3: Compress.] Map each morph to a single unique unicode character.
\textbf{Example:} Consider the 
Hebrew words \textit{magdil} and \textit{gadol}, which share the root \textit{g.d.l}. 
Suppose that the Stage-2 outputs for these words are as in \eqref{ex:magdil} and \eqref{ex:gadol}, respectively. 
\begin{exe}  \ex \label{ex:unicode} \begin{xlist}
	\ex magdil \quad ma-, \,\, g.d.l, \,\, i 
	\label{ex:magdil}
	\ex gadol \, \quad  g.d.l, \,\, a.o
	\label{ex:gadol}
	\end{xlist}
\end{exe}
Altogether, there are four \emph{unique} morphs in \eqref{ex:unicode}, namely \textit{ma-}, \textit{g.d.l}, 
\textit{i}, and \textit{a.o}.
Each of these is mapped to a unique atomic symbol, as in \eqref{ex:map}.
\begin{exe}
	\ex  \textit{ma-} $\mapsto$ \$ \quad \textit{g.d.l} $\mapsto$ \% \quad
\textit{i} $\mapsto$ \textit{i} \quad \textit{a.o} $\mapsto$ \#
\label{ex:map}
\end{exe}
%Let the atomic symbols inherit the sequential order of their counterpart morphs. 
In general, the atomic symbols inherit the ordering of the original morphs. 
The exceptional cases are those of interdigitation. When two morphs are interleaved, 
they are unordered with respect to each other.
However, when they are mapped to atomic symbols, they necessarily 
take on an arbitrary relative order because there is no way to interleave two 
\emph{atomic} units: either $A$ precedes $B$ or $B$ precedes $A$; 
there is no other option.
%but with the atomic symbols now taking the places of of the original morphs. Put the symbols in the same order as their morph counterparts.  hat for morphs that are two or more characters long.. They are put together in the same order as the original morphs. This reducing or elsince it replaces whole character sequences, even discontinuous ones, with atomic symbols (see section~). 
\begin{exe}  
	\ex \label{ex:reassembled} \begin{xlist}
	\ex magdil \quad \$\%\textit{i}
	\label{ex:re-magdil}
	\ex gadol \, \quad \%\#
	\label{ex:re-gadol}
	\end{xlist}
\end{exe}
%Now, when the characters of the two morphs are interleaved, i.e. in the case of interdigitation, the relative order of the morphs is indeterminate. However, when these two morphs are mapped to atomic symbols, they necessarily take on an arbitrary relative order. That is, either $A$ precedes $B$ or $B$ precedes $A$; there is no other option. 
The mapping from morphs to atomic symbols thus abstracts interdigitation.  

\item[Stage 4: Test.]
%-- concatenative morphs.]
Feed both the control file (i.e., the file containing the original wordlist) and the test file (i.e., the output of Stage 3) to
Morfessor \citep{creutz-and-lagus:2005, creutz-and-lagus:2007}. Morfessor then will induce morphological segmentations for each input file, yielding a \emph{control segmentation} and a \emph{test segmentation}. The words in the test segmentation will at this point still consist of the atomic symbols from Stage 3. Without disrupting Morfessor's segmentation decisions, change the atomic symbols back to morphs (i.e., undo Stage 3). Finally, evaluate the two segmentations against a common gold standard.
%control file, will 
%%now be fed to Morfessor. 
%The test file will be the output of Stage 3.  The control file will contain the original, unaltered words. 
%It will provide a baseline 
%for measuring the effect of the compression carried out in Stage 3. 
%The idea here is to see if the MCMM's morphs, 
%now represented as atomic symbols, aide the process of morphological segmentation. 
%That is, do they make the task easier? 
%Do they improve segmentation accuracy? 
\end{description}
