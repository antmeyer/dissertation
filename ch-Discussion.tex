\chapter{DISCUSSION AND CONCLUSION}
\label{ch:conclusion}
%My evaluation scripts have been proving difficult to debug, but once they are ready, I will perform the full evaluation, which 
%will inform a lot of what I say here. For the time being, however, the evidence yielded through manual inspection of the output shows that
%MCMMs are quite capable of learning non-concatenative morphology. See the cluster samples displayed in chapter~\ref{ch-Results} as 
%figures~\label{fig:cl-fem} and \label{fig:cl-hit}.

This dissertation has proposed a multilinear approach to the learning of morphology.
%, both concatenative and nonconcatenative. 
%multilinear is 
%equivalent to nonlinear, i.e., not everything is located in the same line (or in the row). 
%Rather, both multilinear and nonlinear 
%mean that there is more than one row in which things can reside, and moreover, that 
%such rows can interact with each other.  
This dissertation’s qualitative results alone demonstrate the 
viability of the multilinear approach to ULM. 
Even though the quantitative results
are pending, the qualitative evidence yielded by manual 
inspection is already sufficient to demonstrate the validity of the 
multilinear approach to ULM, both for nonconcatenative 
and concatenative morphology. 
Some of this qualitative evidence was 
presented in figure~\ref{fig:cl-hit} in chapter \ref{ch:results}, which 
showed a sample of a cluster whose were almost entirely forms
of the \emph{hitpa`el} binyan, which are nonconcatenative by nature. 

%Please note that I have already run 
%Multimorph on every pairing of $s$ and $\delta$ in table~\ref{tab:results} and have collected to output for each pair.
%I only need to run to the scripts to carry out the quantitative calculations

This dissertation has established a bridge between linguistic theory 
on the one hand and computation on the other. 
It offers a novel view of autosegmental morphological theory. 
In particular, this dissertation has presented the formalism of autosegmental theory
as an instance of a more general graphical framework, namely 
the bipartite graph. In so doing, it has shed light on what the autosegmental formalism is in 
its essence as well as why autosegmental analyses work for 
cases of nonconcatenative morphology. In chapters~\ref{ch:lit-review} and \ref{ch:graph}, I
defined and discussed the notions of the \emph{nonlinearity} and \emph{nonsequentiality}, 
proposing that they were the two essential properties of the autosegmental approach, 
i.e., that they were what enable an autosegmental
analysis to identify discontinuous morphological units. I proceeded to 
argue in chapter~\ref{ch:graph} that these
properties map nicely onto the definition of \emph{bipartite}, 
hence the connection between the autosegmental formalism and bipartite graphs.
This mathematical interpretation of autosegmental morphology 
motivates the approach I have taken in the present project. 
It predicts that if a ULM system is based on a bipartite graph, then 
it will able to identify nonconcenative morphological
units due to its very bipartiteness. The results presented in chapter~\ref{ch-Results} 
 corroborate this prediction, making 
 Multimorph a sort of proof of concept for the 
 computational implementation of autosegmental 
 theory.

In chapter~{autonomous}, I argue that ULM's target of learning should be reassessed; in particular, that ULM by nature
exists in a realm separate from both phonology and syntax/semantics, and that ULM \emph{autonomous} therefore morphological units similar to Aronoff's morphomes \citep{aronoff:1994} rather than morphemes. ULM systems such as 
Multimorph thus lend support to the notion of the morphome and other concepts of autonomous 
morphology. But in order to lend such support, the actual nature of their learning targets must 
be acknowledged. In chapter~7, I outline intrinsic and extrinsic evaluation procedures that are 
designed to evaluate the output of a system whose learning target is expressly 
non-morphemic, i.e., something like the morphome.

% and moreover, that it will treat nonconcatenative and concatenative morphological units as the same essential thing, via the same mechanisms and structures. 

