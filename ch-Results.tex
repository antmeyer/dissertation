\chapter{RESULTS}
\label{ch:results}

%\begin{CJK}{UTF8}
%工作\UTF{7ECF}\UTF{5386}%
%\end{CJK}

\section{Introduction}
%\begin{CJK}{Bg5}{fs}
%我很喜歡吃中國飯。
%\end{CJK}

As discussed in chapter~\ref{ch:MCMM}, Multimorph's \ac{MCMM} 
in effect groups its input words into morphological clusters. 
During the course of learning, a particular set of hidden-unit values 
(namely, the vector $\mathbf{m}_{i}$) is induced for each or word $i$. 
The vector $\mathbf{m}_{i}$ consists of $K$ elements, each a real number on the interval $[0,1]$. 
Each of these elements corresponds to a particular cluster $k$ and indicates 
the extent to which word $i$ is a member that cluster.
The threshold for cluster membership is $\theta$. That is, if hidden unit 
$\mathbf{m}_{ik}$ has a value equal to or greater than 
$\theta$, then word $i$ is a member of cluster $k$. Otherwise, it is not a member of cluster $k$. 
For example, suppose that word $i$ had the hidden-unit vector $\mathbf{m}_{i} = [0.2, 0.0,0.9,0.1,0.8]$, 
wherein the activities of the third and fifth clusters, 0.9 and 0.8, respectively, 
%$\mathbf{m}_{i,2}$ and $\mathbf{m}_{i,4}$, at 0.9 and 0.8, respectively, 
exceed the cluster-membership 
threshold, while the other three values
are well below it. Thus, of the five clusters in this hypothetical model, word $i$ is a member of the third and fifth clusters.
%2 and 4 (i.e., the 3rd and 5th clusters).

In this way, the $I \times K$ matrix $\mathbf{M}$, i.e., the collection of all $I$ hidden-unit vectors $\mathbf{m}_i$,
defines a \emph{disjunctive clustering}, i.e., a clustering in which a single item can 
%\emph{fully} (not just \emph{partially} or \emph{possibly}) 
belong to multiple clusters at once \citep{manning-and-schutze:1999}. 
A disjunctive clustering is thus to be distinguished from soft clusterings 
%that a standard mixture model would produce as well as the 
and mixed-membership clusterings (see the discussion of mixture and mixed-membership models 
in chapter~\ref{ch:MCMM}). Each of the $I$ input words can belong to up to $K$ clusters. 
Each of the $K$ columns serves as a particular cluster's membership indicator, indicating the membership status of each data point as one moves down the rows of  $\textbf{M}$. Multimorph produced such a clustering for 
each of the experimental parameter combinations described in chapter~\ref{ch:experi}. In chapter~\ref{ch:eval}, 
we motivated and outlined a multi-faceted approach to evaluating Multimorph's output, an approach comprising 
both qualitative and quantitative components as well as both intrinsic and extrinsic components. 
In this chapter, I present the results of this multi-faceted evaluation. 

As a final preliminary matter for this chapter, recall from section~\ref{sec:mcmm-learning} %chapter~\ref{ch:MCMM} 
that Multimorph begins its learning process with a single cluster ($K = 1$) 
and adds a cluster (by splitting the worst cluster) whenever the global reconstruction can no longer be decreased significantly at the current cluster count. In principle, this incremental addition of clusters is to continue until the global error is sufficiently close to $0$, i.e., falls below a predesignated 
threshold close to $0$.  In this study, this predesignated threshold $\epsilon$ was
 $0.0001$.
%falls below a threshold $\epsilon$ close to 0, such as $\epsilon = 0.0001$).
In theory, therefore, Multimorph should create just enough clusters to reduce the error
below $\epsilon$. 
However, in practice,
% in the course of this dissertation's experiments,
 Multimorph's error never reached this threshold.
 Rather, in every experimental trial reported in this thesis, Multimorph's 
 error decreased steadily to a point greater than $\epsilon$, 
 whereupon it reversed direction, starting to 
increase, and continued to increase until the experiment was stopped at $K = 3000$ or $4000$.  
The error minimum generally occurred when $K$ was between 400 and 1200. 
The average $K$ was 589.4 for TS (transcriptions with stress markings), 833.0 for 
TR (transcriptions without stress markings), and 742.9 for O (orthographic data). I thus 
report results both at $K = 500$ and $K = 1000$.

\section{Qualitative Analysis}
\label{sec:qual}
The value of quantitative methods lies in their systematicity and objectivity, but
they are by no means guaranteed to capture every salient fact regarding a system's output. 
especially when the system in question is an unsupervised learning system. 
 This dissertation thus incorporates a qualitative analysis of Multimorph's output to supplement the  
quantitative results presented later in this chapter.
This qualitative analysis consists primarily in a ``manual" inspection the MCMM's clusters.

\subsection{``Hand-Inspection" of Clusters}
Hand-inspection reveals linguistically-interpretable clusters. For example, Figure~\ref{cl-fem} displays a group 
of 60 words have been randomly selected from a 1632-word cluster produced by Multimorph’s MCMM. This group 
is intended as an abridgment of its superset, which is too large to display here. This cluster was produced by the 
MCMM at the experimental settings $\delta = 2$, $s = 2$, and $K = 1000$, i.e. it was one of other 1000 clusters 
that the MCMM produced during this experimental run.


\begin{figure}[t]
\begin{mdframed}
\begin{tabbing}
\hspace*{1ex}\= \hspace*{14ex}\= \hspace*{14ex}\=\hspace*{14ex}\=\hspace*{14ex}\=\hspace*{14ex}\=\hspace*{14ex} \kill
\> nafalt \> pizart \> webarakevet \> werevi\textipa{P}it \> we\textipa{P}omeret \> \textipa{P}emet\\
\> bubot \> hakapot \> mexaberet \> wela\textipa{P}alot \> wemakot \>\v{s}ehizmant\\
\> bamesilot \> bami\v{s}qefet \> fizit \> mistateret \> ye\textsubdot{k}olot \> \textipa{P}o\textsubdot{k}elet\\
\> be\textipa{P}emet \> kazo\textipa{P}t \> ktumot \> laxalonot \> mat\textipa{P}imot \> \textipa{P}axeret\\
\> hatmunot \> hawilonot \> ha\textipa{P}otiyot \> li\v{s}tot \> moxeqet \>\v{s}era\textipa{P}it\\
\> dri\v{s}at \> lanequdot \> maxliqot \> mitxape\textsubdot{s}et \> wexalonot \>\v{s}e\textglotstop{P}amart\\
\> daqot \> habdiqot \> megare\v{s}et \> ni\textsubdot{k}nast \> pi\textsubdot{t}riyot \>\v{s}ehaxanuyot\\
\> ha\v{s}amenet \> hiclaxt \> laxalalit \> meha\textsubdot{s}aqit \> nimce\textipa{P}t \>\v{s}lulonet\\
\> hahit\textipa{P}amlut \> labanot \> mela\textsubdot{k}le\textsubdot{k}et \> safart \> xada\v{s}ot \> \textipa{P}acuvot\\
\> bakisa\textipa{P}ot \> madregot \> melu\textsubdot{k}la\textsubdot{k}ot \> melu\textsubdot{k}le\textsubdot{k}et \> mit\textipa{P}aqe\v{s}et \> \textipa{P}orot
\end{tabbing}
\caption{Sixty words randomly selected from a 1632-word cluster generated by Multimorph's MCMM (at $s = 2$, and $\delta = 2$). The endings on these words are the feminine endings discussed in section~\ref{sec:heb-example} of chapter~\ref{autonomous}.} 
\label{fig:cl-fem}
\end{mdframed}
\end{figure}
\begin{figure}[tb!]
\begin{mdframed}
\begin{tabbing}
\hspace*{1ex}\= \hspace*{17ex}\= \hspace*{17ex}\=\hspace*{17ex}\=\hspace*{17ex}\=\hspace*{17ex} \kill
\> \texttt{w@[1]} (1.0000) \> \texttt{r<t} \, (0.0663) \> \texttt{l<t} \, (0.0455) \> \texttt{q<t} \, (0.0392) \> \texttt{v<t} \, (0.0258) \\
\> \texttt{\textsubdot{k}<t} \, (0.0257) \> \texttt{u<t} \, (0.0240) \> \texttt{d<t} \, (0.0176) \> \texttt{o<t} \, (0.0153) \> \texttt{c<t} \, (0.0143)
\end{tabbing}
\label{fig:fem-features}
\caption{The ten most active features in the cluster represented in figure~\ref{fig:cl-fem}. The feature (surface-unit) activities are included in parentheses.}
\end{mdframed}
\end{figure}

This cluster represents feminine endings discussed these words events one of the feminine endings 
discussed in chapter~{ch:autonomous}, section~\ref{sec:heb-example} namely, the set of suffixes that 
involve combinations of \textit{-u}, \textit{-i}, and \textit{-t}. Notice that all of the endings in figure~\ref{fig:cl-fem} at least share the \textit{t}. 
[Here, I intend to consult another of Multimorph's output documents, one that lists
the top ten most active features for each each cluster. Presumably, these lists would more clearly and
accurately characterize each cluster.
Perhaps, for example, 
the feature \texttt{t@[-1]} is among the most active features underlying the cluster associated with figure~\ref{fig:cl-fem}, but perhaps others 
contribute significantly as well.] 

Another cluster, consisting of 60 words randomly selected from a 426-word cluster, is displayed 
in figure~\ref{fig:cl-hit}. This cluster represents a \emph{binyan}, which, in Hebrew and other Semitic 
languages is a set of verb stems that share the same vowel pattern and combine with the same set 
of affixes. Vowel patterns are the complements of consonantal roots; i.e., a root and a pattern are 
interleaved (or interdigitated) to form a stem. 
Nearly every verb in this cluster is of the \textit{hitpa`el} binyan, which is distinguished by the 
%\{h,m,t,y,n,\textipa{P}\}itCaCeC 
$\diamond$itCaCeC 
pattern, which undergoes some alternations due to inflectional and phonological influences. For instance, the ``$\diamond$'' in $\diamond$itCaCeC  can be occupied by \textit{h}, \textit{m}, \textit{t}, \textit{y}, \textit{n}, or \textit{\textipa{P}}, depending on tense, person, and number. 
Some of the words in figure~\ref{subfig:cl-hit-sample} are nouns derived from the \textit{hitpa`el} binyan, e.g., \textit{ha-hitna\v{s}m-uy-\a'{o}t }, where \textit{ha-} is the definite-article prefix, \textit{hitna\v{s}em}\footnote{the \textit{e} in the stem \textit{hitna\v{s}em} is deleted when the suffixes \textit{-uy-\a'{o}t} are added.}
is the \textit{hitpa\textipa{`}el} stem, and \textit{-uy} is the nominalizing suffix \textit{-ut}, whose \textit{t} becomes become \textit{y} when it immediately precedes a stressed vowel, and \textit{-ot} is the feminine plural suffix. 
%which bears the nominal suffix \textit{-ut} ($\to$ \textit{-uy} before the\textit{o}) and the fem.pl suffix \textit{-ot} 

%words randomly selected from a 426-word cluster generated by Multimorph's MCMM 
%at $s = 2$, and $\delta = 2$. These words are almost entirely of the \textit{hitpa`el} binyan


%\subfigure[60]{
%\begin{tabbing}
%\hspace*{14ex}\= \hspace*{14ex}\=\hspace*{14ex}\=\hspace*{14ex}\=\hspace*{14ex}\=\hspace*{14ex} \kill
%hamitnag\v{s}\a'{o}t \> hitgalgel\a'{a} \> hitnadn\a'{e}d \> lehistap\a'{e}r \> welehitra\textipa{P}\a'{o}t \> wemistak\a'{e}l\\
%hithap\textsubdot{k}\a'{a} \> mitra\v{s}\a'{e}met \> titnagv\a'{i} \> titxat\a'{e}n \> wemistak\a'{e}let \> yitqalq\a'{e}l\\
%hahitna\v{s}muy\a'{o}t \> histak\a'{a}lt \> hit\textipa{P}amc\a'{u} \> hi\v{s}tat\a'{a}ta \> lehit\textipa{Q}as\a'{e}q \> titqa\v{s}r\a'{i}\\
%hitgalg\a'{a}lti \> hitragz\a'{u} \> hitwakx\a'{a} \> mitlah\a'{e}vet \> mitmac\a'{e}\textipa{P}t \> mitrag\a'{e}z\\
%behitxa\v{s}\a'{e}v \> hit\textipa{Q}orer\a'{a} \> mamtaq\a'{i}m \> mitgal\a'{e}\c{c}et \> mit\textipa{P}am\a'{e}cet \> \v{s}emistarq\a'{i}m\\
%hitnah\a'{e}g \> hitpazr\a'{u} \> hitpoc\a'{e}c \> hi\v{s}tan\a'{a} \> lehitxab\a'{e}\textipa{P} \> \v{s}ehitpoc\a'{e}c\\
%histar\a'{a}qt \> hitpocec\a'{u} \> hitrag\a'{e}z \> hitya\v{s}\a'{e}v \> lehithap\a'{e}\textsubdot{k} \> titqalx\a'{i}\\
%hit\textipa{Q}anyen\a'{a} \> lehit\textipa{Q}acb\a'{e}n \> mitpan\a'{e}qet \> mitqa\v{s}\a'{e}r \> nitgalg\a'{e}l \> tistarq\a'{i}\\
%mistak\a'{e}l \> mitno\textipa{Q}\a'{e}a\textipa{Q} \> mitpa\textipa{Q}\a'{e}l \> nitlab\a'{e}\v{s} \> titgalgel\a'{i} \> \v{s}emitkad\a'{e}r\\
%hitparq\a'{a} \> lehitqa\v{s}\a'{e}r \> mitpar\a'{e}q \> mit\textipa{Q}aq\a'{e}\v{s}et \> titxal\a'{e}q \> titya\v{s}v\a'{i} \\
%\end{tabbing}
%\label{subfig:cl-hit-sample}
%}


\begin{figure}[t]
\begin{mdframed}
\centering
%\subfigure[]{
\begin{tabbing}
%\hspace*{14ex}\=\hspace*{14ex} \kill
%rony  \> tony \\
%%nice  dice 
%\end{tabbing}
%\label{t}
%} 
\hspace*{14ex}\= \hspace*{14ex}\=\hspace*{14ex}\=\hspace*{14ex}\=\hspace*{14ex}\=\hspace*{14ex} \kill
hamitnag\v{s}\a'{o}t \> hitgalgel\a'{a} \> hitnadn\a'{e}d \> lehistap\a'{e}r \> welehitra\textipa{P}\a'{o}t \> wemistak\a'{e}l\\
hithap\textsubdot{k}\a'{a} \> mitra\v{s}\a'{e}met \> titnagv\a'{i} \> titxat\a'{e}n \> wemistak\a'{e}let \> yitqalq\a'{e}l\\
hahitna\v{s}muy\a'{o}t \> histak\a'{a}lt \> hit\textipa{P}amc\a'{u} \> hi\v{s}tat\a'{a}ta \> lehit\textipa{Q}as\a'{e}q \> titqa\v{s}r\a'{i}\\
hitgalg\a'{a}lti \> hitragz\a'{u} \> hitwakx\a'{a} \> mitlah\a'{e}vet \> mitmac\a'{e}\textipa{P}t \> mitrag\a'{e}z\\
behitxa\v{s}\a'{e}v \> hit\textipa{Q}orer\a'{a} \> mamtaq\a'{i}m \> mitgal\a'{e}\c{c}et \> mit\textipa{P}am\a'{e}cet \> \v{s}emistarq\a'{i}m\\
hitnah\a'{e}g \> hitpazr\a'{u} \> hitpoc\a'{e}c \> hi\v{s}tan\a'{a} \> lehitxab\a'{e}\textipa{P} \> \v{s}ehitpoc\a'{e}c\\
histar\a'{a}qt \> hitpocec\a'{u} \> hitrag\a'{e}z \> hitya\v{s}\a'{e}v \> lehithap\a'{e}\textsubdot{k} \> titqalx\a'{i}\\
hit\textipa{Q}anyen\a'{a} \> lehit\textipa{Q}acb\a'{e}n \> mitpan\a'{e}qet \> mitqa\v{s}\a'{e}r \> nitgalg\a'{e}l \> tistarq\a'{i}\\
mistak\a'{e}l \> mitno\textipa{Q}\a'{e}a\textipa{Q} \> mitpa\textipa{Q}\a'{e}l \> nitlab\a'{e}\v{s} \> titgalgel\a'{i} \> \v{s}emitkad\a'{e}r\\
hitparq\a'{a} \> lehitqa\v{s}\a'{e}r \> mitpar\a'{e}q \> mit\textipa{Q}aq\a'{e}\v{s}et \> titxal\a'{e}q \> titya\v{s}v\a'{i}
\end{tabbing}
\label{fig:cl-hit}
\caption{60 words randomly selected from a 426-word cluster generated by Multimorph's MCMM 
at $s = 2$, and $\delta = 2$. These words are almost entirely of the \textit{hitpa`el} binyan.}
\end{mdframed}
\end{figure}
\begin{figure}[tb!]
\begin{mdframed}
\begin{tabbing}
\centering
\hspace*{17ex}\= \hspace*{17ex}\=\hspace*{17ex}\=\hspace*{17ex}\=\hspace*{17ex} \kill
\texttt{a<e} \, (1.0000) \> \texttt{e<a} \, (0.0280) \>  \texttt{b<e} \, (0.0202) \> \texttt{y<e} \, (0.0174) \> \texttt{\v{s}<e} \, (0.0091) \\ 
\texttt{m<e} \, (0.0080) \> \texttt{p<e} \, (0.0079) \> \texttt{d<t} \, (0.0065) \>\texttt{e<f} \, (0.0064) \> \texttt{e<h} \, (0.0055) 
\end{tabbing}
\label{fig:cl-hit-features}
\caption{The ten most active features in the cluster represented figure~\ref{fig:cl-hit}. The feature (surface-unit) activities are included in parentheses.}
\end{mdframed}
\end{figure}
%\label{subfig:cl-hit-sample}
%}
%\subfigure[The ten most active features in the cluster represented above. The feature (surface-unit) activities are included in parentheses.]{
%%\hrule 
%%\vspace{12pt}
%fff
%%\vspace{12pt}
%%\hrule
%}
%\end{center}





This \textit{hitpa\textipa{`}el} example demonstrates that Multimorph is capable of learning non-concatenative morphology. 
The \textit{a} and the \textit{e} in CaCeC are separated by a consonant. Among the words in \ref{fig:cl-hit}, 
the intervening C between the \textit{a} and \textit{e} varies. It follows that Multimorph is not merely recognizing 
continuous substrings that contain both \textit{a} and \textit{e}.


\section{Quantitative Results}
The quantitative results stem from the dual-paradigm evaluation method outlined in chapter~\ref{ch:eval}. 
There are thus two distinct bodies of quantitative results, one from the \emph{intrinsic} component, and 
the other from the \emph{extrinsic}. We turn first to the intrinsic results.
Whereas the qualitative analysis discussed in section~\ref{sec:qual} was performed 
manually, the quantitative evaluation was performed computationally.
While human eyes can be beneficial to evaluating the results of unsupervised learning, 
a human may be less effective at judging overall consistency. 

\subsection{Intrinsic Results}
\label{sec:intr-results}
The intrinsic results are displayed in tables~\ref{tab:intr-500} and \ref{tab:intr-1000}. 
These are the ``master" tables, so to speak, for the intrinsic results, wherein each row 
represents the intrinsic evaluation of a \emph{model}. 
Each \emph{model} is distinguished by its input data type (TS, TR, or O), its feature set, which depends 
on the values of $s$ and $\delta$, and the number of clusters of clusters it was permitted to accumulate. 
The evaluation metrics, discussed in chapter~\ref{ch:eval}, are \textbf{average cluster-wise purity} (Purity), 
\textbf{BCubed precision} (BP), \textbf{BCubed Recall} (BR), 
and the \textbf{F1-score} (F), i.e., the harmonic mean of BP and BR.  The tables also state the \emph{coverage} 
(Cov.) of each clustering, which is the number of words belonging to at least one cluster, and $K^{\prime}$, 
which the set of \emph{active} clusters, i.e., the clusters with at least one member. 
Each table  is associated with a $K$ cutoff value, either 500 or 1000, and each divided into three subtables, 
one for each of the three types of input data representation (DR).

One of the most salient observations regarding tables~\ref{tab:intr-500} and \ref{tab:intr-1000}
 is that the models trained on O, the orthographic data, performed just as well as as those trained on TS
  (the transcriptional datasets with stress marking), if not \emph{slightly better}. The average F-score for all O models 
 at $K=1000$ was 0.371. (Note that this is the average F-score computed over \emph{all} valuations of $s$
   and $\delta$, i.e., all rows in table~\ref{subtab:intr-O-1000}.) 
   The average TS F-score at $K=1000$ was 0.365. 
   These are substantially higher that average F-score of the TR models at $K=1000$, namely 0.308.
  Two important points emerge from this observation: 


\emph{First, the O models performed surprisingly well.} They performed as well as the TS models and substantially better than the
  TR models.
 This is a surprising result because, as discussed in chapter~\ref{ch:experi}, 
 the orthography of Hebrew is basically consonantal; that is, the alphabet lacks vowel symbols. 
It would not be unreasonable to expect the orthographic dataset to be the least efficacious of the three because it
lacks the information contributed by vowels, which would seem to be important information, as vowel patterns seem to be crucial to Hebrew's 
root-and-pattern morphology. However, the quantitative results in tables~\ref{tab:intr-500} and \ref{tab:intr-1000} 
show no real difference in performance between the O and TS models.
%(Their results do differ qualitatively, however, as discussed above in section~\ref{sec:qual}.) 
This suggests that vowel symbols \emph{per se} are not necessarily 
helpful to the task of inducing a model of Hebrew morphology. Indeed, 
they appear to have been more harmful than helpful in this study. It should be 
noted, however, that there are many possible ways to encode vowels in features, and 
there could be feature formats that encode vowels to better effect than the present study's.

\emph{Second, the TS models outperformed the TR models by a substantial margin.} That is, 
the transcriptions containing both stressed and unstressed vowels engendered better features than  the
transcriptions that lacked stress information. We might infer from this observation that stressed vowels are more informative and thus more useful than 
vowels whose stress is not specified.  It is important to note, however, that the alphabets used in this study were consisted solely of atomic unicode characters. That is, the stressed \textit{\'a}, 
for example, was not encoded as \textit{a} (\texttt{U+0061}) followed by the acute-accent combining character \texttt{U+0301}, but 
rather as the single \emph{precomposed} character{\texttt{U+00E1}. The stressed (or accented) vowels \textit{\'e}, \textit{\'o}, and \textit{\'u} were similarly each represented as a single, precomposed unicode character. 
Such atomic symbols are incapable of expressing inter-symbol relationships. 
For example, \textit{\'a} (\texttt{U+00E1}) and \textit{a} (\texttt{U+0061}) are completely distinct, as (un)related as the characters \textit{a} and \textit{t} as far as the unicode is concerned.
There is likewise
no relationship between the characters \texttt{U+00E1} (\textit{\'a}) and  
\texttt{U+00E9} (\textit{\'e}) or any other combination of stressed-vowel 
precomposed characters. There is nothing in these characters that represents 
categorical traits such as ``stressed" or ``unstressed" or ``vowel."  

In effect, therefore, to include stress marking was to increase the size of the 
alphabet from 29 to 34 atomic symbols, which
in turn resulted in a considerable increase in the number of features, as the number features depended 
directly on the size of the alphabet, as discussed in chapter~\ref{ch:experi}. It seems noteworthy 
that his increase in information apparently did not result in an
information overload. On the contrary, it led to models that were substantially 
better than the models trained on the stressless transcriptions.  
%It thus seems to have counteracted the decrease in performance that apparently 
%resulted from the inclusion of non-stressed vowels in the stressless transcriptional data. 


%\begin{table}
%\small
%\centering
%\subtable[Transcriptions with stress (TS), $K=500$ \label{subtab:intr-TS-500}]{
%\setlength{\extrarowheight}{6pt}
%\begin{tabular}{cc|ccccrr}
%$s$ & $\delta$ & Purity &  BPrc & BR & F & Cov. & $K^{\prime}$ \\ \hline\hline
%0 & 1 & 0.450 & 0.483 & 0.371 & 0.419 & 11585 & 88 \\%500 %TS %0_1_K1000_N12272_basic_181104_14-26_k-500
%0 & 2 & 0.479 & 0.329 & 0.504 & 0.386 & 11984 & 104 \\
%0 & 3 & 0.480 & 0.341 & 0.424 & 0.378 & 11348 & 75 \\ \hline %500 %TS %0_3_K1000_N12272_basic_181104_15-15_k-500
%2 & 1 & 0.880 & 0.429 & 0.440 & 0.434 & 12103 & 346 \\%500 %TS %2_1_K6000_N12272_basic_180621_23-57_k-500
%2 & 2 & 0.505 & 0.360 & 0.497 & 0.418 & 12076 & 74 \\%500 %TS %2_2_K6000_N12272_basic_180621_21-24_k-500
%2 & 3 & 0.503 & 0.317 & 0.513 & 0.392 & 12084 & 96 \\ \hline %500 %TS %2_3_K1000_N12272_basic_181014_23-53_k-500
%4 & 1 & 0.465 & 0.302 & 0.549 & 0.390 & 12164 & 81 \\%500 %TS %4_1_K6000_N12271_basic_180616_12-50_k-500
%4 & 2 & 0.487 & 0.328 & 0.473 & 0.388 & 12214 & 153 \\%500 %TS %4_2_K6000_N12271_basic_180616_13-06_k-500
%4 & 3 & 0.488 & 0.308 & 0.494 & 0.378 & 12121 & 165 \\ \hline 
%6 & 1 & 0.455 & 0.323 & 0.514 & 0.397 & 12101 & 82 \\%500 %TS %6_1_K6000_N12271_basic_180616_13-06_k-500
%6 & 2 & 0.494 & 0.336 & 0.510 & 0.405 & 11958 & 64 \\%500 %TS %6_2_K6000_N12272_basic_180620_02-37_k-500
%6 & 3 & 0.529 & 0.345 & 0.474 & 0.399 & 11858 & 76 \\ \hline \hline%500 %TS %6_3_K6000_N12271_basic_180621_07-59_k-500
% \multicolumn{2}{r|}{\textit{Avgs:}} & 0.518 & 0.350 & 0.480 & 0.399 & 11966 & 117 \\
%\end{tabular}
%}
%\subtable[Transcriptions, no stress marking (TR), $K=500$ \label{subtab:intr-TR-500}]{
%\setlength{\extrarowheight}{6pt}
%\begin{tabular}{cc|ccccrr}
%$s$ & $\delta$ & Purity &  BPrc & BR & F & Cov. & $K^{\prime}$ \\ \hline\hline
%0 & 1 & 0.460 & 0.366 & 0.366 & 0.365 & 11994 & 108 \\
%0 & 2 & 0.517 & 0.310 & 0.398 & 0.340 & 11748 & 74 \\
%0 & 3 & 0.416 & 0.243 & 0.571 & 0.341 & 12079 & 110 \\ \hline %500 %TR %0_3_K6000_N12222_basic_180626_18-27_k-500
%2 & 1 & 0.462 & 0.311 & 0.466 & 0.370 & 12182 & 116 \\
%2 & 2 & 0.456 & 0.269 & 0.583 & 0.368 & 12187 & 111 \\%500 %TR %2_2_K6000_N12222_basic_180626_18-27_k-500
%2 & 3 & 0.433 & 0.236 & 0.629 & 0.343 & 12194 & 106 \\ \hline %500 %TR %2_3_K1000_N12222_basic_181015_00-00_k-500
%4 & 1 & 0.426 & 0.259 & 0.599 & 0.361 & 12198 & 134 \\  %500 %TR %4_1_K6000_N12222_basic_180619_21-27_k-500
%4 & 2 & 0.441 & 0.279 & 0.573 & 0.374 & 12164 & 106 \\
%4 & 3 & 0.431 & 0.272 & 0.565 & 0.367 & 12149 & 102 \\ \hline %500 %TR %4_3_K6000_N12222_basic_180621_01-45_k-500
%6 & 1 & 0.429 & 0.236 & 0.639 & 0.345 & 12191 & 130 \\  %500 %TR %6_1_K6000_N12221_basic_180616_12-58_k-500
%6 & 2 & 0.445 & 0.263 & 0.594 & 0.364 & 12153 & 102 \\
%6 & 3 & 0.432 & 0.268 & 0.576 & 0.366 & 12078 & 66 \\ \hline \hline %500 %TR %6_3_K6000_N12222_basic_180621_01-59_k-500
% \multicolumn{2}{r|}{\textit{Avgs:}} & 0.446 & 0.276 & 0.547 & 0.359 & 12110 & 105 \\
%\end{tabular}
%}
%\subtable[Orthographic data (O), $K=500$ \label{subtab:intr-O-500}]{
%\setlength{\extrarowheight}{6pt}
%\begin{tabular}{cc|ccccrr}
%$s$ & $\delta$ & Purity &  BPrc & BR & F & Cov. & $K^{\prime}$ \\ \hline\hline
%0 & 1 & 0.431 & 0.348 & 0.152 & 0.212 & 10962 & 229 \\%500 %O %0_1_K6000_N11166_basic_180727_16-17_k-500
%0 & 2 & 0.422 & 0.295 & 0.247 & 0.269 & 11107 & 293 \\%500 %O %0_2_K6000_N11166_basic_180727_16-17_k-500
%0 & 3 & 0.435 & 0.283 & 0.325 & 0.303 & 11051 & 247 \\ \hline %500 %O %0_3_K6000_N11166_basic_180727_16-17_k-500
%1 & 1 & 0.559 & 0.447 & 0.553 & 0.494 & 4488 & 340 \\%500 %O %1_1_K6000_N11166_basic_180723_13-16_k-500
%1 & 2 & 0.462 & 0.392 & 0.567 & 0.463 & 4493 & 122 \\%500 %O %1_2_K6000_N11166_basic_180723_13-15_k-500
%1 & 3 & 0.488 & 0.344 & 0.583 & 0.433 & 4485 & 156 \\ \hline %500 %O %1_3_K1000_N11166_basic_181104_15-26_k-500
%2 & 1 & 0.405 & 0.362 & 0.524 & 0.428 & 3447 & 66 \\%500 %O %2_1_K6000_N11166_basic_180723_13-11_k-500
%2 & 2 & 0.422 & 0.329 & 0.602 & 0.426 & 3441 & 96 \\%500 %O %2_2_K6000_N11166_basic_180723_13-11_k-500
%2 & 3 & 0.441 & 0.298 & 0.622 & 0.403 & 3445 & 132 \\ \hline %500 %O %2_3_K1000_N11166_basic_181015_00-18_k-500
%3 & 1 & 0.371 & 0.330 & 0.575 & 0.419 & 3447 & 78 \\%500 %O %3_1_K6000_N11166_basic_180727_03-42_k-500
%3 & 2 & 0.396 & 0.303 & 0.582 & 0.399 & 3444 & 113 \\%500 %O %3_2_K6000_N11166_basic_180723_13-19_k-500
%3 & 3 & 0.436 & 0.280 & 0.603 & 0.382 & 3444 & 142 \\ \hline \hline %500 %O %3_3_K6000_N11166_basic_180727_16-15_k-500
% \multicolumn{2}{r|}{\textit{Avgs:}} & 0.439 & 0.334 & 0.495 & 0.386 & 5605 & 168 \\
%\end{tabular}
%}
%\caption{Intrinsic evaluation results at $K = 500$. In the table headers, $s$ and $\delta$ are the parameters for positional and precedence features, respectively. \textit{BP} and \textit{BR} stand for BCubed Precision and BCubed Recall, respectively, and \textit{Cov} is the \textit{coverage}, i.e., the number of words that are active members of at least one cluster.}
%\label{tab:intr-500}
%\end{table}
%
%
%\begin{table}
%\small
%\centering
%\subtable[Transcriptions with stress (TS), $K=1000$ \label{subtab:intr-TS-1000}]{
%\setlength{\extrarowheight}{6pt}
%\begin{tabular}{cc|ccccrr}
%$s$ & $\delta$ & Purity &  BPrc & BR & F & Cov. & $K^{\prime}$ \\ \hline\hline
%0 & 1 & 0.448 & 0.481 & 0.370 & 0.418 & 11613 & 90 \\%1000 %TS %0_1_K1000_N12272_basic_181104_14-26_k-1000
%0 & 2 & 0.486 & 0.397 & 0.438 & 0.417 & 11759 & 106 \\%1000 %TS %0_2_K1000_N12272_basic_181104_14-25_k-1000
%0 & 3 & 0.376 & 0.340 & 0.426 & 0.379 & 11691 & 999 \\ \hline %1000 %TS %0_3_K1000_N12272_basic_181104_15-15_k-1000
%2 & 1 & 0.650 & 0.421 & 0.442 & 0.431 & 12144 & 986 \\%1000 %TS %2_1_K6000_N12272_basic_180621_23-57_k-1000
%2 & 2 & 0.464 & 0.356 & 0.504 & 0.417 & 12103 & 989 \\%1000 %TS %2_2_K6000_N12272_basic_180621_21-24_k-1000
%2 & 3 & 0.488 & 0.315 & 0.513 & 0.391 & 12111 & 107 \\ \hline %1000 %TS %2_3_K1000_N12272_basic_181014_23-53_k-1000
%4 & 1 & 0.457 & 0.299 & 0.536 & 0.384 & 12193 & 103 \\%1000 %TS %4_1_K1000_N12272_basic_181104_06-06_k-1000
%4 & 2 & 0.577 & 0.367 & 0.459 & 0.408 & 12011 & 591 \\%1000 %TS %4_2_K1000_N12272_basic_181104_04-36_k-1000
%4 & 3 & 0.473 & 0.273 & 0.519 & 0.358 & 12242 & 297 \\ \hline %1000 %TS %4_3_K1000_N12272_basic_181015_00-21_k-1000
%6 & 1 & 0.648 & 0.295 & 0.552 & 0.385 & 12139 & 995 \\%1000 %TS %6_1_K6000_N12271_basic_180616_13-06_k-1000
%6 & 2 & 0.655 & 0.333 & 0.499 & 0.400 & 11908 & 1000 \\%1000 %TS %6_2_K6000_N12272_basic_180620_02-37_k-1000
%6 & 3 & 0.703 & 0.333 & 0.497 & 0.399 & 11877 & 999 \\ \hline \hline%1000 %TS %6_3_K1000_N12272_basic_181104_07-21_k-1000
% \multicolumn{2}{r|}{\textit{Avgs:}} & 0.535 & 0.351 & 0.480 & 0.399 & 11983 & 605 \\
%\end{tabular}
%}
%\subtable[Transcriptions, no stress marking (TR), $K=1000$ \label{subtab:intr-TR-1000}]{
%\setlength{\extrarowheight}{6pt}
%\begin{tabular}{cc|ccccrr}
%$s$ & $\delta$ & Purity &  BPrc & BR & F & Cov. & $K^{\prime}$ \\ \hline\hline
%0 & 1 & 0.553 & 0.356 & 0.346 & 0.350 & 11936 & 427 \\
%0 & 2 & 0.615 & 0.302 & 0.516 & 0.381 & 12041 & 994 \\%1000 %TR %0_2_K6000_N12222_basic_180623_01-57_k-1000
%0 & 3 & 0.547 & 0.230 & 0.593 & 0.331 & 12168 & 860 \\ \hline %1000 %TR %0_3_K6000_N12222_basic_180626_18-27_k-1000
%2 & 1 & 0.786 & 0.326 & 0.522 & 0.401 & 12205 & 966 \\%1000 %TR %2_1_K6000_N12222_basic_180628_05-27_k-1000
%2 & 2 & 0.445 & 0.264 & 0.588 & 0.365 & 12202 & 133 \\%1000 %TR %2_2_K6000_N12222_basic_180626_18-27_k-1000
%2 & 3 & 0.551 & 0.252 & 0.323 & 0.283 & 12111 & 75 \\ \hline %1000 %TR %2_3_K1000_N12222_basic_181014_23-53_k-1000
%4 & 1 & 0.812 & 0.257 & 0.601 & 0.360 & 12203 & 432 \\%1000 %TR %4_1_K6000_N12222_basic_180619_21-27_k-1000
%4 & 2 & 0.436 & 0.258 & 0.612 & 0.363 & 12177 & 109 \\%1000 %TR %4_2_K6000_N12222_basic_180626_18-27_k-1000
%4 & 3 & 0.426 & 0.235 & 0.637 & 0.344 & 12136 & 113 \\ \hline %1000 %TR %4_3_K6000_N12222_basic_180621_01-45_k-1000
%6 & 1 & 0.705 & 0.234 & 0.643 & 0.343 & 12189 & 886 \\%1000 %TR %6_1_K6000_N12221_basic_180616_12-58_k-1000
%6 & 2 & 0.431 & 0.253 & 0.597 & 0.356 & 12152 & 114 \\%1000 %TR %6_2_K6000_N12222_basic_180626_17-48_k-1000
%6 & 3 & 0.446 & 0.271 & 0.569 & 0.367 & 12082 & 72 \\ \hline \hline%1000 %TR %6_3_K6000_N12222_basic_180621_01-59_k-1000
% \multicolumn{2}{r|}{\textit{Avgs:}} & 0.563 & 0.270 & 0.546 & 0.354 & 12133 & 432 \\
%\end{tabular}
%}
%
%\subtable[Orthographic data (O), $K=1000$ \label{subtab:intr-O-1000}]{
%\setlength{\extrarowheight}{6pt}
%\begin{tabular}{cc|ccccrr}
%$s$ & $\delta$ & Purity &  BPrc & BR & F & Cov. & $K^{\prime}$ \\ \hline\hline
%0 & 1 & 0.565 & 0.348 & 0.152 & 0.212 & 11000 & 808 \\%1000 %O %0_1_K6000_N11166_basic_180727_16-17_k-1000
%0 & 2 & 0.427 & 0.295 & 0.248 & 0.269 & 11113 & 303 \\%1000 %O %0_2_K6000_N11166_basic_180727_16-17_k-1000
%0 & 3 & 0.439 & 0.280 & 0.326 & 0.302 & 11098 & 1000 \\ \hline %1000 %O %0_3_K6000_N11166_basic_180727_16-17_k-1000
%1 & 1 & 0.401 & 0.440 & 0.557 & 0.492 & 4490 & 1000 \\%1000 %O %1_1_K6000_N11166_basic_180727_03-24_k-1000
%1 & 2 & 0.385 & 0.384 & 0.571 & 0.459 & 4493 & 1000 \\%1000 %O %1_2_K6000_N11166_basic_180723_13-15_k-1000
%1 & 3 & 0.442 & 0.338 & 0.590 & 0.430 & 4489 & 385 \\ \hline %1000 %O %1_3_K1000_N11166_basic_181104_15-26_k-1000
%2 & 1 & 0.402 & 0.355 & 0.534 & 0.426 & 3447 & 1000 \\%1000 %O %2_1_K6000_N11166_basic_180727_03-16_k-1000
%2 & 2 & 0.596 & 0.328 & 0.602 & 0.424 & 3444 & 942 \\%1000 %O %2_2_K6000_N11166_basic_180723_13-11_k-1000
%2 & 3 & 0.443 & 0.295 & 0.623 & 0.400 & 3446 & 137 \\ \hline %1000 %O %2_3_K1000_N11166_basic_181015_00-18_k-1000
%3 & 1 & 0.361 & 0.325 & 0.596 & 0.421 & 3446 & 385 \\
%3 & 2 & 0.389 & 0.317 & 0.617 & 0.418 & 3444 & 101 \\%1000 %O %3_2_K6000_N11166_basic_180727_16-16_k-1000
%3 & 3 & 0.418 & 0.271 & 0.609 & 0.375 & 3443 & 155 \\ \hline \hline%1000 %O %3_3_K6000_N11166_basic_180727_16-15_k-1000
% \multicolumn{2}{r|}{\textit{Avgs:}} & 0.439 & 0.331 & 0.502 & 0.386 & 5613 & 601 \\
%\end{tabular}
%}
%\caption{Intrinsic evaluation results at $K = 1000$. In the table headers, $s$ and $\delta$ are the parameters for positional and precedence features, respectively. \textit{BP} and \textit{BR} stand for BCubed Precision and BCubed Recall, respectively, and \textit{Cov} is the \textit{coverage}, i.e., the number of words that are active members of at least one cluster.}
%\label{tab:intr-1000}
%\end{table}
%
\begin{table}[t]
\centering
\subtable[Transcriptions with Stress Marking (TS)]{
\footnotesize
\centering
%\caption{ Transcriptions with Stress Marking (TS)}
\setlength{\extrarowheight}{3pt}
\begin{tabular}{cc|ccccrr}
$s$ & $\delta$ & Purity &  BP & BR & F & Cov. & $K^{\prime}$ \\ \hline\hline
0 & 1 & 0.450 & 0.483 & 0.371 & 0.419 & 11585 & 88 \\%500 %TS %0_1_K1000_N12272_basic_181104_14-26_k-500
0 & 2 & 0.479 & 0.329 & 0.504 & 0.386 & 11984 & 104 \\
0 & 3 & 0.480 & 0.341 & 0.424 & 0.378 & 11348 & 75 \\ \hline %500 %TS %0_3_K1000_N12272_basic_181104_15-15_k-500
2 & 1 & 0.880 & 0.429 & 0.440 & 0.434 & 12103 & 346 \\%500 %TS %2_1_K6000_N12272_basic_180621_23-57_k-500
2 & 2 & 0.505 & 0.360 & 0.497 & 0.418 & 12076 & 74 \\%500 %TS %2_2_K6000_N12272_basic_180621_21-24_k-500
2 & 3 & 0.503 & 0.317 & 0.513 & 0.392 & 12084 & 96 \\ \hline %500 %TS %2_3_K1000_N12272_basic_181014_23-53_k-500
4 & 1 & 0.465 & 0.302 & 0.549 & 0.390 & 12164 & 81 \\%500 %TS %4_1_K6000_N12271_basic_180616_12-50_k-500
4 & 2 & 0.487 & 0.328 & 0.473 & 0.388 & 12214 & 153 \\%500 %TS %4_2_K6000_N12271_basic_180616_13-06_k-500
4 & 3 & 0.488 & 0.308 & 0.494 & 0.378 & 12121 & 165 \\ \hline 
6 & 1 & 0.455 & 0.323 & 0.514 & 0.397 & 12101 & 82 \\%500 %TS %6_1_K6000_N12271_basic_180616_13-06_k-500
6 & 2 & 0.494 & 0.336 & 0.510 & 0.405 & 11958 & 64 \\%500 %TS %6_2_K6000_N12272_basic_180620_02-37_k-500
6 & 3 & 0.529 & 0.345 & 0.474 & 0.399 & 11858 & 76 \\ \hline \hline%500 %TS %6_3_K6000_N12271_basic_180621_07-59_k-500
 \multicolumn{2}{r|}{\textit{Avgs:}} & 0.518 & 0.350 & 0.480 & 0.399 & 11966 & 117 \\
\end{tabular}
}
 

%\subtable[Transcriptions, no stress marking (TR), $K=500$ \label{subtab:intr-TR-500}]{
%\tiny
%\setlength{\extrarowheight}{3pt}
%\frame{
%\frametitle{Intrinsic Evaluation, $K = 500$}
%\begin{table}
\subtable[Transcriptions, No Stress Marking (TR)]{
\footnotesize
\centering
\setlength{\extrarowheight}{3pt}
\begin{tabular}{cc|ccccrr}
$s$ & $\delta$ & Purity &  BP & BR & F & Cov. & $K^{\prime}$ \\ \hline\hline
0 & 1 & 0.460 & 0.366 & 0.366 & 0.365 & 11994 & 108 \\
0 & 2 & 0.517 & 0.310 & 0.398 & 0.340 & 11748 & 74 \\
0 & 3 & 0.416 & 0.243 & 0.571 & 0.341 & 12079 & 110 \\ \hline %500 %TR %0_3_K6000_N12222_basic_180626_18-27_k-500
2 & 1 & 0.462 & 0.311 & 0.466 & 0.370 & 12182 & 116 \\
2 & 2 & 0.456 & 0.269 & 0.583 & 0.368 & 12187 & 111 \\%500 %TR %2_2_K6000_N12222_basic_180626_18-27_k-500
2 & 3 & 0.433 & 0.236 & 0.629 & 0.343 & 12194 & 106 \\ \hline %500 %TR %2_3_K1000_N12222_basic_181015_00-00_k-500
4 & 1 & 0.426 & 0.259 & 0.599 & 0.361 & 12198 & 134 \\  %500 %TR %4_1_K6000_N12222_basic_180619_21-27_k-500
4 & 2 & 0.441 & 0.279 & 0.573 & 0.374 & 12164 & 106 \\
4 & 3 & 0.431 & 0.272 & 0.565 & 0.367 & 12149 & 102 \\ \hline %500 %TR %4_3_K6000_N12222_basic_180621_01-45_k-500
6 & 1 & 0.429 & 0.236 & 0.639 & 0.345 & 12191 & 130 \\  %500 %TR %6_1_K6000_N12221_basic_180616_12-58_k-500
6 & 2 & 0.445 & 0.263 & 0.594 & 0.364 & 12153 & 102 \\
6 & 3 & 0.432 & 0.268 & 0.576 & 0.366 & 12078 & 66 \\ \hline \hline %500 %TR %6_3_K6000_N12222_basic_180621_01-59_k-500
 \multicolumn{2}{r|}{\textit{Avgs:}} & 0.446 & 0.276 & 0.547 & 0.359 & 12110 & 105 \\
\end{tabular}
%\end{table} 
}
%\subtable[Orthographic data (O), $K=500$ \label{subtab:intr-O-500}]{


%\setlength{\extrarowheight}{6pt}
%\frame{
%\frametitle{Intrinsic Evaluation, $K=500$}
%\begin{table}

\subtable[Orthographic (O), $K=500$]{
\footnotesize
\centering
\setlength{\extrarowheight}{3pt}
\begin{tabular}{cc|ccccrr}
$s$ & $\delta$ & Purity &  BP & BR & F & Cov. & $K^{\prime}$ \\ \hline\hline
0 & 1 & 0.431 & 0.348 & 0.152 & 0.212 & 10962 & 229 \\%500 %O %0_1_K6000_N11166_basic_180727_16-17_k-500
0 & 2 & 0.422 & 0.295 & 0.247 & 0.269 & 11107 & 293 \\%500 %O %0_2_K6000_N11166_basic_180727_16-17_k-500
0 & 3 & 0.435 & 0.283 & 0.325 & 0.303 & 11051 & 247 \\ \hline %500 %O %0_3_K6000_N11166_basic_180727_16-17_k-500
1 & 1 & 0.559 & 0.447 & 0.553 & 0.494 & 4488 & 340 \\%500 %O %1_1_K6000_N11166_basic_180723_13-16_k-500
1 & 2 & 0.462 & 0.392 & 0.567 & 0.463 & 4493 & 122 \\%500 %O %1_2_K6000_N11166_basic_180723_13-15_k-500
1 & 3 & 0.488 & 0.344 & 0.583 & 0.433 & 4485 & 156 \\ \hline %500 %O %1_3_K1000_N11166_basic_181104_15-26_k-500
2 & 1 & 0.405 & 0.362 & 0.524 & 0.428 & 3447 & 66 \\%500 %O %2_1_K6000_N11166_basic_180723_13-11_k-500
2 & 2 & 0.422 & 0.329 & 0.602 & 0.426 & 3441 & 96 \\%500 %O %2_2_K6000_N11166_basic_180723_13-11_k-500
2 & 3 & 0.441 & 0.298 & 0.622 & 0.403 & 3445 & 132 \\ \hline %500 %O %2_3_K1000_N11166_basic_181015_00-18_k-500
3 & 1 & 0.371 & 0.330 & 0.575 & 0.419 & 3447 & 78 \\%500 %O %3_1_K6000_N11166_basic_180727_03-42_k-500
3 & 2 & 0.396 & 0.303 & 0.582 & 0.399 & 3444 & 113 \\%500 %O %3_2_K6000_N11166_basic_180723_13-19_k-500
3 & 3 & 0.436 & 0.280 & 0.603 & 0.382 & 3444 & 142 \\ \hline \hline %500 %O %3_3_K6000_N11166_basic_180727_16-15_k-500
 \multicolumn{2}{r|}{\textit{Avgs:}} & 0.439 & 0.334 & 0.495 & 0.386 & 5605 & 168 \\
\end{tabular}
}
\caption{Intrinsic evaluation results at $K = 500$. In the table headers, $s$ and $\delta$ are the parameters for positional and precedence features, respectively. \textit{BP} and \textit{BR} are BCubed Precision and BCubed Recall, respectively, and \textit{Cov} (coverage) is the number of words that are active members of at least one cluster.}
\label{tab:intr-500}
\end{table}
%}
%\frame{
%\frametitle{Intrinsic Evaluation, $K=1000$}
\begin{table}
\centering
\subtable[Transcriptions with Stress Markings (TS)\label{subtab:intr-TS-1000}]{ %(TS), $K=1000$ \label{subtab:intr-TS-1000}]{
\footnotesize
\setlength{\extrarowheight}{4pt}
\begin{tabular}{cc|ccccrr}
$s$ & $\delta$ & Purity &  BPrc & BR & F & Cov. & $K^{\prime}$ \\ \hline\hline
0 & 1 & 0.448 & 0.481 & 0.370 & 0.418 & 11613 & 90 \\%1000 %TS %0_1_K1000_N12272_basic_181104_14-26_k-1000
0 & 2 & 0.486 & 0.397 & 0.438 & 0.417 & 11759 & 106 \\%1000 %TS %0_2_K1000_N12272_basic_181104_14-25_k-1000
0 & 3 & 0.376 & 0.340 & 0.426 & 0.379 & 11691 & 999 \\ \hline %1000 %TS %0_3_K1000_N12272_basic_181104_15-15_k-1000
2 & 1 & 0.650 & 0.421 & 0.442 & 0.431 & 12144 & 986 \\%1000 %TS %2_1_K6000_N12272_basic_180621_23-57_k-1000
2 & 2 & 0.464 & 0.356 & 0.504 & 0.417 & 12103 & 989 \\%1000 %TS %2_2_K6000_N12272_basic_180621_21-24_k-1000
2 & 3 & 0.488 & 0.315 & 0.513 & 0.391 & 12111 & 107 \\ \hline %1000 %TS %2_3_K1000_N12272_basic_181014_23-53_k-1000
4 & 1 & 0.457 & 0.299 & 0.536 & 0.384 & 12193 & 103 \\%1000 %TS %4_1_K1000_N12272_basic_181104_06-06_k-1000
4 & 2 & 0.577 & 0.367 & 0.459 & 0.408 & 12011 & 591 \\%1000 %TS %4_2_K1000_N12272_basic_181104_04-36_k-1000
4 & 3 & 0.473 & 0.273 & 0.519 & 0.358 & 12242 & 297 \\ \hline %1000 %TS %4_3_K1000_N12272_basic_181015_00-21_k-1000
6 & 1 & 0.648 & 0.295 & 0.552 & 0.385 & 12139 & 995 \\%1000 %TS %6_1_K6000_N12271_basic_180616_13-06_k-1000
6 & 2 & 0.655 & 0.333 & 0.499 & 0.400 & 11908 & 1000 \\%1000 %TS %6_2_K6000_N12272_basic_180620_02-37_k-1000
6 & 3 & 0.703 & 0.333 & 0.497 & 0.399 & 11877 & 999 \\ \hline \hline%1000 %TS %6_3_K1000_N12272_basic_181104_07-21_k-1000
 \multicolumn{2}{r|}{\textit{Avgs:}} & 0.535 & 0.351 & 0.480 & 0.399 & 11983 & 605 \\
\end{tabular}
}
%\end{table}
%%\frame {
%\frametitle{Intrinsic Evaluation, $K=1000$}
%\begin{table}
\subtable[Transcriptions, No Stress Marking (TR)\label{subtab:intr-TR-1000}]{
\footnotesize
\centering
\setlength{\extrarowheight}{4pt}
\begin{tabular}{cc|ccccrr}
$s$ & $\delta$ & Purity &  BPrc & BR & F & Cov. & $K^{\prime}$ \\ \hline\hline
0 & 1 & 0.553 & 0.356 & 0.346 & 0.350 & 11936 & 427 \\
0 & 2 & 0.615 & 0.302 & 0.516 & 0.381 & 12041 & 994 \\%1000 %TR %0_2_K6000_N12222_basic_180623_01-57_k-1000
0 & 3 & 0.547 & 0.230 & 0.593 & 0.331 & 12168 & 860 \\ \hline %1000 %TR %0_3_K6000_N12222_basic_180626_18-27_k-1000
2 & 1 & 0.786 & 0.326 & 0.522 & 0.401 & 12205 & 966 \\%1000 %TR %2_1_K6000_N12222_basic_180628_05-27_k-1000
2 & 2 & 0.445 & 0.264 & 0.588 & 0.365 & 12202 & 133 \\%1000 %TR %2_2_K6000_N12222_basic_180626_18-27_k-1000
2 & 3 & 0.551 & 0.252 & 0.323 & 0.283 & 12111 & 75 \\ \hline %1000 %TR %2_3_K1000_N12222_basic_181014_23-53_k-1000
4 & 1 & 0.812 & 0.257 & 0.601 & 0.360 & 12203 & 432 \\%1000 %TR %4_1_K6000_N12222_basic_180619_21-27_k-1000
4 & 2 & 0.436 & 0.258 & 0.612 & 0.363 & 12177 & 109 \\%1000 %TR %4_2_K6000_N12222_basic_180626_18-27_k-1000
4 & 3 & 0.426 & 0.235 & 0.637 & 0.344 & 12136 & 113 \\ \hline %1000 %TR %4_3_K6000_N12222_basic_180621_01-45_k-1000
6 & 1 & 0.705 & 0.234 & 0.643 & 0.343 & 12189 & 886 \\%1000 %TR %6_1_K6000_N12221_basic_180616_12-58_k-1000
6 & 2 & 0.431 & 0.253 & 0.597 & 0.356 & 12152 & 114 \\%1000 %TR %6_2_K6000_N12222_basic_180626_17-48_k-1000
6 & 3 & 0.446 & 0.271 & 0.569 & 0.367 & 12082 & 72 \\ \hline \hline%1000 %TR %6_3_K6000_N12222_basic_180621_01-59_k-1000
 \multicolumn{2}{r|}{\textit{Avgs:}} & 0.563 & 0.270 & 0.546 & 0.354 & 12133 & 432 \\
\end{tabular}
}
%\begin{table}
%\scriptsize
%\centering
\subtable[Orthographic (O)\label{subtab:intr-O-1000}]{
\centering
\footnotesize
\setlength{\extrarowheight}{4pt}
\begin{tabular}{cc|ccccrr}
$s$ & $\delta$ & Purity &  BPrc & BR & F & Cov. & $K^{\prime}$ \\ \hline\hline
0 & 1 & 0.565 & 0.348 & 0.152 & 0.212 & 11000 & 808 \\%1000 %O %0_1_K6000_N11166_basic_180727_16-17_k-1000
0 & 2 & 0.427 & 0.295 & 0.248 & 0.269 & 11113 & 303 \\%1000 %O %0_2_K6000_N11166_basic_180727_16-17_k-1000
0 & 3 & 0.439 & 0.280 & 0.326 & 0.302 & 11098 & 1000 \\ \hline %1000 %O %0_3_K6000_N11166_basic_180727_16-17_k-1000
1 & 1 & 0.401 & 0.440 & 0.557 & 0.492 & 4490 & 1000 \\%1000 %O %1_1_K6000_N11166_basic_180727_03-24_k-1000
1 & 2 & 0.385 & 0.384 & 0.571 & 0.459 & 4493 & 1000 \\%1000 %O %1_2_K6000_N11166_basic_180723_13-15_k-1000
1 & 3 & 0.442 & 0.338 & 0.590 & 0.430 & 4489 & 385 \\ \hline %1000 %O %1_3_K1000_N11166_basic_181104_15-26_k-1000
2 & 1 & 0.402 & 0.355 & 0.534 & 0.426 & 3447 & 1000 \\%1000 %O %2_1_K6000_N11166_basic_180727_03-16_k-1000
2 & 2 & 0.596 & 0.328 & 0.602 & 0.424 & 3444 & 942 \\%1000 %O %2_2_K6000_N11166_basic_180723_13-11_k-1000
2 & 3 & 0.443 & 0.295 & 0.623 & 0.400 & 3446 & 137 \\ \hline %1000 %O %2_3_K1000_N11166_basic_181015_00-18_k-1000
3 & 1 & 0.361 & 0.325 & 0.596 & 0.421 & 3446 & 385 \\
3 & 2 & 0.389 & 0.317 & 0.617 & 0.418 & 3444 & 101 \\%1000 %O %3_2_K6000_N11166_basic_180727_16-16_k-1000
3 & 3 & 0.418 & 0.271 & 0.609 & 0.375 & 3443 & 155 \\ \hline \hline%1000 %O %3_3_K6000_N11166_basic_180727_16-15_k-1000
 \multicolumn{2}{r|}{\textit{Avgs:}} & 0.439 & 0.331 & 0.502 & 0.386 & 5613 & 601 \\
\end{tabular}
}
%\end{table}
\caption{Intrinsic evaluation results at $K = 1000$. The headers $s$ and $\delta$ are the parameters for positional and precedence features, respectively. \textit{BP} and \textit{BR} stand for BCubed Precision and Recall, and \textit{Cov} (coverage) is the number of words that are active members of at least one cluster.}
\label{tab:intr-1000}
\end{table}


\subsection{Extrinsic Results}
\label{sec:extr-results}
The extrinsic evaluation was itself a sort of experiment, one that sought to evaluate Multimorph according to its 
ability to positively influence a downstream application. It involved a four-stage procedure that is described in 
chapter~\ref{ch:eval}. 
Basically, this process compressed Multimorph's input words by substituting original characters with atomic 
representations of morphs (i.e., morphs represented as single characters). The resulting file was then fed to 
Morfessor, the idea being to give Morfessor a pre-analyzed (or pre-digested, as it were) list of input words and
 see if Morfessor was able perform better on the pre-analyzed, i.e., compressed, data than on ordinary input, i.e., 
 the control group, which was in this case the original Hebrew wordlists, untouched by Multimorph.
 
Table~\ref{tab:extr-results-1000} shows the results of the extrinsic evaluation at $K=1000$. 
Morfessor clearly did not perform 
better on the data material that was preprocessed by Multimorph, as the control 
results are always better than the corresponding 
experimental results. But even so, experimental and control results do seem to be \emph{comparable}; that is,
the experimental and control results do not seem large enough to suggest a \emph{fundamental} inferiority in the 
experimental models. One must bear in mind that this 4-stage process is both new---it was executed for the first time in this study---and highly complex. The process's results in table~\ref{tab:extr-results-1000} are close enough to 
the control results at this early in its development to indicate promise, a potential for considerable improvement.
%comparable. the differences between the control and experimental do not appear large enough to   experimental and control results are 
%comparable. 
%Interestingly, the scores of the orthographic (O) models come closest to matching the control results.

The scores of the orthographic (O) models come closest to matching the control results. Recall that the orthographic
also had surprisingly good extrinsic results (see section~\ref{sec:extr-results}). The relative success of the orthographic models in the extrinsic component of the
evaluation reinforces the notion that there really is something about the orthographic representation that
is inherently and unexpectedly beneficial---that more and/or better information is present in the orthographic representation than what seems immediately apparent.
%orthographic data that is inherently beneficial.

The reader may have noticed that the tables contain the 
results of many control datasets, one for each experimental model, in fact. This may seem strange 
because the parameters $s$ and $\delta$ were irrelevant to the control data, untouched as it was 
by Multimorph. This is indeed true, but the actual reason for the many control models is 
that each experimental model yielded a different word coverage. The control datasets were made 
to have the same words as their corresponding models covered. 

%\begin{table}[htb]
%\subtable[Transcriptions with stress (TS), $K = 1000$\label{subtab:extr-1000-ts}]{
%\small
%\setlength{\extrarowheight}{6pt}
%\begin{tabular}{cc|ccc|ccc}
%\multicolumn{2}{c}{} & \multicolumn{3}{c}{4-stage process} & \multicolumn{3}{c}{Control} \\
%0 & 1 & 0.40 & 0.37 & 0.39 & 0.78 & 0.60 & 0.68 \\
%0 & 2 & 0.37 & 0.35 & 0.36 & 0.77 & 0.60 & 0.68 \\
%0 & 3 & 0.50 & 0.39 & 0.44 & 0.77 & 0.60 & 0.67 \\\hline
%2 & 1 & 0.50 & 0.48 & 0.49 & 0.77 & 0.60 & 0.67 \\
%2 & 2 & 0.40 & 0.40 & 0.40 & 0.78 & 0.59 & 0.67 \\
%2 & 3 & 0.46 & 0.42 & 0.44 & 0.78 & 0.59 & 0.67 \\\hline
%4 & 1 & 0.54 & 0.48 & 0.51 & 0.78 & 0.59 & 0.67 \\
%4 & 2 & 0.54 & 0.46 & 0.50 & 0.79 & 0.61 & 0.68 \\
%4 & 3 & 0.48 & 0.38 & 0.42 & 0.78 & 0.59 & 0.68 \\ \hline
%6 & 1 & 0.55 & 0.50 & 0.52 & 0.78 & 0.60 & 0.68 \\
%6 & 2 & 0.58 & 0.49 & 0.53 & 0.77 & 0.59 & 0.67 \\ \hline\hline
%\multicolumn{2}{r|}{\textit{Avgs:}} & 0.48 & 0.43 & 0.45 & 0.78 & 0.60 & 0.67 \\
%\end{tabular}
%}
%\subtable[Transcriptions with stress (TR), $K = 1000$\label{subtab:extr-1000-tr}]{
%\small
%\setlength{\extrarowheight}{6pt}
%\begin{tabular}{cc|ccc|ccc}
%\multicolumn{2}{c}{} & \multicolumn{3}{c}{4-stage process} & \multicolumn{3}{c}{Control} \\
%0 & 1 & 0.36 & 0.47 & 0.40 & 0.61 & 0.68 & 0.62 \\
%0 & 2 & 0.46 & 0.41 & 0.43 & 0.79 & 0.63 & 0.70 \\
%0 & 3 & 0.50 & 0.37 & 0.43 & 0.80 & 0.64 & 0.71 \\ \hline
%2 & 1 & 0.50 & 0.46 & 0.48 & 0.81 & 0.65 & 0.72 \\
%2 & 2 & 0.44 & 0.40 & 0.42 & 0.81 & 0.64 & 0.71 \\
%2 & 3 & 0.31 & 0.56 & 0.40 & 0.40 & 0.73 & 0.52 \\ \hline
%4 & 1 & 0.51 & 0.47 & 0.49 & 0.80 & 0.65 & 0.72 \\
%4 & 2 & 0.46 & 0.42 & 0.44 & 0.80 & 0.64 & 0.71 \\
%4 & 3 & 0.47 & 0.43 & 0.45 & 0.79 & 0.63 & 0.71 \\ \hline
%6 & 1 & 0.53 & 0.49 & 0.51 & 0.81 & 0.64 & 0.72 \\
%6 & 2 & 0.51 & 0.43 & 0.47 & 0.81 & 0.64 & 0.71 \\
%6 & 3 & 0.59 & 0.46 & 0.52 & 0.79 & 0.64 & 0.71 \\ \hline\hline
% \multicolumn{2}{r|}{\textit{Avgs:}} & 0.48 & 0.44 & 0.46 & 0.76 & 0.65 & 0.69 \\
%\end{tabular}
%}
%\bigskip
%\subtable[Orthographic (O), $K = 1000$ \label{subtab:extr-1000-0}]{
%\small
%\setlength{\extrarowheight}{6pt}
%\begin{tabular}{cc|ccc|ccc}
%\multicolumn{2}{c}{} & \multicolumn{3}{c}{4-stage process} & \multicolumn{3}{c}{Control} \\
%$s$ & $\delta$ & Prc & R & F & Prc & R & F \\ \hline\hline
%0 & 1 & 0.69 & 0.42 & 0.52 & 0.83 & 0.67 & 0.75 \\
%0 & 2 & 0.67 & 0.31 & 0.42 & 0.84 & 0.68 & 0.75 \\
%0 & 3 & 0.67 & 0.28 & 0.39 & 0.83 & 0.68 & 0.75 \\ \hline
%1 & 1 & 0.49 & 0.56 & 0.52 & 0.59 & 0.64 & 0.61 \\
%1 & 2 & 0.53 & 0.48 & 0.51 & 0.60 & 0.64 & 0.62 \\
%1 & 3 & 0.72 & 0.30 & 0.42 & 0.59 & 0.64 & 0.61 \\ \hline
%2 & 1 & 0.47 & 0.62 & 0.53 & 0.57 & 0.71 & 0.63 \\
%2 & 2 & 0.46 & 0.59 & 0.52 & 0.58 & 0.72 & 0.64 \\
%2 & 3 & 0.55 & 0.56 & 0.56 & 0.59 & 0.73 & 0.65 \\ \hline
%3 & 1 & 0.46 & 0.59 & 0.51 & 0.58 & 0.73 & 0.65 \\
%3 & 2 & 0.43 & 0.56 & 0.49 & 0.60 & 0.70 & 0.65 \\
%3 & 3 & 0.49 & 0.58 & 0.53 & 0.58 & 0.74 & 0.65 \\ \hline\hline
% \multicolumn{2}{r|}{\textit{Avgs:}} & 0.55 & 0.49 & 0.49 & 0.65 & 0.69 & 0.66 \\
%\end{tabular}
%}
%\caption{Extrinsic evaluation results at $K = 1000$.}
%\label{tab:extr-results-1000}
%\end{table}

%\frame{
%\frametitle{Extrinsic Evaluation, $K = 1000$} %\label{subtab:extr-1000-0}]{

\begin{table}
%\scriptsize
\centering
\setlength{\extrarowheight}{3pt}
%\caption{ Transcriptions with Stress Marking (TS)}
\subtable[Transcriptions with stress (TS), $K = 1000$\label{subtab:extr-1000-ts}]{
\footnotesize
\centering
\begin{tabular}{cc|ccc|ccc}
\multicolumn{2}{c}{} & \multicolumn{3}{c}{4-stage process} & \multicolumn{3}{c}{Control} \\
0 & 1 & 0.40 & 0.37 & 0.39 & 0.78 & 0.60 & 0.68 \\
0 & 2 & 0.37 & 0.35 & 0.36 & 0.77 & 0.60 & 0.68 \\
0 & 3 & 0.50 & 0.39 & 0.44 & 0.77 & 0.60 & 0.67 \\\hline
2 & 1 & 0.50 & 0.48 & 0.49 & 0.77 & 0.60 & 0.67 \\
2 & 2 & 0.40 & 0.40 & 0.40 & 0.78 & 0.59 & 0.67 \\
2 & 3 & 0.46 & 0.42 & 0.44 & 0.78 & 0.59 & 0.67 \\\hline
4 & 1 & 0.54 & 0.48 & 0.51 & 0.78 & 0.59 & 0.67 \\
4 & 2 & 0.54 & 0.46 & 0.50 & 0.79 & 0.61 & 0.68 \\
4 & 3 & 0.48 & 0.38 & 0.42 & 0.78 & 0.59 & 0.68 \\ \hline
6 & 1 & 0.55 & 0.50 & 0.52 & 0.78 & 0.60 & 0.68 \\
6 & 2 & 0.58 & 0.49 & 0.53 & 0.77 & 0.59 & 0.67 \\ 
%0 & 1 & 0.40 & 0.37 & 0.39 & 0.78 & 0.60 & 0.68 \\
%0 & 2 & 0.37 & 0.35 & 0.36 & 0.77 & 0.60 & 0.68 \\
%0 & 3 & 0.50 & 0.39 & 0.44 & 0.77 & 0.60 & 0.67 \\
%2 & 1 & 0.50 & 0.48 & 0.49 & 0.77 & 0.60 & 0.67 \\
%2 & 2 & 0.40 & 0.40 & 0.40 & 0.78 & 0.59 & 0.67 \\
%2 & 3 & 0.46 & 0.42 & 0.44 & 0.78 & 0.59 & 0.67 \\
%4 & 1 & 0.54 & 0.48 & 0.51 & 0.78 & 0.59 & 0.67 \\
%4 & 2 & 0.54 & 0.46 & 0.50 & 0.79 & 0.61 & 0.68 \\
%4 & 3 & 0.48 & 0.38 & 0.42 & 0.78 & 0.59 & 0.68 \\
%6 & 1 & 0.55 & 0.50 & 0.52 & 0.78 & 0.60 & 0.68 \\
%6 & 2 & 0.58 & 0.49 & 0.53 & 0.77 & 0.59 & 0.67 \\
6 & 3 & 0.53 & 0.44 & 0.48 & 0.76 & 0.59 & 0.66 \\ \hline\hline
\multicolumn{2}{r|}{\textit{Avgs:}}  & 0.49 & 0.43 & 0.46 & 0.77 & 0.60 & 0.67 \\
%\multicolumn{2}{r|}{\textit{Avgs:}} & 0.48 & 0.43 & 0.45 & 0.78 & 0.60 & 0.67 \\
\end{tabular}
}
%\frame{
%\frametitle{Extrinsic Evaluation, $K = 1000$} %\label{subtab:extr-1000-0}]{
%\begin{table}

\setlength{\extrarowheight}{3pt}
%\caption{ Transcriptions, No Stress Marking (TR)}
\subtable[Transcriptions with stress (TR), $K = 1000$\label{subtab:extr-1000-tr}]{
\footnotesize
\centering
\begin{tabular}{cc|ccc|ccc}
\multicolumn{2}{c}{} & \multicolumn{3}{c}{4-stage process} & \multicolumn{3}{c}{Control} \\
0 & 1 & 0.36 & 0.47 & 0.40 & 0.61 & 0.68 & 0.62 \\
0 & 2 & 0.46 & 0.41 & 0.43 & 0.79 & 0.63 & 0.70 \\
0 & 3 & 0.50 & 0.37 & 0.43 & 0.80 & 0.64 & 0.71 \\ \hline
2 & 1 & 0.50 & 0.46 & 0.48 & 0.81 & 0.65 & 0.72 \\
2 & 2 & 0.44 & 0.40 & 0.42 & 0.81 & 0.64 & 0.71 \\
2 & 3 & 0.31 & 0.56 & 0.40 & 0.40 & 0.73 & 0.52 \\ \hline
4 & 1 & 0.51 & 0.47 & 0.49 & 0.80 & 0.65 & 0.72 \\
4 & 2 & 0.46 & 0.42 & 0.44 & 0.80 & 0.64 & 0.71 \\
4 & 3 & 0.47 & 0.43 & 0.45 & 0.79 & 0.63 & 0.71 \\ \hline
6 & 1 & 0.53 & 0.49 & 0.51 & 0.81 & 0.64 & 0.72 \\
6 & 2 & 0.51 & 0.43 & 0.47 & 0.81 & 0.64 & 0.71 \\
6 & 3 & 0.59 & 0.46 & 0.52 & 0.79 & 0.64 & 0.71 \\ \hline\hline
 \multicolumn{2}{r|}{\textit{Avgs:}} & 0.48 & 0.44 & 0.46 & 0.76 & 0.65 & 0.69 \\
\end{tabular}
}
%\frame{
%\frametitle{Extrinsic Evaluation, $K = 1000$} %\label{subtab:extr-1000-0}]{
%\begin{table}

\setlength{\extrarowheight}{3pt}
%\caption{ Orthographic (O)}
\subtable[Orthographic (O), $K = 1000$\label{subtab:extr-1000-o}]{
\footnotesize
\centering
\begin{tabular}{cc|ccc|ccc}
\multicolumn{2}{c}{} & \multicolumn{3}{c}{4-stage process} & \multicolumn{3}{c}{Control} \\
$s$ & $\delta$ & Prc & R & F & Prc & R & F \\ \hline\hline
0 & 1 & 0.69 & 0.42 & 0.52 & 0.83 & 0.67 & 0.75 \\
0 & 2 & 0.67 & 0.31 & 0.42 & 0.84 & 0.68 & 0.75 \\
0 & 3 & 0.67 & 0.28 & 0.39 & 0.83 & 0.68 & 0.75 \\ \hline
1 & 1 & 0.49 & 0.56 & 0.52 & 0.59 & 0.64 & 0.61 \\
1 & 2 & 0.53 & 0.48 & 0.51 & 0.60 & 0.64 & 0.62 \\
1 & 3 & 0.72 & 0.30 & 0.42 & 0.59 & 0.64 & 0.61 \\ \hline
2 & 1 & 0.47 & 0.62 & 0.53 & 0.57 & 0.71 & 0.63 \\
2 & 2 & 0.46 & 0.59 & 0.52 & 0.58 & 0.72 & 0.64 \\
2 & 3 & 0.55 & 0.56 & 0.56 & 0.59 & 0.73 & 0.65 \\ \hline
3 & 1 & 0.46 & 0.59 & 0.51 & 0.58 & 0.73 & 0.65 \\
3 & 2 & 0.43 & 0.56 & 0.49 & 0.60 & 0.70 & 0.65 \\
3 & 3 & 0.49 & 0.58 & 0.53 & 0.58 & 0.74 & 0.65 \\ \hline\hline
 \multicolumn{2}{r|}{\textit{Avgs:}} & 0.55 & 0.49 & 0.49 & 0.65 & 0.69 & 0.66 \\
\end{tabular}
}
\caption{Extrinsic evaluation results at $K = 1000$.}%(Second Pass: Cluster-membership threshold set at 0.8 instead of 0.5.)}
\label{tab:extr-results-1000}
\end{table}