%\addtocontents{toc}{\protect\contentsline{chapter}{Curriculum Vitae}{}}
%\addcontentsline{toc}{chapter}{Curriculum Vitae}
%\chapter{Curriculum Vitae}
%\documentclass[12pt]{article}
%\usepackage{setspace}% http://ctan.org/pkg/setspace
%\usepackage{fancyhdr}
%\usepackage[margin=3cm]{geometry}
%% margin setup
%%\setlength{\evensidemargin}{0in}
%\usepackage{hyperref}
%\usepackage{url}
%\hypersetup{
%    bookmarks=true,         % show bookmarks bar?
%    unicode=false,          % non-Latin characters in Acrobat?s bookmarks
%    pdftoolbar=true,        % show Acrobat?s toolbar?
%    pdfmenubar=true,        % show Acrobat?s menu?
%    pdffitwindow=false,     % window fit to page when opened
%    pdfstartview={FitH},    % fits the width of the page to the window
%    pdftitle={My title},    % title
%    pdfauthor={Author},     % author
%    pdfsubject={Subject},   % subject of the document
%    pdfcreator={Creator},   % creator of the document
%    pdfproducer={Producer}, % producer of the document
%    pdfkeywords={keyword1, key2, key3}, % list of keywords
%    pdfnewwindow=true,      % links in new PDF window
%    colorlinks=true,       % false: boxed links; true: colored links
%    linkcolor=red,          % color of internal links (change box color with linkbordercolor)
%    citecolor=green,        % color of links to bibliography
%    filecolor=magenta,      % color of file links
%    urlcolor=blue           % color of external links
%}
%\setlength{\headheight}{0pt}
%\setlength{\headsep}{0pt}
%\setlength{\footskip}{11pt}
%%\setlength{\oddsidemargin}{0in}
%
%\setlength{\paperheight}{11in}
%\setlength{\paperwidth}{8.5in}
%\setlength{\topmargin}{-0.5in} % start text higher on the page
%%\setlength{\topmargin}{0.0in} 
%%\setlength{\textheight}{9in}
%%\setlength{\textwidth}{6.5in}
%\setlength{\topmargin}{0.0in}
%
%\setlength{\oddsidemargin}{0.25in}
%\setlength{\evensidemargin}{0.25in}
%\setlength{\textheight}{9in}
%\setlength{\textwidth}{6in}
%
%\setlength{\topskip}{0in}
%\setlength{\voffset}{0in}
%%\newlength{\outerbordwidth}
%%\pagestyle{plain}
%\nopagenumbers
%\pagestyle{fancy}
%\lhead{}
%\chead{}
%\rfoot{\vspace{8pt} \textit{A. Meyer, p. \thepage} }
%\cfoot{} % get rid of the page number 
%\renewcommand{\headrulewidth}{0pt}
%\renewcommand{\footrulewidth}{0pt}
%\raggedbottom
%\raggedright
%\usepackage{scrextend}
%\usepackage[svgnames]{xcolor}
%\usepackage{framed}
%\usepackage{tocloft}
%\usepackage{sectsty}
%\usepackage{enumitem}
%%\usepackage{cite}
%\usepackage{natbib}
%\usepackage{bibentry}
%\usepackage{hanging}
%\usepackage{lipsum}
%\chapterfont{\centering}
%\sectionfont{\centering}
%\subsectionfont{\centering}
%\usepackage{array, xcolor}
%\definecolor{lightgray}{gray}{0.8}
%\newcolumntype{L}{>{\raggedleft}p{0.14\textwidth}}
%\newcolumntype{R}{p{0.8\textwidth}}
%\newcommand\VRule{\color{lightgray}\vrule width 0.5pt}
%% margin setup
%%\setlength{\evensidemargin}{0in}
%%\setlength{\headheight}{0in}
%%\setlength{\headsep}{0in}
%%\setlength{\oddsidemargin}{0in}
%%\setlength{\paperheight}{11in}
%%\setlength{\paperwidth}{8.5in}
%%\setlength{\tabcolsep}{0in}
%%\setlength{\textheight}{9in}
%%\setlength{\textwidth}{6.5in}
%%\setlength{\topmargin}{0in}
%%\setlength{\topskip}{0in}
%%\setlength{\voffset}{0in}
%\nobibliography{my-cv-bib}
%%%\bibliographystyle{styles/aclnat}
%\newcommand\hangbib[1]{%
%	%\setlength{\leftmargin}{#1}}%
%    \smallskip\par\hangpara{1em}{1}\bibentry{#1}\smallskip\par %{indent}{afterline}
%}
%%
%\newenvironment{myindentpar}[1]%
% {\begin{list}{}%
%         {\setlength{\leftmargin}{#1}}%
%         \item[]%
% }
%{\end{list}}
% 
%\newcommand{\hangbib}[1]{\begin{verse} \bibentry{#1} \end{verse}}
%\nobibliography{my-cv-bib}
%\begin{document}
%\thispagestyle{empty}
%%\nobibliography*
%\moveleft.5\hoffset\centerline{\textbf{ Anthony J. Meyer}}
%\moveleft.5\hoffset\vbox{\hrule width 6in height 1pt}\smallskip 
%%\noindent\makebox[\linewidth]{\rule{6.5in}{0.5pt}}\smallskip 
%%\begin{tabular*}{6.5in}{l@{\extracolsep{\fill}}r}
%\moveleft.5\hoffset\centerline{\small{120 S. Kingston Drive, Apt. E54 $\cdot$ Bloomington, IN 47408}}
%\moveleft.5\hoffset\centerline{\small{antmeyer@indiana.edu $\cdot$ (419) 615-8460}}
%%\moveleft.5\hoffset\centerline{Bloomington, IN 47408}
%%\moveleft.5\hoffset\centerline{(419) 615-8460}
%%\end{tabular*}
\thispagestyle{empty}
%\begin{document}
%\thispagestyle{empty}
\begin{singlespace}
%\centerline{\textbf{ Anthony J. Meyer}}
\centerline{\textbf{Anthony Meyer}}
%\begin{center}
%\textbf{ Anthony J. Meyer} \\
%\vbox{\hrule width 6in height 1pt}
\bigskip 
%\noindent\makebox[\linewidth]{\rule{6.5in}{0.5pt}}\smallskip 
%\begin{tabular*}{6.5in}{l@{\extracolsep{\fill}}r}
%\centerline{\small 6373 State Route 65 $\cdot$ Leipsic, Ohio 45856 }
%\centerline{\small antmeyer@iu.edu $\cdot$ (419) 615-8460 }
%\end{center}
%\end{singlespace}
%\setlength{\leftmargini}{8mm}
%\setlength{\leftmarginii}{5mm}
%\setlength{\leftmarginiii}{8mm}


%\centerline{\textbf{ Highly Versatile Computational Linguist}}

%\textbf{Knowledgeable linguist} Worked on Prof. Cavar's   Regarded as an concerning matters of general linguistics and linguistic theory / Regarded as one of the group's primary linguistic experts.
%
%\textbf{Proficient computer programmer}
%\textbf{Highly versatile computational linguist} with strong backgrounds both in general linguistics and computer science. 
%Recognized among my (immediate) colleagues as an expert in computational morphology and machine learning. 
%Knowledgeable (and experienced in) algorithm analysis, code optimization, and parallel computing. Gave a tutorial on writing code for Graphical Processing Units (GPUs), which perform many computations in parallel rather than in series. 
%Known within the department as an expert in Hebrew linguistics and CL.
%What programming languages? 
%What software? FOMA HFST Flags Diacritcs(?) Lexicon building techniques, doing developing them, SpellChecker to compile lists, Fancy tricks, Extracting crap from Brown corpus. Extracting morphological classes.
%What about grammar engineering? What about named-entity recognition? What do I know about that? When have I worked on it?
%During the fall semester of the eventful academic year of 2015-2016, I worked on Prof. Cavar's LFG project. I was a key component of the group---one of the main guys where general linguistic knowledge was concerned. The go-to-guy, in fact. Have been working on puzzling out English's rich derivational morphology, figuring out how to encode it in a FST via FOMA.
%%Experienced, knowledeable incubatee.
%HFST - The Helsinki Finite State Toolkit
%XFST - The Xerox Finite State Toolkit
%FOMA
%What is HFST? ``Helsinki Finite-State Technology (HFST) is a computer programming library and set of utilities for natural language processing with finite-state automata and finite-state transducers."
%What is XFST? ``A general purpose utility for compiling and using finite-state networks.''
%What is FOMA? ``Foma is a compiler, programming language, and C library for constructing finite-state automata and transducers for various uses. It has specific support for many natural language processing applications such as producing morphological analyzers. \\ Foma is a free and open source finite-state toolkit created and maintained by Mans Hulden. It includes a compiler, programming language, and C library for constructing finite-state automata and transducers (FST's) for various uses, most typically Natural Language Processing uses such as morphological analysis"

%\onehalfspacing
%%\hangindent=1em \textbf{ Trained linguist} with special expertise in morphology, Hebrew and Semitic linguistics, and in particular the non-concatenative morphology of Semitic languages. Thoroughly trained in each of the core subfields of linguistics, i.e., phonetics, phonology, morphology, syntax, and semantics. Regarded by colleagues as an authority in matters of general linguistics and linguistic theory.  
%\hangindent=1em \textbf{ Trained linguist} with special expertise in morphology, Hebrew and Semitic linguistics, and in particular the non-concatenative morphology of Semitic languages. Thoroughly trained in each of the core subfields of linguistics, i.e., phonetics, phonology, morphology, syntax, and semantics. Regarded by colleagues as an authority in matters of general linguistics and linguistic theory.
%\vspace{16pt}
%
%%\hangindent=1em \textbf{ Proficient programmer} with expertise in computational morphology, machine learning,  
%%neural networks, auto-encoders, and finite-state models of natural language.
%%Able to analyze algorithms and optimize code.
%%Led a tutorial on parallel computing at Indiana University's computational linguistics discussion group.
%%Well-acquainted with classical NLP algorithms, e.g., the Viterbi algorithm and the CKY chart parser, having implemented several from scratch.
%%Have been working with Prof. Damir Cavar to develop an open-source parser based on Lexical Functional Grammar (LFG).
%\hangindent=1em \textbf{ Proficient programmer} with expertise in computational morphology, machine learning, neural networks, auto-encoders, and finite-state models of natural language. Experienced in analyzing algorithms and optimizing code. Led a tutorial on parallel computing at Indiana University's computational linguistics discussion group. Well-acquainted with classical NLP algorithms, e.g., the Viterbi algorithm and the CKY chart parser, having implemented several from scratch. 
%
%
%%\vspace{14pt}
%%\textbf{ Areas of Expertise}
%%%\begin{itemize}[label=\textbullet, leftmargin=*, itemsep=.01pt]
%%\begin{itemize} \itemsep0em
%%\item Able to program in a variety of languages, including Python, Cython, C, C++, Java, CUDA, OpenCL, PyCUDA, PyOpenCL, and OpenMP.
%%\item Parallel Computing via Graphical Processing Units.
%%\item Grammar Engineering through finite-state technology, i.e., finite-state machines (FSMs), particularly finite-state transducers (FSTs)
%%\item Writing Finite-state language-processing applications via toolkits like HFST (Helsinki Finite-State Toolkit) and FOMA.
%%\item Computational morphology
%%\item Machine Learning, particularly unsupervised machine learning
%%\item General linguistic theory
%%\end{itemize}
%%\vspace{8pt}
%
%\vspace{16pt}
%\textbf{ Professional Competencies}
%\begin{itemize} \itemsep0em
%\item Able to program in a variety of languages, including Python, Cython, C, C++, Java, CUDA, OpenCL, PyCUDA, PyOpenCL, and OpenMP.
%\item Parallel Computing via Graphical Processing Units.
%\item Grammar Engineering through finite-state technology, i.e., finite-state machines, particularly finite-state transducers.
%\item Writing Finite-state language-processing applications via toolkits like HFST (Helsinki Finite-State Toolkit) and FOMA.
%\item Knowledge representation frameworks, including OWL, RDF, and RDFS.
%\item Proteg\'e and WebProteg\'e.
%\end{itemize}
%\vspace{8pt}

%\newgeometry{textwidth=6.5in,textheight=9in}
%\begin{singlespace}
%\centerline{\textbf{EDUCATION}} \\
%\centerline{\textbf{Education}}
%\vspace{22pt}
%\centerline{ Education}
\vspace{8pt}
\centerline{\textbf{Education}}
%\vspace{11pt}
\begin{description}
\item \textbf{PhD, Linguistics, Concentration in Computational Linguistics}\hfill June 2019
%\vspace{-12pt}
\begin{description}
%\begin{myindentpar}{1em}
%\par\hangpara{1em}{1}
%\hspace{2em}
\item Indiana University, Bloomington, IN
%\hspace{1em}
%\hspace{2em}
\item Advisor: Prof. Markus Dickinson
%\hspace{1em}
%\hspace{2em}
\item Minor: Computer Science
%\end{myindentpar}
\end{description}
\vspace{8pt}
\item \textbf{MA, Computational Linguistics} \hfill August 2010
\begin{description}
%\vspace{1.5pt}
\item Indiana University, Bloomington, IN
\item Advisor: Prof. Markus Dickinson
\end{description}
\vspace{8pt}
\item \textbf{BA, Linguistics} \hfill June 2006
%\vspace{1.5pt}
\begin{description}
\item The Ohio State University, Columbus, OH
\item Graduated \textit{Magna cum Laude} with Honors in the Arts and Sciences
\item Minor: Hebrew
\end{description}
\end{description}
\thispagestyle{empty}
%\end{singlespace}
\vspace{12pt}
%\begin{singlespace}
\centerline{\textbf{Research Interests}}
%\centerline{ Research Interests}
%\textbf{Research}
%\vspace{8pt}
%\begin{itemize}[label=\textbullet, leftmargin=*, nolistsep]
\begin{itemize}[label=\textbullet, leftmargin=*, itemsep=.01pt]
\item Machine-learning approaches to Natural Language Processing (NLP)
\item Computational morphology, particularly the unsupervised learning of morphology
\item Non-morphemic approaches to morphology
\item Non-concatenative (e.g., root-and-pattern) morphological systems
\item Hebrew and Semitic linguistics
\item Knowledge Graphs
\item Harnessing parallel-computing to speed up complex NLP applications
\item Intelligent Computer Aided Language Learning (ICALL) %and the development ICALL-based educational applications
\end{itemize}
%\end{singlespace}
%Unsupervised learning of nonconcatenative morphology
%%\begin{spacing}{1.5}
%\begin{mdframed}
%\noindent My research focuses on the unsupervised learning of \textit{nonconcatenative} morphology. I am developing an approach that applies the Multiple Cause Mixture Model (MCMM), a variant of the classical autoencoder network, to the problem of identifying nonconcatenative morphological units.
%\end{mdframed}
%%\end{spacing}
\vspace{18pt}
%\begin{singlespace}
\centerline{\textbf{Skills}}
%\centerline{ Skills}
%\textbf{Skills}
%\vspace{8pt}
%\textit{Computing}:
%Computer programming (including web programming); algorithms analysis; linguistic analysis (semantics, syntax, morphology, phonology, and phonetics); relational database design \\
%\hangindent=1em 
\thispagestyle{empty}
\begin{description}
\item[Programming Languages:]
Python, Cython, C, C++, CUDA, OpenCL, PyCUDA, PyOpenCL, OpenMP, Java, Matlab, FOMA, HFST, Unix/Linux Shells 
%\vspace{6pt}
%\hangindent=1em \textit{Web Markup Languages}: 
\item[Web Markup Languages:] HTML, XML, OWL, RDF, RDFS
%\vspace{6pt}
%\hangindent=1em \textit{Selected Software Applications}: 
\item[Selected Software Applications:] Adobe Creative Suite, Eclipse, Microsoft Access, Proteg\'e, WebProteg\'e, Autocad
%\LaTeX, Wavesurfer \\
%\vspace{6pt}
%\hangindent=1em \textit{Analytical Skills}: 
\item[Analytical Skills:] Analysis of Algorithms; linguistic analysis at the levels of pragmatics, semantics, syntax, morphology, phonology, and phonetics
%\vspace{6pt}
%\hangindent=1em 
\item[Other Skills:] %Teaching at the university level; public speaking; graphic design and visual communication
Visual art (e.g., painting and drawing), design (including graphic design), draughtsmanship, and visual communication; creative writing
\end{description}
%\end{singlespace}
%\newpage
\vspace{21pt}
%\begin{spacing}{1.3}
%\begin{singlespace}
\centerline{\textbf{Research Experience}}
%\centerline{ Research Experience}
\vspace{12pt}
\noindent\textbf{Research Assistant} \hfill Spring 2013 \\
\hspace{2ex} Experigen: A platform for creating linguistic experiments \\
\hspace{2ex} \url{https://github.com/tlozoot/experigen} \\
\hspace{2ex} Phonology Lab, Department of Linguistics, Indiana University \\
\hspace{2ex} Supervisor: Prof. Michael Becker
%\end{spacing}
\begin{itemize} %\itemsep 0em
%%\begin{singlespace}
\item Researched techniques for developing tablet and smart-phone apps. Became familiar with app development software.
\item Developed a procedure for creating mobile-app versions of Experigen-created online phonological experiments. As apps, such experiments do not require the internet for functionality and can thus be used to gather data in remote locations where no internet is available.
\item Programmed apps using JavaScript and HTML.
%%\end{singlespace}
\end{itemize}
\vspace{10pt}
%\begin{spacing}{1.3}
\thispagestyle{empty}
\noindent\textbf{Research Assistant} \hfill May 2010--August 2011 \\
Hebrew Online Proficiency Evaluation (HOPE) \\
Borns Jewish Studies Program, Indiana University  \\
Supervisors: Prof. Markus Dickinson, Prof. Sandra K\"ubler, and Ayelet Weiss
\begin{itemize} %\itemsep 0em 
%%\begin{singlespace}
\item Designed and programmed key components of an automated Hebrew-language placement exam. 
\item Provided supervisors with weekly written progress reports.
\item Contributed to original research that was ultimately published.
%%\end{singlespace}
\end{itemize}
%\end{spacing}
\vspace{10pt}
%%\begin{singlespace}
\noindent\textbf{Intern and Research Assistant} \hfill Fall 2016--Spring 2018 \\
Free Linguistic Environment (FLE) Project \\
\url{https://gorilla.linguistlist.org/fle/} \\
The LINGUIST List, Indiana University \\
Supervisor: Prof. Damir \'Cavar
\begin{itemize} %\itemsep2pt
\item Contributed to the development of an open-source parser based on Lexical Functional Grammar. %(LFG) \\
%\vspace{2pt}
%\item Extracted a crapton of words from the effing Brown kisspus. Why did I do this? Oh F word! To expand the lexicon, of course.
%What did this contribute to the project? SERIOUSLY!? Well, I automated the process to a large extent, and in way, defined it, or at least gave it
%more definition that it had had previously. New Boundaries, or rather, New Defined Domain for Word Gathering: Brown Corpus. How did I decide which words to extract? \\
%\vspace{2pt}
%\item Went poop nearly every day in the decrepit Linguist List bathroom. \\
%\vspace{2pt}
\item Wrote various programs and scripts to the support the expansion of the parser's lexicon and morphological-analysis components. 
%\vspace{2pt}
%\item Worked out how to use FSTs to model English's derivational morphology, whereby new words are derived from other words. 
%Often, multiple rounds of derivation are possible, e.g., ( \textit{un}$_2$ $+$ ( \textit{friend}$_0$ $+$ \textit{ly}$_1$ ) ) $+$ \textit{ness}$_3$ $\to$ \textit{unfriendliness}. 
%\item Learned some OWL and Protege. Typed some words pertaining to the organization of information. Represented morphology (tried to, anyway) is semantic web-form.
%\item Wrote the 2017 annual \url{https://blog.linguistlist.org/uncategorized/incredible\-parrot\-speech\-decoded\-as\-300\-years\-old\-english\-dialect/}{April Fools' news spoof}  for the LINGUIST List blog. % \url{https://
\item Wrote the  annual April Fools' news spoof  for the LINGUIST List blog in 2017 and 2018. % \url{https://
%blog.linguistlist.org/uncategorized/incredible-parrot-speech-decoded-as-300-years-old-english-dialect/} 
\begin{description}
\item[2017:] ``Incredible Parrot Speech Decoded as 300-Year-Old English Dialect''
%\url{https://blog.linguistlist.org/uncategorized/incredible-parrot-speech-decoded-as-300-years-old-english-dialect/}{``Incredible Parrot Speech Decoded As 300 Years Old English Dialect''}
\item[2018:] ``Optimality Theory: the Future of the Justice System?''
%\href{url}{https://blog.linguistlist.org/fund-drive/optimality-theory-the-future-of-the-justice-system/}{Optimality Theory: the Future of the Justice System?}
\end{description}
\end{itemize}
%\end{singlespace
%\end{singlespace}
%\vspace{20pt}
\newpage
\thispagestyle{empty}
\centerline{\textbf{ Teaching Experience}}
%\centerline{ Teaching Experience}
\vspace{12pt}
\noindent\textbf{Associate Instructor} \hfill 2009-2010; 2011-2012 \\
\noindent{D}epartment of Linguistics, Indiana University

\vspace{6pt}
\noindent{Planned} and taught discussion-section lessons. Provided students with individualized assistance through office hours, appointments, and email. Graded students' work and maintained electronic records of grades.
\vspace{3pt}
\begin{itemize} \itemsep6pt
\item{LING-L103:} \textit{Introduction to the Study of Language} \hfill Spr 2010; 2011-2012 \\
Supervisor: Dr. Richard Janda \\
%\vspace{2pt}
\noindent Facilitated visual learning through power-point presentations and video clips. Designed and distributed worksheets to encourage active learning and classroom participation.
\item LING-L303: \textit{Introduction to Linguistic Analysis}
\hfill Fall 2009, Fall 2012 \\
Supervisor: Prof. Kenneth de Jong \\
%\vspace{2pt}
Introduced students to phonological, morphological, and syntactic analysis. Focused on the rationale of the analytical process to help students sharpen their problem-solving skills.
\end{itemize}
%%\end{singlespace}
\vspace{20pt}
%\centerline{\textbf{OTHER EXPERIENCE}}
\centerline{\textbf{ Other Experience}}
%\centerline{ Other Experience}
%\vspace{8pt}
%\textbf{Intern} \hfill Fall 2016 \\
%The LINGUIST List, Indiana University \\
%%\vspace{4pt}
%\begin{itemize} \itemsep2pt
%\item Worked on the development of an open-source parser based on Lexical Functional Grammar (LFG) \\
%\vspace{2pt}
%%\item Extracted a crapton of words from the effing Brown kisspus. Why did I do this? Oh F word! To expand the lexicon, of course.
%%What did this contribute to the project? SERIOUSLY!? Well, I automated the process to a large extent, and in way, defined it, or at least gave it
%%more definition that it had had previously. New Boundaries, or rather, New Defined Domain for Word Gathering: Brown Corpus. How did I decide which words to extract? \\
%%\vspace{2pt}
%%\item Went poop nearly every day in the decrepit Linguist List bathroom. \\
%%\vspace{2pt}
%\item Wrote various programs and scripts to the support the expansion of the parser's lexicon and morphological-analysis components \\
%\vspace{2pt}
%\item Worked out how to use FSTs to model English's derivational morphology, whereby new words are derived from other words. Often, multiple rounds of derivation are possible, e.g., ( \textit{un}$_2$ $+$ ( \textit{friend}$_0$ $+$ \textit{ly}$_1$ ) ) $+$ \textit{ness}$_3$ $\to$ \textit{unfriendliness}. 
%\end{itemize}
\vspace{6pt}
\noindent\textbf{Graduate Assistant} \hfill Fall 2014 \\
Department of Linguistics, Indiana University
%\vspace{4pt}
\begin{itemize} \itemsep2pt
\item Maintained the computers in the Computational Linguistics Lab. 
%\vspace{2pt}
\item Performed software updates. 
%\vspace{2pt}
\item Installed software packages as requested by professors and students.
%\vspace{2pt}
\item Designed a logo for IU's Computational Linguistics Program %, now visible on 
%the program's website. % \url{http://cl.indiana.edu}. \\
\end{itemize}
\thispagestyle{empty}
%\textbf{Associate Instructor} \hfill Spr 2010; Fall 2011--Spr 2012 \\
%\begin{itemize} \itemsep0em
%\item Planned and taught discussion-section lessons. Provided students with individualized assistance through office hours, appointments, and email.
%Facilitated visual learning through power-point presentations and video clips. Designed and distributed worksheets to encourage active learning and classroom participation.
%\item Introduced students to phonological, morphological, and syntactic analysis. Focused on the rationale of the analytical process to help students sharpen their problem-solving skills.
%\item Provided students with individualized assistance through office hours, appointments, and email.
%\item Graded students' work and maintained electronic records of grades.
%\end{itemize}

%\begin{itemize} \itemsep0em
%\item Planned and taught discussion-section lessons. Made power-point presentations to facilitate visual learning. Designed and distributed worksheets to encourage active learning and classroom participation.
%\item Introduced students to phonological, morphological, and syntactic analysis. Helped students sharpen their problem solving skills by focusing on the rationale of the analytical process.
%%\item Graded homework; administered and graded quizzes. 
%%\item Responded promptly to student emails. 
%\item Provided students with individualized assistance through office hours, appointments, and email.
%\item Graded students' work and maintained electronic records of grades.
%\end{itemize}
%\vspace{4pt}
%
%LING-L303 \textit{Introduction to Linguistic Analysis} \hfill Fall 2009; Fall 2012 \\
%Supervisor: Prof. Kenneth de Jong
%\begin{itemize} \itemsep0em
%\item Led discussion sections, during which students practiced their emerging linguistic problem solving solving skills.
%\item Demonstrated the process of doing phonological, morphological, and syntactic analysis, walking
%students through the rationale of the analytical process step by step.
%%\item Helped plan lessons, the main goal of which was to teach students how to perform phonological, morphological, and syntactic analyses. 
%\item Assisted students individually through email, office hours, and appointments. 
%\item Graded homework and quizzes; maintained records of grades in Microsoft Excel.
%\end{itemize}
%\vspace{12pt}

%\section*{Research Interests}
%     \begin{itemize}
%     	\setlength{\parsep}{0ex plus0ex minus0.5ex}
%         \setlength{\itemsep}{0ex plus0ex minus0.5ex}
%         \item Intelligent Computer Aided Language Learning (ICALL)
%         \item Computational morphology, particularly the unsupervised learning of morphology
%         \item Applications of machine learning to Natural Language Processing (NLP)
%     \end{itemize}


%\section*{Professional Experience}
%\begin{itemize}
%\item \textbf{Associate Instructor},Department of Linguistics, Indiana University
%     \begin{itemize}
%         \item \textbf{LING-L103}: \textit{Introduction to the Study of Language} \hfill Spring 2010; 2011--2012
%         
%         Supervisor: Dr. Richard Janda
%         
%         Planned and taught discussion-section lessons. Graded homework. Administered and graded quizzes. Responded promptly to student emails. Provided students with individualized assistance during office hours and appointments. Maintained records of students' grades in OnCourse. \\
%              \begin{itemize}
%                 \item Spring 2012: Responsible for 77 students (3 sections)\\
%                 \item Fall 2011: Responsible for 75 students (3 sections)\\
%                 \item Spring 2010: Responsible for 51 students (3 sections) \\
%              \end{itemize}
%         
%         
%         \item \textbf{LING-L303}: \textit{Introduction to Linguistic Analysis} \hfill Fall 2009
%         
%         Supervisor: Prof. Kenneth de Jong
%         
%         Taught two discussion sections, each meeting twice per week. Helped plan lessons, the main goal of which was to teach students how to perform phonological, morphological, and syntactic analyses. Assisted students individually during office hours and appointments. Graded homework and quizzes. Maintained records of grades in Microsoft Excel.\\
%     \end{itemize}
%
%\item \textbf{Research Assistant},
%Borns Jewish Studies Program, Indiana University 
%     \begin{itemize}
%          \item \textbf{Hebrew Online Proficiency Evaluation (HOPE)} \quad 2010--2011
%          
%          Supervisors: Dr. Markus Dickinson, Dr. Sandra K\"ubler, and Ayelet Weiss
%          
%          Designed and programmed key components of the system (an automated placement exam). Provided my supervisors with written weekly reports of my progress. Contributed to original research that was ultimately published.
%     \end{itemize}
%\end{itemize}

%\newpage
%\vspace{16pt}

%\nobibliography{my-cv-bib}
%\bibliographystyle{styles/aclnat}
\vspace{20pt}
%\centerline{\textbf{PUBLICATIONS}}
\centerline{ \textbf{Publications}}
%\centerline{ Publications}
\vspace{10pt} 
%\item{itemize} \itemsep0pt
\centerline{Conference Proceedings}
%\centerline{ Conference Proceedings}
%\vspace{-4pt}
%\begin{myindentpar}{1em}
\begin{description}
\item Damir Cavar, Joshua Herring, Anthony Meyer. 2018. Case Law Analysis using Deep NLP and Knowledge Graphs. 
In \emph{Proceedings of the LREC 2018: 1st Workshop on Language Resources and Technologies for the Legal Knowledge Graph (LegalKG)}. 12 May, 2018, Miyazaki, Japan.
\item Anthony Meyer and Markus Dickinson. 2016. A multilinear approach to the unsupervised
learning of morphology. In \emph{14th ACL Special Interest Group on Computational Morphology and
Phonology (SIGMORPHON 2016)}. Berlin, Germany.
\item Markus Dickinson, Sandra K\"{u}bler, and Anthony Meyer. 2012. Predicting learner levels for online
exercises of Hebrew. In \emph{The 7th Annual Workshop on the Innovative Use of NLP for Building
Educational Applications}, pages 95--104. Association for Computational Linguistics, Montreal,
Canada.
\item Mohammed Khan, Eric Baucom, Anthony Meyer, and Lwin Moe. 2011. Projecting Farsi POS
data to tag Pashto. In \emph{Proceedings of the Student Research Workshop associated with the 8th
International Workshop on Recent Advances in Natural Language Processing (RANLP '11)},
pages 25--32. Hissar, Bulgaria.
\end{description}
\vspace{-3pt}
\centerline{Poster} %\end{center}
%\centerline{ Poster} %\end{center}
\begin{description}
\item Markus Dickinson, Sandra K\"{u}bler, Ayelet Weiss, Anthony Meyer, Chris Riley, and Amber Smith.
2011. Hebrew Online Prociency Evaluation (HOPE). \emph{Workshop on the Automatic Analysis of
Learner Language}. Victoria, BC, May 18, 2011.
\end{description}
%\vspace{-1pt}
\thispagestyle{empty}
\centerline{Presentations}
\begin{description}
\item Anthony Meyer and Markus Dickinson. 2016b. \emph{A Multilinear Approach to the
Unsupervised Learning of Morphology.} CLingDing, the Computational Linguistics
Discussion Group at Indiana University, December 2, 2016.
http://cl.indiana.edu/wiki
\item  Anthony Meyer. 2016. \emph{Parallel Computing with PyCUDA.} CLingDing, the
Computational Linguistics Discussion Group at Indiana University, March 4,
2016. http://cl.indiana.edu/wiki
\item Anthony Meyer and Markus Dickinson. 2014. \emph{A Short Cython Tutorial.} CLingDing,
the Computational Linguistics Discussion Group at Indiana University, April 29,
2014. http://cl.indiana.edu/wiki
\item Anthony Meyer and Markus Dickinson. 2013. \emph{Unsupervised Morphological Learning
for Non-concatenative Morphology.} CLingDing, the Computational Linguistics
Discussion Group at Indiana University, November 6, 2013.
http://cl.indiana.edu/wiki
\end{description}
%\vspace{4pt}
%\hangbib{meyer:dickinson:2016}
%\hangbib{dickinson-et-al:2012}
%\hangbib{khan-et-al:2011}
%\vspace{8pt}
%\textbf{Poster} \\
%\vspace{4pt}
%\hangbib{poster:2011}
%\vspace{16pt}
%\centerline{\textbf{PRESENTATIONS}}
%\vspace{8pt}
%\hangbib{meyer:dickinson:pres:2016}
%\hangbib{meyer:pres:2016}
%\hangbib{meyer:pres:2014}
%\hangbib{meyer:dickinson:pres:2013}
\vspace{18pt}

%\section*{Publications}
%\begin{itemize}
%    \item \textbf{Conference Proceedings}
%      \begin{itemize}
%        \item \bibentry{dickinson-et-al:2012}
%        \item \bibentry{khan-et-al:2011}
%      \end{itemize}
%    \item \textbf{Poster}
%      \begin{itemize}
%        \item \bibentry{poster:2011}
%      \end{itemize}
%\end{itemize}

%\section*{Poster}
%\begin{itemize}
%\item \bibentry{poster:2011}
%\end{itemize}

\centerline{\textbf{ Memberships}}
%\centerline{ Memberships}
\vspace{8pt}
\centerline{Phi Beta Kappa Society}
\centerline{Golden Key International Honour Society}
%\centerline{Bloomington Watercolor Society}
\end{singlespace}

%\vspace{24pt}

%\section*{Memberships}
%\begin{itemize}
%\setlength{\parsep}{0ex plus0ex minus0.5ex}
%\setlength{\itemsep}{0ex plus0ex minus0.5ex}
%\item Phi Beta Kappa Society
%\item Golden Key International Honour Society
%\end{itemize}

%\centerline{\textbf{REFERENCES}}
%\vspace{8pt}
%
%%\begin{itemize}[label=\textbullet, leftmargin=*, nolistsep]
%\centerline{Markus Dickinson, Associate Professor of Linguistics, Indiana University}
%%\centerline{Balantine Hall 851, %\textit{Phone}: (812) 856-2535, 
%\centerline{\textit{Email}: \texttt{md7@indiana.edu}}
%\vspace{8pt}
%\centerline{Sandra K\"ubler, Associate Professor of Linguistics, Indiana University}
%\centerline{Balantine Hall 852, 
%\textit{Email}: \texttt{skuebler@indiana.edu}, \textit{Phone:} (812) 855-3268}
%\vspace{8pt}
%\centerline{Damir Cavar, Associate Professor of Computational Linguistics, Indiana University}
%\centerline{Balantine Hall 850, \textit{Email}: \texttt{dcavar@indiana.edu}}

%\end{itemize}
%\end{itemize}
%\end{itemize}
%\section*{References}
%\begin{itemize}
%\item Dr. Markus Dickinson, Assistant Professor of Linguistics, Indiana University \\
%Memorial Hall 317, \textit{Phone}: (812) 856-2535, \textit{Email}: \texttt{md7@indiana.edu}
%\item Dr. Sandra K\"ubler, Assistant Professor of Linguistics, Indiana University \\
%Memorial Hall 402, \textit{Phone}: (812) 855-3268, \textit{Email}: \texttt{skuebler@indiana.edu}
%\end{itemize}

%\begin{tabular}{L!{\VRule}R}
%2005--2007 & {\bf MSc in Computer Science, Great University, Country.}\vspace{5pt}\\
%2001--2005 & BSc in Life Science, Great University, Country.\\
%\end{tabular}
%\textbf{ Education}
%%\begin{itemize}
%%\item Indiana University
%	\begin{tabular*}{6.5in}{l@{\extracolsep{\fill}}r}
%		Indiana University & Bloomington, Indiana \\
%		Ph.D. in Computational Linguistics & 2013 (Expected)
%	\end{tabular*}
%%\item The Ohio State University
%%\end{itemize}

%\end{document}