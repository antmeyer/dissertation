\chapter{EXPERIMENTAL SETUP}

\section{Input Data}

\subsection{Orthographic Words}
I am using two primary wordlists: an orthographic and a transcribed wordlist. The latter serves as the basis for two wordlists, and we thus have a total of three wordlists: Orthographic Words (O), 
Transcribed Words with Stress Markings (TS), and Transribed Words \emph{without} Stress Markings (TR).
%A third wordlist is derived from the transcriptional wordlist. 

\textbf{Orthographic Words (O):} I obtained the 6656 words in O from the dataset used by \cite{daya-et-al:2008} in their study of automatic root identification. 
%It comprises 6656 unique words. 
These words are transliterations (Romanizations) of Standard Modern Hebrew spellings. Each character in the Hebrew alphabet is unambiguously mapped to a distinct character
in the Roman alphabet. Each Hebrew character is always mapped to the same Roman character.
%, even though it may 

\subsection{Transcribed Words}

\textbf{Transcribed Words:} The source of the transcribed data is the Berman longitudinal corpus \citep{berman-weissenborn:1991}, which is part of the Hebrew section of the CHILDES databse, a multilingual corpus of verbal interactions between young children and adults \citep{macwhinney:2000a}. Each utterance is both transcribed and morphologically analyzed. In the transcriptions, stressed vowels are marked with acute accent marks.  
   \begin{itemize}
   	\item \textbf{Transcribed Words with Stress Markings (TS)}: The wordlist TS consists of 12,416 unique transcribed words extracted from the Berman longitudinal corpus.
	%, I extracted the transcribed words present in the Berman longitudinal corpus. I removed duplicates so that each word in TS was unique. The pre-existing stress markings were left intact. TS contains %12,494 words. 
12,416 words
	\item \textbf{Transribed Words without Stress Markings (TR)}: I obtained TR by stripping the stress markings from the words in TS. That is, TR is simply TS without stress markings. TR contains 12,352 unique words, Each word in TR can be mapped to at least one word in TR by giving it appropriate accent marks.
%	,removing every stress marking from TS. This reduced the number of unique words from 12,494 to 12,445. 
	I will run my system on both TS and TR, but I do not expect the presence/absence of accent marks in the input data to make a significant difference where my system's performance is concerned.
\end{itemize}	



\section{Experimental Variables}
\subsection{Data Representation}
\subsection{Mixing Function}
\subsection{Objective Function}

\subsection{Features}
\subsubsection{Positional}
\subsubsection{Precedence}

%\section{Evaluation paradigms}
%\subsubsection{Intrinsic evaluation}
%\subsubsection{Extrinsic evaluation}
