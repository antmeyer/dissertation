\chapter{EXPERIMENTAL SETUP}

%\section{Introduction}
%The purpose of this chapter is to describe
%\begin{itemize}
%\item what I tested, or the experimental (or independent) variables. (and perhaps why 
%I chose to test these particular things). 
%\item what did I control for? Well, I manipulated 
%only one variable at a time, controlling the others, i.e., holding them constant. 
%\item How I tested these things, i.e., my methodology. \end{itemize}

%In this chapter, I shall describe the setup of my experiments, i.e., the steps I took to 
%``set them up," hence the word \emph{setup}. 
%These steps included:
%
%\begin{enumerate}
%\item Gathering/preparing input data sets: \\ My data source was the CHILDES Databse 
%[\dots describe \dots] %Tricky stuff. Babbling nonsense.
%[Example]. The CHILDES (CITE) databases consists of (transcribed) interactions between 
%children and adults, usually their parents (or perhaps other adults/caregivers.) 
%(Look this up, verify info, and CITE). Thus, a substantial portion of it consists of child speech, 
%which can be noisy. For example \dots

%\item Question: What does Multimorph take as input? What is the origin of the process? 
%What does it do with the input? \\
\section{Introduction}
This chapter will describe the experimental setup. It will describe experiments' data source as well as the steps taken to extract it and reformat it according to the needs of this study. It will also then motivate and enumerate the experimental (i.e., independent) variables addressed by this study.

Multimorph takes as input any list of words. The words can be of any
language, although Multimorph does expect the input strings to be composed
of alphabetic symbols. It also assumes that each symbol will be atomic rather 
than composite. 
For example the symbol \textsf{\textipa{\.*k}} can be represented in Unicode either 
as a sequence of two code points, namely \texttt{U+006B, U+0323} 
(where \texttt{U+0323}
corresponds to the dot), or as the \emph{single} code point \texttt{+U1E33}. 
Multimorph 
expects the latter, at least in its present form. I mention this because data source 
expressed 
such as characters in the former way, i.e., as sequences of code points. I thus had
to map the sequences onto atomic code points such as \texttt{U+0323} 
for \textsf{\textipa{\.*k}}.

%(the first corresponds to the base character and the
%latter to the diacritic), and (2), as the single code point \texttt{\u0161}. 
%or as the single code point. u"\u0161" Multimorph 
    %#k-dot
    %line = unicode(line.replace(u"\u006B\u0323", u"\u1E33"))
    %#s-v
    %line = unicode(line.replace(u"\u0073\u030C", u"\u0161"))
    %#s-dot
    %line= unicode(line.replace(u"\u0073\u0323", u"\u1E63"))
%nput. It also 
%delimited by newline characters.
%In its current configuration, Multimorph expects each word to 
%be on a new line, but this is trivial. The important point is that 
%Multimorph, like many other ULM systems, presumes that each word is contextless, 
%i.e., orthogonal to the other words in the list. (REF and CITE).

Now, when Multimorph reads the input wordlist, it maps each word onto
a sequence of 1s and 0s called a \emph{feature vector}, where each 1 or 0 is the value of a
particular feature.  %being \emph{values} of \emph{features}. 
We shall address the matter of features more fully in \ref{sec:features} below, as feature types
constitute some of this study's main experimental variables. We shall thus be particularly 
concerned with the question of what features to test.

%will be discussed further in section \ref{sec:features}.
%For the present, however, let us note that Multimorph requires a wordlist $W$. And $W$
%must be extracted from some source. We shall discuss this process in section 
%be extracted 

%In addition to the initial wordlist W, I need gold-standard data  
%\item What is the output?
%\item How do we tell \emph{how good} the output is? But this isn't really relevant to this 
%section, right? What this chapter is really concerned with is the input data, 
%its nature, source, and what I did to prepare it. It is also concerned with the experimental 
%variables--what I chose to test and why. (These, of course, are directly related to 
%my research questions. Each experimental variable represents an attempt to answer a 
%research question or part of one. Finally, it is concerned with how the experimental 
%variables were tested. 
%\end{enumerate}
\section{Data Source: The Berman Longitudinal Corpus}

\subsection{Corpus Overview}
The input datasets (i.e., wordlists) were extracted from the Berman Longitudinal Corpus (BLC),
\citep{berman-weissenborn:1991},
which is part of the Hebrew section of the CHILDES Corpus \citep{macwhinney:2000a}. 
The CHILDES corpus consists of transcribed 
conversations between young children and adults. Those whose utterances appear in the corpus are
called \emph{participants}. Each file is the transcription of a particular session. 
The participants, who are identified in each file's
header, can include, for example, the parents,
 other relatives, such as a grandmother, 
the researcher (or \emph{investigator}) conducting the session, and most importantly the child him or herself, who is called
the \emph{target child}. All participants were native speakers of Modern Hebrew. %talks with the child in question 
%(i.e., the \emph{target child}) and thus  becomes part of the recorded interaction. 
The BLC comprises the transcriptions of many individual recording sessions. Each child's sessions 
were conducted over a period of 12 to 19 months.
%there is a set of transcribed sessions for each target child. Each child's set of sessions spanned a period of 12 to 19 months.
%that is, there is a set of transcribed sessions associated with sessions for each  There is thus a set of transcribed sessession  for each child.  each , together spanning
%a period of 12 to 19 months, depending on the child.
%Each target child underwent episodic recording sessions for a period lasting between
%12 and 19 months.
% (i.e., 1 year to 1 year, 7 months). 
Each child was between 16 and 21 months old when his/her 
recording sessions began, and between 28 to 39 months old ($2\frac{1}{3}$ years to $3\frac{1}{4}$) when they ended.
%\citep{albert-et-al:2013}, \citep{albert-et-al:2012}, \citep{macwhinney:2000a}
%and thus contains the 
%speech of adults as well as children. 
On the corpus's website, Berman characterizes the subjects and their backgrounds as follows: ``All four children are native speakers of Hebrew raised in monolingual, highly educated Hebrew-speaking homes, with both parents professionals, in urban communities of central Israel" \citep{berman-long-web}.

%On the corpus's website, Berman characterizes the subjects and their backgrounds as follows:``All four children are native speakers of Hebrew raised in monolingual, highly educated Hebrew-speaking homes, with both parents professionals, in urban communities of central Israel" \citep{berman-long-web}.
%``Finally, both the transcribers and the researchers involved in the project knew the children and their parents, and were familiar with the children's linguistic development beyond the data provided by the recorded sessions." (Berman, Intro to Berman Longitudinal Corpus)
%`` The interactions are natural since they were recorded in the homes, a setting familiar to the children, in the presence of a primary caregiver and / or other members of the family" (Ibid.)

%\subsection{Properties of the Berman Longitudinal Corpus}
%\subsubsection{Structure}

Figure~\ref{fig:excerpt} shows an excerpt from the BLC. This excerpt contains
two utterances, one spoken by the target child (CHI), and the other
by the child's mother (MOT).
\begin{figure}[ht]
\vspace{10pt}
\label{fig:excerpt}
\caption{Excerpt from the Berman Longitudinal Corpus}
\vspace{-12pt}\begin{tabbing}
\small
\hspace{0.6in} \= \hspace{5.5in} \kill
\textsf{*MOT:} \> \textsf{ma\, \textipa{P}\a'{i}ma\textipa{P} \,
\textipa{P}o\textipa{\.*s}\a'{a}\, ba\#\, mi\textipa{\.*t}b\a'{a}x ?} \\
\textsf{\%mor:} \> \textsf{que|ma=what n|\textipa{P}\a'{i}ma\textipa{P}\&gen:fm\&num:sg\&stat:free=mother} \\
 \> \textsf{part|\textipa{P}a\textipa{\.*s}\a'{a}\&root:\textipa{P}\textipa{\.*s}y\&ptn:qal\&gen:fm\&num:sg-\a'{a}=do} \\
   \> \textsf{prep|be~det|ha n|mi\textipa{\.*t}b\a'{a}x\&gen:ms\&num:sg\&stat:unsp=kitchen ?}\\
\textsf{\%gra:} \>	\textsf{1|3|ANONAGR 2|3|AAGR 3|0|ROOT 4|3|MPRE 5|6|MDET 6|4|APREP 7|3|PUNCT}\\
\textsf{*CHI:} \> \textsf{rox\a'{e}cet kel\a'{i}m .}\\
\textsf{\%mor:} \> \textsf{part|rax\a'{a}c\&root:rxc\&ptn:qal\&gen:fm\&num:sg-et=wash} \\
    \>  \textsf{n|kli}\&\textsf{gen:ms\&num:pl\&pl:masc:match\&stat:free-\a'{i}m=tool .} \\ 
\textsf{\%gra:} \> \textsf{1|0|ROOT 2|1|ANONAGR 3|1|PUNCT}
\end{tabbing}
\end{figure}
The first utterance is the mother's,
who says, ``What is Mommy doing in the kitchen?". The child
replies,``\textsf{rox\a'{e}cet kel\a'{i}m .}'' `Washing dishes.'
%by the target child (\textsf{CHI}), and the second by the child's 
%mother (\textsf{MOT}). 
%The first utterance is the mother's.
%She says, ``What is Mommy doing in the kitchen?"  
%The child answers, ``Washing dishes."
The asterisk (*) 
preceding the labels \textsf{CHI} and \textsf{MOT} indicates 
that these tiers are \emph{main} tiers, i.e.,  
utterance tiers. Each utterance is accompanied by two additional tiers, 
namely, a morphological tier, labeled \textsf{\%mor}, and a syntactic 
(or grammatical) tier, labeled \textsf{\%gra}. 
The syntactic tier is not relevant to the present thesis,
so we shall not refer to it further.
The morphological 
tier has a hierarchical structure. At the highest level, it consists of a 
series of space-delimited morphological analyses, 
There is exactly one analysis for each word in the main tier.  
%Just as the words in main tier are delimited by white spaces, so are the morphological analyses in the morphological tier. 
Each \emph{individual} morphological analysis consists
of \textit{morphosyntactic properties}. These are delimited by the ampersand symbol (\textsf{\&}). 
Morphosyntactic properties are 
generally expressed as feature-value pairs of the 
form \textsf{\textit{feature}:\textit{value}}, e.g., \textsf{gen:fm} `gender: feminine.'

% such that a col typically 
%a given morphological analysis are morphosyntactic properties, usually composed
%of a \textit{feature} and \textit{value}, with a colon separating the former
%from the latter, e.g., \textsf{gen:fm} `gender: feminine.'
%The morphosyntactic properties are delimited by the ampersand symbol (\textsf{\&}).
%categories, many of expressed as feature-value pairs, e.g., \textsf{gen:fm}.  

\subsection{Transcription System}\label{sec:transcription}
The BLC's transcription system eludes simple one-word characterizations 
such as ``orthographic,"
``phonetic," or ``phonemic." It is in fact a hybrid system, drawing
both from phonetics and orthography \citep{albert-et-al:2013}. It is a transcription system
in some ways and a transliteration system in others.
It is a transcription system in that it has five dedicated vowel symbols, 
namely \{a, e, i, o, u\}, each of which matches one of the vowels in Modern 
Hebrew's five-vowel inventory. %  that match the five vowels of Modern Hebrew's vowel inventory 
%along with five complementary stressed-vowel symbols, which accurately
%capture the  and part a transliteration system---the former because, for instance, in that its five vowels 
%together with their stressed
%counterparts, accurately represents the vowel inventory of spoken Modern Hebrew. 
At the same time, however, 
It is  of a \emph{transliteration} system in that it is sensitive to orthography. This sensitivity is evident in 
table~\ref{table:alphabet}; note the nearly perfect one-to-one correspondence between the BLC's consonant 
symbols and those of the actual Hebrew alphabet. The consonantal component of the BLC's transcriptional system 
honors a number of historical distinctions that have preserved in the orthography, i.e., the alphabet and word 
spellings, but neutralized or otherwise lost in spoken Modern Hebrew. Table~\ref{tab:phon-neut} compares 
(reconstructed) CH phonemes, which are considered to be in one-to-one correspondence with the letters of 
the Hebrew alphabet, MH speech sounds, and transcription symbols from the BLC. Notice that there are fewest distinctions
in the MH column.
%with an eye to retain a number  orthographic consonants in mind. 
%During the revival of Hebrew care taken to remain as faithful as possible to earlier forms of the language, especially Biblical Hebrew (a sub-case of Classi


%The alphabet of modern Hebrew \emph{is} 
%The alphabet of Classical Hebrew has remained intact for millennia, 
%The Hebrew alphabet has remained unchanged for millennia, though, of course, 
%lettering styles have come and gone. It remains a strictly consonantal alphabet. 
%However, certain consonant letters, namely 
%\begin{cjhebrew}'\end{cjhebrew} (\textit{'alef}), 
%\begin{cjhebrew}w\end{cjhebrew} (\textit{waw}), and 
%\begin{cjhebrew}y\end{cjhebrew} (\textit{yod}), sometimes function as vowel 
%letters, but they always retain their fundamental consonantal affiliation


%As far as graphemes are concerned


%The Modern Hebrew alphabet has the same graphemes as the alphabet of Classical Hebrew. However, while the graphemes themselves have not changed for millennia, the mappings from graphemes to sounds have changed substantially.  In the time of Classical Hebrew, the alphabet was an accurate representation of the consonant phonemic inventory, whereas today two consonants may be mapped to the same sound due to a neutralization process. 

%definitely shows were clearly devised with
%the orthographic Hebrew alphabet in mind. 
%
%
%th\dots, as well as part transliteration 
%system in that is it encodes of orthography of Hebrew (i.e., the Hebrew alphabet 
%and Hebrew spelling) and the phonemic inventory of classical Hebrew, 
%which is not the same as that of MH \dots. It \dots.
%Each utterance is transcribed according to a system
%specially designed for the purpose of representing Modern Hebrew speech. 
%\citep{albert-et-al:2013}. It is based on previous transcription systems 
%devised by Berman (CITE). The alphabet is (thus) in fact neither truly 
%phonetic nor phonemic.  
%It cannot be described as purely phonemic because of its close ties to 
%orthography--its ``consciousness"
%of orthography, so to speak. Like Classical Hebrew, Modern Hebrew is 
%written (and printed) solely with consonant letters. Certain consonant letters, namely 
%\begin{cjhebrew}'\end{cjhebrew} (\textit{'alef}), \begin{cjhebrew}w\end{cjhebrew} (\textit{waw}), and 
%\begin{cjhebrew}y\end{cjhebrew} (\textit{yod}),
%sometimes function as vowel letters, but they always retain their fundamental consonantal affiliation.

%Hebrew was intentionally revived to be as faithful as possible to Classical Hebrew, 
%\marginpar{see discussion in \cite{albert-et-al:2013}}
%particularly in morphology, 
%but also in syntax and phonology. 
Modern Hebrew orthography thus preserves the shadows of lost Classical Hebrew phonemes. Similarly, many morphophonological processes in Modern Hebrew
preserve residues of Classical Hebrew phonotactics and phonological processes. Many such processes have persisted 
even though their motivation, the triggering contexts, 
have been obscured or altogether lost due to sound change. 
%Ancient phonotactics and 
%(morpho-)phonological alternations are often preserved in Modern Hebrew,  
%even though the motivation for these processes, the triggering contexts, 
%have been obscured or altogether lost due to sound change. 
For example, in Classical Hebrew, the consonants \begin{cjhebrew}.h\end{cjhebrew} 
(\textit{\textipa{\textcrh{et}}}) and \begin{cjhebrew}`\end{cjhebrew}
(\textit{`ayin}) 
were both pharyngeal consonants; the former was a voiceless fricative
(IPA \textipa{[\textcrh]}), and the latter a voiced stop (IPA \textipa{[Q]}). There was a process in CH whereby the 
[+low] vowel /a/ was inserted between a [-low] vowel and a pharyngeal consonant to
mediate the transition between the [-low] vowel and the [+low] pharyngeal.
This process is illustrated in table~\ref{tab:a-insertion}. 

MH, however, has no pharyngeals, having inherited its phonemic inventory 
largely from central and eastern European languages \citep{montoya:2014}. In MH, \textipa{[\textcrh]} 
is pronounced as a voiceless velar fricative (IPA \textipa{[x]}), and \textit{`ayin} 
is usually not pronounced at all. It has become phonologically 
equivalent to the historical glottal stop 
\begin{cjhebrew}'\end{cjhebrew} (\textit{'alef}). 
Both \begin{cjhebrew}'\end{cjhebrew} and \begin{cjhebrew}`\end{cjhebrew} 
are sometimes realized as glottal stops, but are often not pronounced at all, 
especially in fast speech (\citep{matras-and-schifff:2005,berman:1985}). Nevertheless,
MH as maintained pre-pharyngeal \textit{a}-insertion, as though the graphemes 
\begin{cjhebrew}`\end{cjhebrew} (\textit{`ayin}) and \textit{\textipa{\textcrh{et}}} 
still corresponded to phonological pharyngeals.

\begin{table}[h]
\centering
\caption{\emph{a}-insertion before historical pharyngeals}\vspace{3pt}
\label{tab:a-insertion}
\begin{tabular}{l c c c}
\hline\hline
CH    & MH   &  BLC  & Gloss     \\
\hline
\textipa{kot\'eB}  & \textipa{kot\'ev} & \textipa{kot\'ev} & `writes(s), writing' \\
\textipa{bor\'e\textbf{a}\textcrh} & \textipa{bor\'e\textbf{a}x} & \textipa{bor\'e\textbf{a}x} & `escape(s), escaping'  \\
\textipa{yod\'e\textbf{a}Q} & \textipa{yod\'e\textbf{a}} & \textipa{yod\'e\textbf{a}Q} & `know(s), knowing' \\\hline
\end{tabular}
\end{table}
	
The \begin{cjhebrew}`\end{cjhebrew} (\textit{`ayin}) is not the only 
grapheme to have 
lost its ancient phonemic affiliation. 
The graphemes \begin{cjhebrew}.t\end{cjhebrew} 
(\textit{\textipa{\.*te\.*t}}), \begin{cjhebrew}s\end{cjhebrew} 
(\textit{\textipa{same\.*k}}), corresponded to emphatic, i.e., pharyngealized, consonants in ancient Hebrew, i.e. pharyngealized versions of [s] of [t], most likely \ref{matras-and-schiff:2005}.
Modern Hebrew, however, lacks pharyngealized consonants just as it lacks pharyngeals. 
The consonant \begin{cjhebrew}q\end{cjhebrew} (\textit{qof}, IPA \textipa{[q]}) was uvular and 
possibly pharyngealized in Classical Hebrew, but in MH, it has merged with the velar voiceless stop
%\begin{cjhebrew}k|\end{cjhebrew} (\textit{kaf}, IPA \textipa{[k]}).
%Thus, 
%as the \emph{emphatic consonants}, were once associated with pharyngealized phonemes, but now they are pronounced simply as [s] and [t]. 
 Thus, the pronunciations (or affiliated phonemes) of \textit{\textipa{same\.*k}} and \textit{\textipa{\.*te\.*t}} have merged with those of \textit{sin} and \textit{tav} respectively, so that \textit{\textipa{same\.*k}} and \textit{\textipa{sin}} are both pronounced as [s], and \textit{\textipa{\.*te\.*t}} and \textit{tav} as [t].
%\begin{cjhebrew}q\end{cjhebrew} (\textit{qof}) have merged phonetically 
%with their non-pharangealized
%counterparts \textit{tav} and \textit{sin}, respectively. Thus, \dots.
%
%The consonant \begin{cjhebrew}q\end{cjhebrew} (\textit{qof}, IPA \textipa{[q]}) was uvular and 
%possibly pharyngealized in Classical Hebrew, but in MH, it has merged with the velar stop 
%\begin{cjhebrew}k|\end{cjhebrew} (\textit{kaf}, IPA \textipa{[k]}). Thus, in Modern Hebrew,
%both 
%\begin{cjhebrew}.h\end{cjhebrew}, and \begin{cjhebrew}.s| \end{cjhebrew}. Thus, \dots.

%The oral obstruents
%\begin{cjhebrew}b\end{cjhebrew} \textit{bet} ,
%\begin{cjhebrew}g\end{cjhebrew} \textit{gimel}, \begin{cjhebrew}d\end{cjhebrew} 
%\textit{dalet}, \begin{cjhebrew}k|\end{cjhebrew} \textit{kaf}, \begin{cjhebrew}p|\end{cjhebrew} 
%\textit{pe}, and \begin{cjhebrew}t\end{cjhebrew} \textit{tav} in Classical Hebrew 
%became fricatives in the context Vowel\_.

\begin{table}[ht]
\centering 
\begin{tabular}{l c c c c }
\hline\hline
\multirow{2}{*}{Heb. Letter} & Special &  \multicolumn{2}{c}{IPA} & \multirow{2}{*}{BLC}\\\cline{3-4}
    & Context & BH  & MH & \\
\hline
\begin{cjhebrew}'\end{cjhebrew} ('alef) & &  \textipa{P} & -- & \textit{\textsf{\textipa{P}}} \\
\begin{cjhebrew}`\end{cjhebrew} (`ayin) & &\textipa{Q} & -- & \textit{\textsf{\textipa{Q}}} \\
\begin{cjhebrew}q\end{cjhebrew} (qof )& & \textipa{q} & \textsf{\textipa{k}} & 
\textit{\textsf{\textipa{q}}} \\
\multirow{2}{*}{\begin{cjhebrew}k|\end{cjhebrew} (kaf)} & \multirow{2}{*}{Vowel\_} &
\textipa{k} & \textsf{\textipa{k}} & \textit{\textsf{\textipa{k}}} \\
		        & & \textipa{x} &\textsf{\textipa{x}} & \textit{\textsf{\textipa{\.*k}}} \\
 \begin{cjhebrew}x\end{cjhebrew} (\textipa{\textsubdot{h}et}) & &\textipa{\textcrh} & \textsf{\textipa{x}} & \textit{\textsf{\textipa{x}}} \\
 \begin{cjhebrew}s\end{cjhebrew} (\textipa{same\.*k}) & & ? &  \textsf{\textipa{s}} & \textit{\textsf{\textipa{s}}} \\
  \begin{cjhebrew},s\end{cjhebrew} (\textipa{sin}) & & \textipa{s}  &  \textsf{\textipa{s}} & \textit{\textsf{\textipa{\.*s}}} \\
 \begin{cjhebrew}.t\end{cjhebrew} (\textipa{\.*te\.*t}) & & ? &  \textsf{\textipa{t}} & \textit{\textsf{\textipa{\.*t}}} \\
  \multirow{2}{*}{\begin{cjhebrew}t\end{cjhebrew} (\textipa{tav})} & \multirow{2}{*}{Vowel\_} & \textipa{t} &  \textsf{\textipa{t}} & \textit{\textsf{\textipa{t}}} \\
&  & \textipa{T} &  \textsf{\textipa{t}} & \textit{\textsf{\textipa{t}}} \\
 \multirow{2}{*}{\begin{cjhebrew}b\end{cjhebrew} (bet)} & \multirow{2}{*}{Vowel\_} & \textipa{b} & \textsf{\textipa{b}} & \textit{\textsf{\textipa{b}}} \\
 &      	  & \textipa{B} & \textsf{\textipa{v}} & \textit{\textsf{\textipa{v}}} \\
 \begin{cjhebrew}w\end{cjhebrew} (\textipa{waw}) & & \textipa{w} & \textsf{\textipa{v}} & \textit{\textsf{\textipa{v}}} \\
\hline
\end{tabular}
\label{tab:phon-neut} 
\caption{Relationships between Orthography and Speech Sounds}
\end{table}

%table of neutralizations and losses in MH phonemic inventory.
%\begin{table}[ht]
%\centering
%\caption{a-insertion before historic pharyngeals}
%\label{tab:a-insertion}
%\begin{tabular}{l c c c}
%Actual MH & BLC & Ancient Heb. & Gloss \\
%\textipa{yod\'ea}  & \textipa{yod\'eaQ} & \textipa{yod\'eaQ} & `know(s), one who knows' \\
%\textipa{bore\'ax} & \textipa{bor\'eax} & \textipa{bor\'ea\textcrh} & `flee(s), escape(s), one who flees'  \\
%\end{tabular}
%\end{table}

\begin{table}[ht]
\centering
\caption{Transcription System of the Berman Longitudinal Corpus}% title of Table
\label{table:alphabet}
\subtable[Consonants\label{subtab:trans-cons}]{
\begin{tabular}{c c c c c c c c c c c c c}
\hline %\hline                      
\begin{cjhebrew}'\end{cjhebrew} & \begin{cjhebrew}b\end{cjhebrew} & \begin{cjhebrew}g\end{cjhebrew} & \begin{cjhebrew}d\end{cjhebrew} 
& \begin{cjhebrew}h\end{cjhebrew} & \begin{cjhebrew}w\end{cjhebrew} & \begin{cjhebrew}z\end{cjhebrew}& \begin{cjhebrew}.h\end{cjhebrew}
& \begin{cjhebrew}.t\end{cjhebrew} & \begin{cjhebrew}y\end{cjhebrew} & \begin{cjhebrew}k|\end{cjhebrew} & \begin{cjhebrew}l\end{cjhebrew} 
&\begin{cjhebrew}m|\end{cjhebrew}\\ 
\textipa{P} & b/v & g & d 
& h & w & z & x 
& \textsubdot{t} & y & k/\textsubdot{k} & l & m \\[12pt]
           
           \begin{cjhebrew}n|\end{cjhebrew} & \begin{cjhebrew}s\end{cjhebrew} & \begin{cjhebrew}`\end{cjhebrew} 
           & \begin{cjhebrew}p|\end{cjhebrew} & \begin{cjhebrew}.s\end{cjhebrew} & \begin{cjhebrew}q\end{cjhebrew} & \begin{cjhebrew}r\end{cjhebrew} 
           & \begin{cjhebrew},s\end{cjhebrew}/\begin{cjhebrew}+s\end{cjhebrew}
           & \begin{cjhebrew}t\end{cjhebrew} & & \begin{cjhebrew}z\end{cjhebrew}$^\prime$ & \begin{cjhebrew}g\end{cjhebrew}$^\prime$ & \begin{cjhebrew}.s|\end{cjhebrew}$^\prime$ \\
	  
	  n & s & \textipa{Q} 
	  & p/f & c & q & r & \v{s}/\textsubdot{s} 
	  & t & &  \v{z} & \textipa{J} & \c{c} \\
\hline
\end{tabular}
}
\subtable[Vowels\label{subtab:trans-vowels}]{
\begin{tabular}{c c c c c}
\hline
a & e & i & o & u \\
\'a & \'e & \'i & \'o & \'u \\
\hline
\end{tabular}
}
\end{table}

\subsection{Morphological Annotation}
The following are key aspects of the BLC's annotation system for morphology \citep{albert-et-al:2012}. 

\paragraph{Prefixal clitics.} In Hebrew, many functional words are \textit{prefixal clitics}; that is, they attach to content 
words as prefixes. They include the definite article \textit{ha-}, the prepositions \textit{le-} `to,' \textit{be-} `be,' 
\textit{ke-} `like/as,' and \textit{me-} `from,' the  complementizer/relativizer \textit{\v{s}e-} `that/which,' and the 
conjunction \textit{we-} `and.' For example,  in
\textsf{wehay\'eled} (`and the boy'), the prefixes \textit{we} and \textit{ha} are attached as clitics to \textit{y\'eled}, 
so that the whole is regarded as a single word. The BLC tokenizes these clitics, separating them from the main (content) 
word, and adding the number symbol (`\texttt{\#}') to the end of each prefixal clitic: \textsf{wehay\'eled} 
would be transcribed as \textsf{we\# ha\# y\'eled}.

\begin{exe}\label{ex:preclitics}
	\ex
	\textsf{we\#\, \v{s}e\# me\textipa{Q}arbev\'im} \textsf{me\textipa{Q}arbev\'im}\\
	\textsf{conj|we conj:subor|\v{s}e\, part|\textipa{P}irb\'ev\&root:\textipa{Q}rbb\&ptn:piel\&gen:ms\&num:pl} 
\end{exe}

%\subparagraph{Phonologically merged clitics.} This is actually good for my buddy Multimorph. 
%We don't want to unmerge these
%puppies. The two are notationally linked via the tilde symbol ($\sim$).

\paragraph{Demarcation of inflectional affixes.}
Past tense verbs take inflectional suffixes. Future-tense verbs take both inflectional prefixes and suffixes. 
Nominals, where inflection
is concerned, take only suffixes. The `\#' and `-' symbols are used to separate inflectional
affixes from the stem's morphological analysis; the `\#' used in the case of prefixes, `-' in the
case of suffixes, as in the following:

\begin{exe}
\ex \begin{tabbing} \label{ex:pre:v:suf}
\hspace{0.6in} \= \hspace{5.5in} \kill
\textsf{*CHI:} \> \textsf{tirt\a'{u} .} \\
\textsf{\%mor:} \> \textbf{\textsf{ti}\#}\textsf{v|ra\textipa{P}\a'{a}\&root:r\textipa{P}y\&ptn:qal\&tense:fut\&pers:2\&
\textbf{gen:unsp\&num:pl-\a'{u}}=see}
\end{tabbing}
\end{exe}

\paragraph{Compound and bound nouns}
Construct-state nouns are labeled either as ``\textsf{stat:bound}'' 
or ``\textsf{stat:comp}'' (i.e.,`compound'). 
These two labels correspond to the same sort of phonological form, 
 so I collapsed them into a single label, namely 
``\textsf{stat:cstr}'' (for `construct state').


\subsection{Anomalous Forms}\label{sec:anomolous}

Even though the BLC contains a good deal of adult speech, it also, 
of course, contains a substantial amount of child speech. After all, the study of
child speech is the corpus's primary reason for existing. But child speech is often rife
with anomalies, i.e., anomalies relative to adult speech. And while such anomalies are 
interesting if one is studying language acquisition in children, they are not helpful where
the present study is concerned. It was there for necessary to ``correct'' such anomalous 
forms where it was possible to do so, and discard them where it was not. Toward this end, 
the BLC's annotation conventions proved to be a great help.

\paragraph{Forms to be replaced}
%Thus, forms that are in some way anomalous (i.e., non-standard or non-adult)
%are not uncommon in the BLC. 
The BLC often deals with anomalous forms through a special 
annotative device consisting of a set square brackets, with the left bracket 
accompanied
by a colon, i.e., \textsf{[: \textit{corrected-form} ]}. Sometimes these brackets 
are followed by another set
of brackets. This latter set, if present, most frequently contains an asterisk 
(`*$\tau$') where $tau$ may be empty, or it may specify some error category. 
In the BLC, it is most frequently empty, as in (\ref{ex:repl2}). 
According to the CHILDES manual \citep{macwhinney:2000b}, the purpose of the ``\textsf{[* $\tau$])}'' 
is to specify that an anomalous form is 
indeed erroneous rather than merely non-standard. 
However, this distinction between 
\textsf{[: \textit{corrected-form} ]} and 
\textsf{[: \textit{corrected-form} ] [*{$\tau$}])} is not always manifest in BLC.

For example, in (\ref{ex:repl1}), the anomalous 
form \textsf{patux} 
(the `\textsf{@c} ' indicates a child-created form); 
the anomaly is an absent \textit{a} before the \textit{x}, thus violating 
a morphophonemic process 
whereby \textit{a} is inserted before pharyngeals, a process that was inherited 
from CH, even though MH no longer has pharyngeals. See (REF).
In (\ref{ex:repl2}), the error is essentially that the pre-\textit{\textipa{P}} 
\textit{a} is absent. The glottal stop
associated with the grapheme \begin{cjhebrew}'\end{cjhebrew} (\textit{'alef}) 
is typically not articulated in MH \citep{montoya:2014}. The distinction
between the two is not entirely clear.

In any case, however, the distinction, if there is one, is not important for our  purposes. 
Either way, we want to replace the anomalous form with the form in enclosed in 
(the first pair of) brackets (see REF). We can just discard the \textsf{[*]} if it is present.

%\begin{figure}[ht]
%\vspace{10pt}
%\label{fig:childerror}
%\caption{CHILDES}
%\vspace{-12pt}
%\paragraph{Replacements}
\begin{exe} 
\ex \begin{xlist} 
   \ex\label{ex:repl1} \begin{tabbing}  
	\hspace{0.6in} \= \hspace{5.5in} \kill
	\textsf{*CHI:} \> \textsf{po ye\v{s} patux@c \textbf{[: pat\a'{u}ax]} d\a'{e}let .}
	\end{tabbing}
     \ex\label{ex:repl2} \begin{tabbing}
	\hspace{0.6in} \= \hspace{5.5in} \kill
	\textsf{*CHI:} \> \textsf{tikri \,\textbf{[: tiqre\textipa{P}\a'{i}]}\, \textbf{[*]}\, 
	sip\a'{u}r\, d\a'{o}da\, Orly .} \\
	\textsf{\%mor:} \> \textsf{ti\#v|qar\a'{a}\textipa{P}\&root:qr\textipa{P}\&ptn:qal\&tense:fut\&pers:		2\&gen:fm\&num:sg-\a'{i}=read} \\
                    \> \textsf{n|sip\a'{u}r\&gen:ms\&num:sg\&stat:unsp } \\
                    \> \textsf{n|dod\&gen:fm\&num:sg\&stat:free-a=uncle/aunt} \textsf{n:prop|Orly .}
	\end{tabbing}
   \end{xlist}
\end{exe}
%\end{figure}

\paragraph{Dropped sounds}
Adults produce anomalous or non-standard speech, too. In (\ref{ex:dropped}), for instance, the investigator 
has dropped the 
\emph{n} from the infinitive form 
\textsf{li\textbf{(n)}s\'oa\textipa{Q}} (`to ride, drive, travel'), 
which is of the root n.s.\textipa{Q}. 
The BLC's transcribers ``restored'' such dropped or deleted sounds and enclosed them in parentheses.
\begin{exe} \label{ex:dropped}
\ex \begin{tabbing}
\hspace{0.6in} \= \hspace{5.5in} \kill
\textsf{*INV:} \> \textsf{li(n)s\a'{o}a\textipa{Q} \, ba\#\, \v{s}en\a'{i} ?}
%\%mor:	li#v|nas\a'{a}\textipa{Q}&root:nsʕ&ptn:qal&form:inf=drive/ride\, prep|be~det|ha \,adj|\v{s}enī&root:tbd&ptn:tbd&gen:ms&num:sg=second ?
\end{tabbing}
\end{exe}

\paragraph{Redundancies}
Sometimes a word, or, as in (\ref{ex:redundant}), a prefixal clitic, is duplicated, and sometimes repeated
several times consecutively. As in (\ref{ex:redundant}), in which the prefixal clitic \textit{we\#} is 
repeated several times, no special annotation is to mark or correct the repetition. Notice that morphological analysis for
for \textit{we\#} is also repeated, so that each \textit{we\#} in the main tier has its ``own'' copy of the morphological analysis.
\begin{exe} \label{ex:redundant}
\ex \textsf{*CHI:\quad we\# we\# we\# we\#\, nigm\textipa{\'a}r } \\
   \textsf{\%mor:\quad conj|we=and\, conj|we=and\, conj|we=and\, conj|we=and} \\
   \textsf{v|nigm\'ar\&root:gmr\&ptn:nifal\&tense:past\&pers:3\&gen:ms\&num:sg=be\_finished }
\end{exe}

%Observations: Note the number (or pound) sign \textsf{\#} at the end of the token \textsf{ba\#}. `in the', which is analyzed in the MA tier as $\texttt{prep|be}$$\sim$$\texttt{det|ha}$. 
%We handle this in different ways in the transcriptions vs. the morphoplogical analyses. Within the 
%morphological analysis tier, the analyses are delimited by spaces, with each single analysis corresponding
%to a word in the transcriptional tier. 
 
\section{Extracting the Input Datasets}\label{sec:extr}
%\subsection{Transformations}\label{sec:extr:transform}
\paragraph{Replacement of Anomalous Forms.} 
The input datasets were extracted from the BLC. However, as noted above in 
section~\ref{sec:anomolous}, the BLC is by no means free of anomalous forms, 
forms that are ungrammatical with respect to the standard form of the language, 
forms that would convey misinformation about the language if taken at face value. 
Such misrepresentative and potentially misleading forms were excluded from the 
extracted input data. 
%xpected form in some way---would clearly would not make good data for 
%Anomalous/nonstandard forms were excluded from the experiments' input datasets. 
Wherever a pair of brackets, i.e., \textsf{[: \textit{corrected-text} ]}, was encountered, 
the material within the brackets was extracted in place of the anomalous word (i.e., the 
word to the immediate left of the left bracket (\textsf{[:})). The 

%That is, 
%corrected forms were always extracted in place of
%their corresponding anomalous form. In effect, the replacement
%indicated by the annotation was simply carried.
%When a non-standard form is followed by a replacement 
%form, the ratio of tokens to morphological analyses is 
%pushed above 1. 
%Of course, two of the tokens
%--the non-standard form and its replacement--refer 
%to the same entity.  
%Thus, by replacing the non-standard form with 
%the bracketed form, we ``restore," in effect, the 
%one-to-one mapping. 
%More importantly, however,
%we avoid harvesting an anomalous form.

\begin{exe}\label{ex:replace}
	\ex Replacing anomalous forms with their bracket-enclosed corrections
	\begin{xlist}
	   \ex \textsf{*CHI:}\quad\textsf{po\, ye\v{s}\, patux@c\, [: pat\'{u}ax ]\, d\'{e}let} $\quad\to\quad$
	   \textbf{\textsf{po ye\v{s} pat\'{u}ax d\'{e}let}}
%	   \textsf{\%mor:}\quad\textsf{adv|po=here exs|ye\v{s}=there\_is} \, \textsf{adj|pat\'uax\&root:ptx\&ptn:qatul\&gen:ms\& \\ num:sg\&src:deverb=open\,  
%	n|d\'elet\&root:dlt\&ptn:qetel\&gen:fm\&num:sg\&stat:unsp} 
	   \ex \textsf{*CHI:}\quad\textsf{\textglotstop\'{o}\textsubdot{t}o\, micpacef@c\, [: mecafc\'ef ]\, [*]} $\quad\to\quad$ \textbf{\textsf{\textglotstop\'o\textsubdot{t}o\, mecafc\'ef}}
%\textsf{\%mor:}\quad\textsf{n|\textipa{P\'o\.*t}\&gen:ms\&num:sg\&stat:unsp=car\, cifc\'ef\&root:cpcp\&ptn:piel\&gen:ms\&num:sg=twitter/beep/ignore} 
	\end{xlist}
\end{exe} 

%\textsf{*CHI:}\quad \textsf{\textipa{P\'o\.*t}o\, micpacef@c\, [: mecafc\'ef]\, [*] .} $\to$ \textsf{*CHI:}\quad \textsf{\textipa{P\'o\.*t}o\, mecafc\'ef .}
%\textsf{\%mor:}\quad \textsfn|\textipa{P\'o\.*t}\&gen:ms\&num:sg\&stat:unsp=car\, part|cifc\'ef\&root:cpcp\&ptn:piel\&gen:ms\&num:sg=twitter/beep/ignore} 

\paragraph{Prefixal clitics.}
Whenever a word in the main tier is preceded by one or more prefixal clitic, we concatenate them one to another and attach the whole sequence to the base word. We discard all `\texttt{\#}' signs.
\begin{exe}\label{ex:preclitics}
	\ex
	\textsf{we\#\, \v{s}e\# me\textipa{Q}arbev\'im} $\quad\to\quad$ \textbf{\textit{\textsf{we\v{s}e}}}\textsf{me\textipa{Q}arbev\'im}\\
	\textsf{conj|we conj:subor|\v{s}e\, part|\textipa{P}irb\'ev\&root:\textipa{Q}rbb\&ptn:piel\&gen:ms\&num:pl} $\quad\to\quad$  \\
	\textit{\textbf{\textsf{conj:we\&conj:subor:sh\&}}}\textsf{part|\textipa{P}irb\'ev\&root:\textipa{Q}rbb\&ptn:piel\&gen:ms\&num:pl}  \\
\end{exe}

\paragraph{Complex adverbs.} In the BLC, all adverbs are characterized as atomic, unanalyzable morphological units. That is,
the analysis any adverb, no matter how morphologically complex, is simply ``adv", as in 
examples (\ref{ex:adv:bediyuq}) and (\ref{ex:adv:dscr})  These show the BLC's treatment of the adverbs \textit{\textsf{bediy\'{u}q}} and \textit{\textsf{me\textglotstop{a}xoran\'{i}t}}, respectively. Both are 
morphologically complex.

\begin{exe}
\ex \label{ex:adv:bediyuq}
	\begin{tabbing}
	\hspace{0.6in} \= \hspace{5.5in} \kill
	\textsf{*INV:} \> \textsf{ken ze \textbf{bediy\a'{u}q} xat\a'{u}l .} \\
	\textsf{\%mor:} \> \textsf{co|ken=yes\, pro:dem|ze\&pers:3\&gen:ms\&num:sg=it/this} \\
				\> \textsf{\textbf{adv|bediy\a'{u}q=exactly/precisely}} \\
				\> \textsf{n|xat\a'{u}l\&gen:ms\&num:sg\&stat:unsp=cat .}
	\end{tabbing}
\ex \label{ex:adv:dscr}
	\begin{tabbing}
	\hspace{0.6in} \= \hspace{5.5in} \kill
	\textsf{*INV:} \> \textsf{\textglotstop{i}\_{\textglotstop}{e}f\v{s}\a'{a}r \, raq \, \textbf{me\textglotstop{a}xoran\a'{i}t}\, 
		nto@c\, h@c .} \\
	\textsf{\%mor:} \> \textsf{adv|{\textglotstop}i\_{\textglotstop}ef\v{s}\a'{a}r=impossible \, adv|raq=only}\\
	 \> \textsf{\textbf{adv|me{\textglotstop}axoran\a'{i}t=from\_behind}\, chi|nto\, chi|h .}
	\end{tabbing}
\end{exe}

The adverb \textit{bediy\'uq} `precisely, exactly'
is composed of the prefixal preposition \textit{be-} `in' and the 
noun \textit{diy\'uk} `exactitude, accuracy', which is itself composed of the 
root $sqrt$d.y.q and the nominal vowel pattern C\textbf{i}C\textbf{\'u}C 
a pattern that implies a relationship to a verb of the \textit{Piel} binyan.  
The adverb \textit{\textsf{me\textglotstop{a}xoran\a'{i}t}} in \ref{ex:adv:descr} 
is even more complex. It has two nominalizing
derivation affixes \textit{\textsf{-an}} and \textit{\textsf{-it}} in addition 
to the prefixal preposition \textit{\textsf{me/i-}}. Moreover, it contains the 
root $sqrt$\textsf{\textipa{P}.x.r}, which is generally associated with meanings 
related to `late' and `after'.

In its morphological analyses, the BLC focuses on inflectional morphology. 
Except for roots and vowel patterns, it largely ignores derivational morphology.
%except for roots and vowel patterns.
%To be fair, derivational morphology is not within the BLC's stated objectives (CITE). 
%One can thus
%hardly criticize its lack of derivational morphological analysis. Unfortunately, adverbs by nature
%have no inflectional morphology, but they often have derivational morphology.

%This adverb has
%segments that do not contribute to its meaning in any clear way. 
%Their semantic contributions are not entirely 
%clear. So how do we handle this?

%In its morphological analyses, the BLC focuses on inflectional morphology, ignoring derivational morphology
%except for roots and vowel patterns.

%In this way, the BLC's morphological analyses are much like 
%those produced by the finite state morphological analyzer MILA (CITE), which was the source of the
%gold-standard morphological analyses used in \cite{meyer:2006} to evaluate morphological
%clusters. Moreover, both MILA analyses and BLC analyses consist of highly abstract
%categories like \texttt{num:pl} and \texttt{gen:ms}, which do not map neatly to Hebrew's
%fusional and synchretic inflectional morphology. 

%That is, its morphological categories are each atomic and abstract they are highly conceptual in nature highly semantic and not grounded in form, at least not in every case, and perhaps not in most cases. 
%That is, if it is not morphomic not at all, it is the opposite, if you will. It lies on the other end of the spectrum. It does not acknowledge and intermediate level of autonomous morphology between phonology and syntax. This is clear in the example above the example of ...
%
%But the BLC's morphological analyses are basically sound from the viewpoint of classical morpheme-based morphology. This is how this is the way morphome[-based] morphological analyses are typically. As is mentioned elsewhere in this dissertation(?), there are no existing gold-standard segmentations of words into their component morphomes, and thus one must work with what one has, however imperfect (from a morphomic point of view).

The simplistic analyses of complex adverbs were expanded so as to capture at 
least some of the derivational morphology of these adverbs. %its morphological analysis. 
The (originally) simplistic morphological analyses of complex adverbs were expanded 
(by yours truly). 
%I did add information. 
But I did not go so far as to delineate morphomes, as this would 
be straying too far from the rest of the analyses in the BLC. Such 
analyses would be out of place among all of the other analyses in corpus. 
This would at best be a futile, and at worst, experimentally detrimental.

The expansion of adverb analyses thus consisted of the following limited set of actions. 
%are principles for expanding the analyses of complex adverbs. We took these steps, and only these steps.
%to expand the the morphological analysis of a complex adverb.
\begin{enumerate}
\item Identify the preposition(s) present in the adverb, if any.
\item Specify the root, if any is present.
\item Specify the pattern, if any is present.
\end{enumerate}

\paragraph{Compound nouns.}
Compound nouns in both CH and MH are made up of a \emph{chain} of 
one or more nouns in the \emph{construct} state followed by a final noun
in the \emph{absolute} state. The latter is marked by reduced stress 
and the latter by a normal stress assignment. The following examples 
%(\ref{ex:cstr:pasey}--\ref{ex:cstr:bdiqat})%in %\ref{ex:cstr:examples}
illustrate the way compound nouns are treated in the BLC.
\begin{exe}
\ex \label{ex:cstr:pasey}
	\textsf{pas\'{e}y+ha+rak\textipa{\'{e}}vet} \\
	\textsf{n:det|+n|pas\&gen:ms\&num:pl\&stat:comp-\'{e}y+det|ha+n|rak\textipa{\'{e}}vet}
\ex \label{ex:cstr:shaat} 
	\textsf{\v{s}\textglotstop{at}+sip\'{u}r .} \\
	\textsf{n|+n|\v{s}a\textipa{Q}\'a\&gen:fm\&num:sg\&stat:comp-\'at+n|sip\'ur .}
\ex \label{ex:cstr:bdiqat} 
	\textsf{bdiq\'{a}t\, \textipa{P}ozn\'{a}yim} \\ 
	\textsf{n|bdiq\'{a}\&root:tbd\&ptn:tbd\&gen:fm\&num:sg\&stat:bound-\'{a}t}
\end{exe}
In the main tier (i.e., the transcription tier), the components of a compound noun are
delimited by the `+' symbol. 
%In the morphological tier, however, things get a bit more
%complicated. Allow me to provide some background, please. I promise that it will be thinnest
%foundation ever laid. 

Now, a compound noun consists of at least two nouns. One of these 
(and only one) is the head of compound, that is to say, that it carries the morphosyntactic 
features of the whole compound.
In a Hebrew compound noun, the first noun
is always the morphosyntactic head, despite its phonologically reduced state.
%, which, perhaps counterintuitively, is always phonologically reduced. 
%But in any case, the first noun is indeed the head of any given compound noun.
Thus, in (\ref{ex:cstr:pasey}), \textsf{pas\'{e}y} 'stripes/bands' is the head of 
\textsf{pas\'{e}y+ha+rak\'{e}vet} `the railroad tracks.'\footnote{The \textit{ha} at the beginning of  
\textsf{\textit{ha}rak\'{e}vet} 
is the definite article, a prefixal clitic. The definite-article clitic attaches not 
to the compound's morphosyntactic head, but rather to its \emph{phonological} 
head, i.e., the final component noun, which bears the compound's primary stress.}

%There remains the question of what exactly we should extract where compound 
%nouns are concerned. Should, one, for instance, take the \emph{whole} the 
%compound, i.e., \emph{all} of is component nouns? 

%Now, in this study, whenever the data-extraction script encountered a compound 
%noun, it 
For our purposes, we extract only the first noun of a compound noun, discarding the others. %when confronted with a compound noun. 
% the morphosyntactic head, were
%extracted and included in the input datasets; the other component nouns were discarded. 
% and any other component noun was discarded. 
There are two reasons for doing so: %keeping only the first noun. 
The first is word length; 
i.e., compound nouns can, in principle, be indefinitely long and comprise
indefinitely many component nouns.\marginpar{What 
is the longest in the data?} 
Even a compound of two nouns begins to exceed the scope of a ULM study. morphological learning. 
It is challenge enough to identify a single content stem or a single content root.  in a word. 
The second reason the distribution 
of features among the component nouns: Because the initial noun is head, its 
features are \emph{the same as} the features of the whole compound. The means that we only 
have to extract the morphosyntactic features of the initial noun to extract the features of the whole 
compound. In fact, the other nouns in the compound will be morphosyntactically empty. They still have
morphology, of course, but this sort of morphology is not annotated in the BLC. The BLC provides only morphosyntactic features, which are the morphosyntactic features of the initial noun. There is nothing else to extract where compound nouns are
concerned. % (so that the features of the 
%whole compound are equal to the features of the initial noun, and vice versa).
%noun's morphological analysis will consist of only the first noun's features. 
%The other component nouns essentially have no morphosyntactic features. The are 
%morphosyntactically empty. Since we need to extract morphological analyses
%along with the words themselves, there is no point in extracting words  %---namely, 
%non-head nouns within compound nouns---that 
%that in effect have no morphological analysis.

%There are a couple of reasons for keeping
%only the first noun in a compound-noun construction, namely:
%\begin{enumerate}
%\item Hebrew compounds (or ``construct chains'') can be indefinitely long---
%long in terms of characters (or symbol) number as well as  number of stems.
%\item The existence of multitude
%\end{enumerate}
% 
%What are the relevant points? 1) Long-ass words. 2) multiple stems, i.e., multiple 
%``free" morphological units; that is, they would be free if only they were not in the 
%construct state, which is 
%a phonological property, not a lexical one (per se?). 
%Therefore, each component noun in a compound is \emph{lexically} independent(?)
%Hebrew compounds have also been called \emph{construct chains}, an apt term, 
%since they can be indefinitely.
%long in principle. [CITE example?]  The multiple stems and the potentially prodigious 
%lengths of compound nouns make them seem impractical as subjects of a ULM study.

%Alternative Course of Action:
%Just think for moment about the problem of really long words, compound words, 
%compounds nouns, no less. They seem to be too syntactic. 
%Instead, I extracted just the morphosyntactic head (i.e., the first component noun) 
%of each compound noun. As the morphosyntactic head, its features \emph{are} 
%the features of the whole compound noun, and compound noun are in fact features 
%of the first component noun. 

\begin{exe}
\ex \label{ex:cstr:pasey2}
	\textsf{pas\textipa{\'{e}}y+ha+rak\textipa{\'{e}}vet} $\quad\to\quad$ 
	\textbf{\textsf{pas\textipa{\'{e}}y}} \\
	\textsf{n:det|+n|pas\&gen:ms\&num:pl\&stat:comp-\'{e}y+det|ha+n|rak\textipa{\'{e}}vet} $\quad\to\quad$ \\
	\textbf{\textsf{pos:n\&gen:ms\&num:pl\&stat:cstr}}
\ex \label{ex:cstr:shaat2} 
	\textsf{\v{s}\textglotstop{at}+sip\'{u}r} $\quad\to\quad$ \textbf{\textsf{\v{s}\textglotstop{at}}}\\
	\textsf{n|+n|\v{s}a\textipa{Q}\'a\&gen:fm\&num:sg\&stat:comp-\'at+n|sip\'ur} $\quad\to\quad$ \\
	\textbf{\textsf{pos:n\&gen:fm\&num:sg\&stat:cstr}}
\ex \label{ex:cstr:bdiqat2} 
	\textsf{bdiq\'{a}t\, \textipa{P}ozn\'{a}yim} $\quad\to\quad$ \textbf{\textsf{\textsf{bdiq\'{a}t}}} \\ 
	\textsf{n|bdiq\'{a}\&root:tbd\&ptn:tbd\&gen:fm\&num:sg\&stat:bound-\'{a}t} $\quad\to\quad$ \\
	\textbf{\textsf{pos:n\&root:tbd\&ptn:tbd\&gen:fm\&num:sg\&stat:cstr}}
\end{exe}

\paragraph{Other multiword expressions} Sometimes one encounters two or more words joined 
by an underscore character. These are multiword expressions. We exclude them from
the input datasets. 
\begin{exe}
\ex \begin{tabbing}
\hspace{0.6in} \= \hspace{5.5in} \kill
\textsf{\*MOT:}\>\textsf{kol\_ha\_kav\a'{o}d .} \\
\textsf{\%mor:} \> \textsf{co|kol\_ha\_kav\a'{o}d=well\_done}
\end{tabbing}
\end{exe}


\section{Experimental Variables}\label{sec:expvars}
Hey hey hey. What do we want to learn? I want t Multimorph to ``train" on 
Research Questions
The experimental portion of this dissertation consisted of two major axes of inquiry. One concerned the \textsc{data representation}, and the other the \textsc{features}. We shall address each of these in turn. 


\subsection{Data Representation}\label{sev:expvars:datarep}

%Alright, now let us turn to question of what I needed to obtain/extract from the corpus
%in order to run the experiments.
%following is what I needed to get from the corpus:
The \textsc{data representation} variable refers to the way the words were encoded in the input
word lists. Essentially, it refers to a choice between one of two alphabets (and the spelling conventions that 
accompany each alphabet). The two alphabets are the standard consonantal Hebrew alphabet, consisting of 22 letters,
and the 34-symbol BLC transcriptional system. 




%Multimorph was run on each type of representation, and the results were compared. 

%What is Hebrew orthography like?  
%Look at that table, table~\ref{}
%How did you obtain the orthographic data from the 
%BLC's transcribed data? 
%The symbols in T's transcriptional system are mapped to those of     %(see table \ref{}) 
%Hebrew writing. As is evident in table~\ref{},

%The source of the transcriptional data was the BLC, whose orthography-conscious 
%transcriptional system we discussed above in section (REF). %Many of this system's 
%%consonant symbols
%Table~\ref{table:alphabet} above demonstrates the close correspondence between the BLC 
%consonant symbols and those of
% those of standard Hebrew orthography. The vowel-to-vowel correspondence, 
% on the other hand, is
% nonexistent. This is because there are no dedicated vowel symbols in standard 
% Hebrew orthography. 
 
 % What shall we call the BLC set? 
% This is because the BLC transcriptional system 
%is sensitive to orthographic distinctions. The consonant-to-consonant mappings were thus mostly 
%straightforward.

%to its performance on 
%orthographic data (\textbf{O}).
%The source of the transcriptional data was the BLC, whose orthography-conscious 
%transcriptional system we discussed above in section (REF). Many of this system's 
%consonant symbols 
%map directly to Hebrew letters rather than actual MH speech sounds (see REF).
%The transcriptional dataset actually comprises two closely datasets. These are 
%\textbf{T}, which in no way indicates stress assignment, and \textbf{TS}, in which 
%an accented vowel 
%(\'a, \'e, \'i, \'o, or \'u) appears in each stressed syllable instead of a ``regular''
%(accent-less) vowel.   Well, in the present study, these stress marks were terminated with
%extreme prejudice. 
%Stress is effin' worthless. 
%over the vowel of the stressed syllable.  represent accented w bearing vowels, namely \'a, \'e, \'i, \'o, or \'u.      The first, which we will label simply as T, does not indicate primary (or any other) stress; i.e., it contains no stress marks of any kind. The second, is the same  in which stress \emph{is} marked, so that the vowel nuclei of stressed syllables appear with an accent mark, i.e., as in \'a, \'e, \'i, \'o, or \'u. Let us adopt the shorthand T for the stressless transcriptional dataset and TS for the stress-marked transcriptional S stands for ``stress."
%one in which stress is not marked. We shall abbreviate the former TS
% which each word's primary stress is indicated via an accented vowel, i.e., \'a, \, 
%and the other with no stress markings. we shall label \textbf{TR}
The morphological analyses are, in their raw form, just lists of morphological categories, 
i.e., lists of simple feature-value pairs.
The following is a BLC morphological analysis after undergoing the transformations 
described in section~\ref{sec:extr}
\begin{exe} 
\ex \label{ex:finished}
\begin{tabbing}
\hspace{0.8in} \= \hspace{5.5in} \kill
\textsf{matxilim} \> \textsf{\textbf{matxil+im}\$\$part\&root:txl\&ptn:hifil\&gen:ms\&num:pl}\, \\
\> \textsf{\textbf{matxil+im}\$\$adj\&root:txl\&ptn:tbd\&gen:ms\&num:pl}
\end{tabbing}
\end{exe}
Note that there two separate analyses in (\ref{ex:finished}), a result of ambiguity. Note also that each analysis is preceded by a segmentation (in boldface). In cases of ambiguity, i.e., of multiple analyses, each analysis gets its own segmentation, even though
the segmentations may be identical.

%We also need a \emph{morphological segmentation} for each morphological analysis.
We need two sets of morphological segmentations, namely $T_{SEG}$ and $O_{SEG}$, where:
\begin{itemize}
\item $T_{SEG}$ = segmentations for 10 percent of the words in T.
\item  $O_{SEG}$ = segmentations for 10 percent of the words in O.
\end{itemize}
%as segmentations for 10 percent of the words in each dataset (i.e., both $T$  and $O$). % both T and O. % i.e., we need a segmentations for 1/10 Recall that this employs two methods of evaluation; one is intrinsic, and the other extrinsic. 
Each will act as a gold-standard in the extrinsic 
part of the evaluation (see chapter \ref{ch:eval}). 
%component, we need segmentations for 1/10 of the words of each wordlist. The  
%For the extrinsic evaluation component, we need to extract a morphological analysis (or multiple 
%morphological analyses if the word in question is ambiguous). We need such an analysis set for 
%each word in \emph{each} wordlist.  Each word can have multiple possible morphological 
%analyses, and a segmentation is more accurately regarded as a correlate---perhaps even a 
%result---of a \emph{particular} morphological analysis (i.e., one of potentially many in the 
%analysis set) than it is a innate property of the string (i.e., word) itself.

%A list of pairs ${(t_n, A_n)}_n$ % n \in N$ where $A$ is an inner list of pairs, namely $A {=}_{def} {({seg}_m, {a}_m)}_n$
%\begin{enumerate}
%    \item $T$: a list of transcribed words to serve as input to Multimorph.
%    \item $A$: list  of morphological analyses $\textbf{a}$ Each $t \in T$. Some
%\end{enumerate}

%Transformations of/on the transcriptions.
%What does one input to Multimorph? What is the origin of the process? What does Multimorph 
%do with the input? 
%Multimorph takes as input any list of words. In its current configuration, 
%Multimorph expects each word to be on a new line, but this is trivial. What is important is
%that Multimorph, like %many
%other ULM systems, presumes that each word is contextless, i.e., 
%independent of the other words in its list (REF and CITE).

%\subsection{Orthographic Words}
%I am using two primary wordlists: an orthographic and a transcribed wordlist. The 
%latter serves as the basis for two wordlists, and we thus have a total of three wordlists: 
%Orthographic Words (O), Transcribed Words with Stress Markings (TS), and Transcribed 
%Words \emph{without} Stress Markings (TR).
%
%\textbf{Orthographic Words (O):} I obtained the 6888 [or 12236 with vowels; this is the main 
%number, as the O list is derived from the TR and TS lists]. The following stuff is wrong: in 
%O from the dataset used by \cite{daya-et-al:2008} in their study of automatic root identification.
%These words are transliterations (Romanizations) of Standard Modern Hebrew spellings. Each 
%character in the Hebrew alphabet is unambiguously mapped to a distinct character
%in the Roman alphabet. Each Hebrew character is always mapped to the same Roman character. 
%
%\subsection{Transcribed Words}

%\textbf{Transcribed Words:} The source of the transcribed data is the Berman longitudinal 
%corpus \citep{berman-weissenborn:1991}, which is part of the Hebrew section of the 
%CHILDES databse, a multilingual corpus of verbal interactions between young children 
%and adults \citep{macwhinney:2000a}. Each utterance in the corpus is both transcribed and 
%morphologically analyzed. 
%In the transcriptions, stressed vowels are marked with acute accent marks.  
%   \begin{itemize}
%   	\item \textbf{Transcribed Words with Stress Markings (TS)}: The wordlist TS consists 
%	of 12,416 unique transcribed words extracted from the Berman longitudinal corpus.
%	%, I extracted the transcribed words present in the Berman longitudinal corpus. I removed duplicates so that each word in TS was unique. The pre-existing stress markings were left intact. TS contains %12,494 words. 
%12,416 words
%	\item \textbf{Transcribed Words without Stress Markings (TR)}: I obtained TR by 
%	stripping the stress markings from the words in TS. That is, TR is simply TS without 
%	stress markings. TR contains 12,352 unique words, Each word in TR can be mapped 
%	to at least one word in TR by giving it appropriate accent marks.
%	I will run my system on both TS and TR, but I do not expect the presence/absence of accent marks in the input data to make a significant difference where Multimorph's performance is concerned.
%\end{itemize}	

%What is a transcription? What are the ``quasi-transcriptions" in the BLC like?
%How is it and different and similar to the Orthographic Data?

%\subsection{Mixing Function}
%\subsection{Objective Function}
In \ref{ch:intro}, we discussed the problem of choosing a feature set. We described it as a problem of selecting one finite subset of 
features from an infinite number of possible features.
In the remainder of this chapter, we revisit this question. Our approach will be to think in terms of feature categories, i.e., to define categories of features that have favorable properties. We shall also consider the properties of our learning framework, discarding feature types (or categories) that are at odds with our learning framework, whether conceptually or mathematically. Ultimately, we cannot consider every possible feature or feature set. What we can do is produce some principled candidate feature sets with which to experiment. 

We shall glean considerable insight
from the field of computer vision. That is, we shall consider feature categories that 
are significant to computer vision and do so with an eye to adapting them for the purposes of ULM. The idea is that 
%the task of clustering objects, which one might identifying significant similarities amid a sea of differences.   makes two objects similar and what makes them different,
%and figuring out which objects to group together and which to separate, 
is actually a very general problem. And thus the
same types of features sometimes should apply to different types of objects
%These categories are \textit{variant} features and \textit{invariant} features. 
%>>>
We shall focus in particular on \emph{invariant} and \emph{variant} features. 
%As discussed in chapter\ref{ch:intro}, 
The distinction between these two feature types is highly significant in
computer vision and its various subfields. In the present dissertation, we hypothesize that this distinction is
also relevant to ULM. 


\subsection {Data Representation.} 
 %\emph{representation} of the input data,
By \emph{representation} we refer to mainly the alphabet used to spell out the strings (i.e., words) 
%---the
%particular set of symbols---that compose the strings (i.e., words) 
of a input wordlist.
In the present work, two data representations were tested, one transcriptional 
(\textbf{T}), the other orthographic (\textbf{O}). We shall identity the dataset with transcriptional representation as``T,' and
the one with the orthographic as ``O.''
T is essentially the BLC (but without stress markings). Its alphabet is thus BLC's 
transcriptional system, which we discussed above in section. % \ref{sec:transcription}.
T and O differ both qualitatively and quantitatively. 
Arguably, T contains more information than O,
since T has a distinct symbol for each of MH's five vowels sounds, %, %and each vowel sound has its own distinct symbol 
%that is devoted exclusively to that vowel.
as well as all of the archaic consonantal distinctions found in O.

\subsection{Experimental Program}
The experiments tested 

\subsection{Features}\label{sec:expvars:features}
\label{sec:features}

%is relevant to morphological learning. We noted that in Optical Character 
%Recognition (OCR), both 
%variant and invariant features are necessary to distinguish some character pairs, 
%e.g., 6 and 9. 

%We hypothesized that both variant and invariant features would be required for 
%morphological learning also, i.e. that a combination would be most advantageous, i.e., 
%yield the best results. 
%
%We thus asked the question of whar We thus devised two types of features. S
%
%In particular, bullshit oh man oh man 


%Now, let $F$ be the infinite set of all possible features. In keeping with the spirit of my primary research objective, we will assume that there exists at least one $\phi$ such that that $\phi \subset F$, $\phi$ is finite, and $\phi$ allows an ULM system to learn a multilinear model of morphology. The objective here is to confirm this assumption, i.e., to find such a subset.

%Feature-set design is a crucial component of any machine-learning project. It therefore
%constitutes one this dissertation's main research questions. One way to come with likely feature subsets is to
%find feature properties that are likely to be beneficial, namely, \textit{Which types of features are most
%beneficial to a non-linear non-sequential morphological learner such as the MCMM-based Multimorph?}

\subsection{Feature-set Desiderata: Insights from Computer Vision}
%Anyway, the point is that that \cite{dudani-et-al:1977} present a kind of theory about what constitutes good features. However, they do not really justify these desiderata; rather they treat them as obvious. Basically, they just make a sort of pronouncement?

%Q001: But is this pronouncement revisited later in the paper? If so, why? What do they say about it?

%Q002: What are the desiderata for my features, and what is the rationale for the desiderata? In testing the features, am I testing the desiderata? Am I testing the rationale behind the desiderata?

%Why precedence features? Why positional features? What would happen if 
%all features were positional features? What if all features 
%were precedence features? What other types of features might be possible? %[What are some other such questions?]
%Why precedence features? 

%[A year ago, I suspected that precedence features may not be sufficient per se. 
%But why did I suspect this?]

%%These involve rational argumentation rather than experimentation. I?m 
%%saying that these feature-related insights are important. But what are they?
%There is a great deal of information tacitly present in a string of alphabetic 
%symbols. This is true regardless of whether the string is written, in which 
%case the symbols are graphemes, or spoken, in which case they are phonemes.
%Many machine learning models require that each input object or learning 
%instance (in our case, each word) be represented as a vector of binary features, 
%not alphabetic features. Moreover, all input objects (learning instances) must 
%be described by the same feature set, and all feature vectors must be of 
%the same length.
%One cannot, of course, have an infinite number of features, since feature vectors 
%must be of finite length. 
%Now, let $F$ be the infinite set of all possible features. 
%In keeping with the spirit of my primary research objective, we will assume that 
%there exists at least one  such that that $F$, is finite, and  allows an ULM 
%system to learn a multilinear model of morphology. The objective here is to 
%confirm this assumption, i.e., to find such a subset.
%Goals for designing a feature set: Invariant and variant (adapted from computer vision) We want to glean from literature, 
%perhaps even literature from other fields, insights for assembling a list or set of desiderata for a 
%feature set. We also want to find conceptually coherent (i.e., natural) categories of features 
%used in other fields (or for other tasks). We will then be able to ask the question, Which 
%of these features are most important for our task? Which seem unnecessary (or even 
%detrimental)?

%``In order to recognize many variations of the same character, features that are invariant to certain transformations on the character need to be used" \citep{trier-et-al:1996}. 
%
%``But rotation-variant features will be necessary to differentiate certain pairs of characters like 6 and 9" \citep{trier-et-al:1996}.
%
%``Devijver and Kittler define feature extraction as the problem of `extracting from the raw data the information which is most relevant for classification purposes, in the sense of minimizing the within-class pattern variability while enhancing the between-class pattern variability'. It should be clear that different feature extraction methods fulfill this requirement to a varying degree, depending on the specific recognition problem and available data" \citep{trier-et-al:1996}.
\cite{dudani-et-al:1977} describe three desiderata for features in the 
domain of aircraft identification:
\begin{quote}
\begin{enumerate}
\item The features should be informative. That is, the dimensionality of a 
vector of measurements (feature vector) should be as low as possible, 
consistent with acceptable recognition accuracy.
\item The features should be invariant with translation of the object 
normal to the camera optical axis and with rotation about this axis.
\item The features should either be invariant or depend in a known 
way upon the distance of the object from the camera.
\citep[][p. 40]{dudani-et-al:1977}.
\end{enumerate}
\end{quote}
Though stated in terms of computer vision and image recognition, 
these desiderata (or criteria) are relevant to all varieties of machine 
learning and clustering, including, of course, what Multimorph does, which is to
cluster words according to shared components.  
(see below).

Invariant features are generally considered to be preferable to variant 
features \citep{hossain-et-al:2012}. 
There are different kinds of invariance for different kinds of problems.
Scale and rotation invariance are just two examples. If a feature 
is scale-invariant, its
value is independent of the size of the object in question. That is, a 
scale-invariant feature
can have same value for objects $A$ and $B$, regardless of their 
relative sizes. $A$ could be much larger than $B$, for instance, or much smaller,
but a scale-invariant feature would not be sensitive to the difference in size. 
This is important because $A$ and $B$ could in fact be the one and the same shape; it might appear 
larger in one photograph simply because 
it was closer to the camera when that photograph was taken.
%Suppose $A$ and $B$ are shapes in different photographs. 
%$A$ and $B$ could in fact be the same shape. 
%Thus, if an image recognition system is to be capable of recognizing 
%$A$ and $B$ as the same object, at least some of its features must 
%be scale-invariant.
%In general, in image recognition, invariant features are preferable 
%to variant ones \citep{hossain-et-al:2012}.
 

%Scale-invariant features are oblivious to changes in size. That is, all else being equal, 
%of the object in question. if the size the object in question is enlarged or shrunk. 


%They should be informative
%They should be ``Invariant with translation of the object normal to the camera optical axis to and with rotation about this axis"
%They should be either invariant with or dependent in a known way upon the distance of the object from the camera.

Sometimes, however, both invariant and variant features are necessary, 
as in optical character recognition (OCR)\citep{trier-et-al:1996}.
%For example, in the field of Optical Character Recognition (OCR) 
%provides some good examples of this scenario. 
Rotation-invariant features are considered essential in OCR \citep{trier-et-al:1996}, 
at least
insofar as a character's identity is independent 
of its orientation.
%i.e., to the extent that it will not turn into a different character
%if rotated a certain number of degrees.
%it does not turn into a different letter under rotation. 
And indeed, the identities of most Roman letters, both upper and lower case, are invulnerable to rotation.
%most characters of the Roman alphabet, both upper and lower case, are 
%invariant in this way, i.e.,
For example, if the letter \textsf{A} is rotated 180 degrees, it remains 
recognizable as an \textsf{A}, since its distinguishing features do not change when it is rotated. 
%(Of course without its cross bar, it would turn in into a \textsf{V}, 
%making the cross bar 
%an important visual feature of the \textsf{A}). 
%There are many other 
%rotation-invariant 
%characters; \textsf{i}, \textsf{B} and \textsf{Q} are 
%just a few examples.
 
However, this is not true of all letters. In many type faces, for instance, 
\textsf{q} can be rotated to become \textsf{b}, and \textsf{u} can be rotated to 
to become \textsf{n}. Often, a rotated (upside-down) \textsf{M} closely 
resembles a \textsf{W}. And should we consider numerals, we would have to deal 
with the famously rotation-variant \textsf{6} and \textsf{9}. If an 
OCR system were to rely solely upon rotation-invariant features, it
would not be able to distinguish between, for instance, \textsf{6} and 
\textsf{9}, since the only features that can distinguish \textsf{6} from \textsf{9}
are rotation variant.
% This is not always the case. However, if all the characters are expected to have the same rotation, then rotation-variant features should be used to distinguish between such characters as `6' and `9', and `n' and `u' \citep{trier-et-al:1996}.

%What about global and local features? How do global and local relate 
%to variant and invariant? Are global and local independent of invariant and 
%variant? If so we have a 2x2 confusion matrix.
Multimorph clusters words on the basis of shared components. 
Word length---even normalized word length---is not %really
germane to whether or not two words share a component. 


Global features describe an image as a whole, whereas local features are concerned
only with a particular region in the image. That is, the value of local feature is dependent
only upon its particular region; it is not affected by other regions. 
In OCR, an example of a global feature is \emph{aspect ratio}, defined 
as the width of a character's rectangular bounding box divided by the same 
box's height. Note that aspect ratio is size invariant. Global features 
can be made invariant 
through normalization, and ratios are means of normalization.

A global feature for ULM might be something like the number 
of characters in a given word, which could be binned. Another 
could be the consonant to vowel ratio in a word. The question, however, is 
whether such properties are relevant and beneficial to Multimorph's task,
which is to group together words that share components.

%Multimorph detects systematic similarity by grouping together words that share
%components (see chapter \ref{autonomous}). 
%Features like word length---even 
%normalized word length---are not helpful to this task.  

Global features are not conducive to bipartite graphical learning frameworks.
Recall from (REF) that an MCMM is a bipartite graph, and from (REF) that 
a bipartite graph satisfies the following:
%[Insert definition here:]
%A multipartite graph is a graph 
%whose nodes can be partitioned into $N$ \emph{disjoint} sets of 
%\emph{mutually nonadjacent} nodes, i.e., $N$ sets such that no two
%nodes within the \emph{same} set are connected by an edge. A bipartite graph
%is simply a multipartite graph such that $N = 2$. That is, 
%a graph is bipartite if it satisfies the following criteria:
\begin{enumerate}
\item its nodes are separated into two disjoint sets (or partitions).
\item within each partition, all nodes are independent, i.e., mutually 
nonadjacent.
\end{enumerate}
In the case of an MCMM, the two partitions are the vectors $\textbf{m}_{i}$
and $\textbf{r}_{i}$ (for each word $i$). The former is the vector of cluster activities; the latter
is the \emph{reconstruction} of the feature-vector representation of word $i$. (Recall from REF that $\textbf{r}_i$ is a working
reconstruction of $\textbf{x}_i$, the original and target feature vector for word $i$.)
By the definition of \emph{bipartite}, \emph{every} feature in $\textbf{r}_i$ is independent,
and thus every feature is local. 

Moreover, words tend to be complex objects, composed multiple morphological units.
These morphological units are often to some extent orthogonal. For example, a given 
plural suffix can appear with wide variety of stems. Any two words may share 
just one morphological component among many. The discrepancies could easily 
outnumber the one shared component, and yet 

and a single stem can appear with many different suffixes. 
rather, each is composed one or more---potentially many more---morphological units. 
Moreover, these morphological units are more or less orthogonal and thus freely recombinable. 
For instance, a given plural suffix can appear with wide variety of stems, 
and a single stem can appear with many different suffixes. The point 
is that global features, features 
that pertain to an object as a whole, are probably not going to benefit the %useful in %for the purpose of 
clustering of complex objects such as words. That is, any two words may share 
just one morphological component among multiple components in either word,
being otherwise entirely different. The differences  could easily 
outnumber the one shared component. 
%Recall that an MCMM is a bipartite graph because its layer of cluster-activity 
%nodes is separate from the ``surface'' feature nodes of the reconstruction later. 
%Thus, each cluster-activity node $m_{ik}$ can either connect to the
%feature node $r_{ij}$ via weight $c_{jk}$, or not; the question of whether 
%or not $m_{ik}$ connects $r_{ij}$ is determined separately (i.e., independently) 
%for each $(m_{ik}, r_{ij})$ pair.

%Each of the features in the vector $\textbf{r}$ is local: 
%the scope of each is rather narrow 
%both in a conceptual and ``spatial" sense: Each precedence feature
%represents a span of characters (or character indices) no longer than the 
%parameter $\delta$. The positional features each represent a single character 
%position. Both positional and precedence features ``move" left to right, so to 
%speak; that is, as
%the feature index increases, the character index in the original word also increases, though 
%at a slower rate, as
%the original word is much shorter.
%that is, a feature near near the beginning of the feature vector $\textbf{r} will 
%correspond to a position near the beginning the beginning of the original word. 
%Likewise, a feature in the middle of a word will correspond to more or less  \dots
%**cumulative = bad**
%**orthogonal, independent elements**

%the discordant component pairs, they would far
%outnumber %---and thus may be fellow members of the class that represents this component, 
%but differ
%Global features thus do not seem appropriate for the task of learning the morphology of 
%a natural 

\subsection{(In)variant Features for Morphological Learning}

\subsubsection{Positional Features}
Positional features
indicate the presence of a particular
character at a certain position relative to either the beginning or the end of 
a word. For example, \texttt{i@[0]} indicates that \textit{i} is the first 
character, while \texttt{i@[-1]} indicates that \textit{i} is the last character. 
Indices relative to the beginning are non-negative, while those relative 
to the end are negative. In any case, the number of positions considered 
at the beginning is always equal to the number at the end. This number, 
$s$, is an experimental variable.
%the variable $s$ is equal to the absolute value of the smallest negative index. 
For example, $s=3$ means that for each character $\alpha$, the following 
features are generated: $\alpha$\texttt{@[0]}, $\alpha$\texttt{@[1]}, 
$\alpha$\texttt{@[2]}, $\alpha$\texttt{@[-1]}, $\alpha$\texttt{@[-2]}, 
$\alpha$\texttt{@[-3]}.
%\footnote{Note, however, that the textual labels 
%of features are ultimately arbitrary; it is a feature's index that truly matters.} 
That is, we count three positions inward from each word boundary.

\subsubsection{Precedence Features}
Precedence features indicate, for any two characters $x$ and $y$,
whether $x$ precedes $y$ within $\delta$ characters,
where $\delta$ is an experimental variable defined as
	\begin{equation*}
	\delta = \text{index}(x) - \text{index}(y)
	\end{equation*}
Note that if $\delta = 1$, the precedence features are pairs of adjacent 
characters, i.e., bigrams.
Precedence features are affixation-invariant. Consider for example 
the word \textit{bediy\'uq}. If $\delta = 1$, \dots 

Precedence features are invariant with respect to affixation; that is to say, 
they are 
\emph{affixation-invariant}.
Precedence features are not bound to absolute character positions, i.e., 
absolute indices. They are not defined relative to fixed points such as 
the beginning or end of a word.
On the other hand, positional features are \emph{variant} 
with (respect to) affixation, which is to say that positional features are 
defined relative to the (absolute) beginning 
or end of a given word. 
%\section{Evaluation paradigms}
%\subsubsection{Intrinsic evaluation}
%\subsubsection{Extrinsic evaluation}
