\chapter{AUTONOMOUS MORPHOLOGY AS A TARGET OF LEARNING}

\section{Introduction} %Or (Toward) Learning Morphology By Itself
%%%%%%%%%%%%%%%%%%%%%%%%%%%%%%%%
%% The main point of this chapter is the following: "It's not implausible to 
%% take a non-morpheme-based approach, to assume that morphology
%% is independent of phonology and syntax." Put another way, the main goal
%% is to justify taking a non-morpheme-based approach.
%%
%% This justification will mainly involve pointing out that many scholars
%% argue for autonomous morphology in one way or another. I will basically
%% be citing precedents. These the authors of the "Word-and-Paradigm" (WP) camp.
%% Aronoff is in this group, but he is not the only one. Who else? Whom shall describe
%% in some depth, and whom shall I merely mention in passing? In passing: Anderson,
%% Mathews, [Network morphology], ...
%% In some depth: Stump, Aronoff
%% However, most WP proponents these days, assume that Aronoff's morphomes exist.
%% and use the term morphome in their own theories.
In this section, I shall justify my use of a list of individual context-less words as input data. 
Why might this be controversial? Because, without context, it is impossible to learn pairings of
meaning (i.e., morphosyntactic properties) and form in the absence of context. But is this really impossible?
What about Morfessor and Linguistica? These both take isolated words as input and seem
to be to learn actual morphemes. (Is this not the case?)

Some morphomes---those that are one-to-one mappings---are equivalent to conventional morphomes.
Those that map from one ``property set" to one form. 

The purpose of this subsection is to delineate what an MCMM can and cannot learn 
with regard to morphology, i.e., to differentiate feasible from infeasible learning targets.
An MCMM learns and outputs a clustering, but the range of possible 
clusters is restricted by certain constrains. Two major constraints 
are outlined in the following paragraphs:
%; the first stems from the very nature of MCMMs; the second stems from the nature of the learning task.

First, the range of possible clusters that can appear in an MCMM's output is constrained by the very nature of MCMMs. In particular, an MCMM has no way of representing hierarchical structure.
An MCMM's analysis (i.e., clustering) is flat; all clusters occupy the same level of structure. This is a significant limitation because morphological structure is often hierarchical. In Modern Hebrew, for instance, stems are derived first, at the lowest level. Next, any inflectional morphology is added, and finally, affixal particles are attached.
%All of the clusters ($=$ \emph{morphs})
%%in an MCMM's output occupy the same level of structure. 
%Or more accurately, there are no levels of structure in an MCMM's output; all clusters occupy the same level of structure. An MCMM's analysis is thus flat.

Second, an MCMM's output is constrained by the nature of its input. The input in the present study is a list of contextless words; I am using no syntactic or otherwise word-external features. The output clusters thus cannot correspond to morphosyntactic units, i.e., conventional morphemes. 
%Morphosyntactic categories are simply out of the question. Instead, % \dots \dots \dots
Because its input features will be strictly word-internal, my system will only be able to find \emph{intermediate} building
blocks, structural units that reside somewhere between phonology and
morphosyntactic meaning, i.e., on a level similar to Aronoff's \emph{morphomic} level
\citep{aronoff:1994}.

% part by certain constraints.  
%is constrained by certain natural limitations. That is,
%there are certain kinds of clusters that an MCMM, by its very nature, could never ever produce, no matter 
%These clusters are just out of the question.
%The output of an MCMM, recall, is a clustering. 
% possible and impossible for an MCMM   hypothesis regarding the MCMM's output,

%It would make no sense to evaluate against an impossible target.
%One should not expect an outcome that has an a priori probability of zero.
%But what exactly is impossible? What constrains an MCMM? 
%What are its limitations? 
%an MCMM has no way of representing hierarchical structure. All of the clusters ($=$ morphs)
%in an MCMM's output occupy the same level of structure. Or more accurately, there are no levels of structure in an MCMM's output; an MCMM's analysis is flat. 

%Second, the input dataset is a list of contextless words, and 
%I am using no syntactic or otherwise word-external features.
%in the present study. 
%My system thus clusters words solely on the bases of 
%word-internal similarities in form,
%i.e., shared subsequences of characters, 
%thereby finding form-based atomic building blocks.

%These building
%blocks, however, are not necessarily morphemes in the conventional
%sense. A morpheme is traditionally defined as the coupling of a form
%and a meaning, with the meaning often being a set of one or more
%morphosyntactic features.

%Therefore, it would be nonsensical---sheer foolishness---to evaluate MCMM-generated clusters
%against morphosyntactic gold-standard categories. We need gold-standard categories that
%My system, by contrast, discovers building
%blocks that reside on a level between phonological form and
%morphosyntactic meaning, i.e., on the \emph{morphomic} level
%\citep{aronoff:1994}. 

%\cite{stump:2001} captures this distinction in his classification of morphological theories,
%distinguishing \emph{incremental} and \emph{realizational} theories.
%Incremental theories view morphosyntactic properties as intrinsic to morphological markers.
%Accordingly, a word's morphosyntactic content grows monotonically
%with the number of markers it acquires. 
%By contrast, in realizational theories, certain sets of morphosyntactic properties \emph{license}
%certain morphological markers; thus, the morphosyntactic properties cannot be inherently present in the markers.
%\cite{stump:2001} presents considerable evidence for realizational morphology, e.g., the fact that ``a given property may be expressed by more than one morphological marking in the same word'' (p. 4). 

\cite{aronoff:1994} observes that the mapping 
between phonological and
morphosyntactic units is not always one-to-one. Often, one
morphosyntactic unit maps to more than one phonological form, or
vice versa. There are even many-to-many mappings. Aronoff cites
the English past participle: depending on the verb, the past
participle can by realized by the suffixes \textit{-ed} or
\textit{-en}, by ablaut, and so on. 

And yet for any given verb lexeme, the \emph{same} marker is used
for both the perfect tense and the passive voice, despite the lack of a
relationship between these disparate syntactic categories.
Aronoff argues that the complexity of these mappings between
(morpho-)syntax and phonology necessitates an intermediate level, namely the
morphomic level.

My system thus looks for \emph{morphs}, form-based structural units
inspired by Aronoff's morphomes.
Morphs have phonological form, but they have not yet been assigned syntactic or semantic meaning. 
In a larger pipeline, such 
building blocks could serve as an interface between morphosyntax 
and phonology. For instance, while an MCMM can find Hebrew's default 
masculine suffix \textit{-im}, it cannot say whether it is
masculine or feminine in a given word, as this suffix
also occurs in idiosyncratic feminine plurals. The extrinsic part of this project's
 evaluation will examine my system's utility as a component within such a pipeline.
(See section~\ref{sec:paradigms}.)

\section{Two approaches to morphology}
%We have on the one hand Word-and-Paradigm approaches and on the other hand morpheme-based approaches. This is the central distinction. But also separate and non-separate.
There are two main opposing camps concerning the nature of morphology. These camps go by different names. 
But first let us arbitrarily a choose. A ``primary monicker" for each category in order to avoid avoid confusion in the discussion that follows. 
We shall call one the \textit{morpheme-based} camp and the other the lexeme-based camp, following \cite{aronoff:1994}. Item and Arrangement
Lexical (And incremental) \citep{stump:2001}. But most importantly, \textit{lexical}.
%0. Describe the morpheme-based approach.
one-to-one mapping between a [what] and a form, i.e., a single exponent.

%1. What are the origins of the morpheme? Who are its principle tenets?
Sanskrit. Minimal sign stuff. What is a minimal sign? Smallest chunk of sound(s) that has meaning.  Ferninand Saussure is regarded as the founder of Structuralism, which in the United States developed into American Structuralism in the 1930s.
Item-and-Arrangement ``One of the main problems for the IA model is that often the mapping betweenmorphosyntactic information and phonological information is not a one-to-one relationship." % Bonet
Item-and-Process

%2. Why/How are morphemes problematic? 
Well, there are many examples of many-to-many mappings between morphosyntactic properties and form(s). 

%3. What are the origins of the lexeme-based (or Word-and-Paradigm) approach?
Greco-Roman. Word and Paradigm (WP). Word-based.

%4. What is the lexeme-based approach all about? What is it's central creed?

%5. Who are the key proponents of the lexeme-based approach, and what do they say?

\subsection{Morpheme-based morphology}
\subsection{Lexeme-based (or Autonomous) morphology}

\section{A Hebrew example}
%But what is it exactly that distinguishes these two broad categories? 
%What makes Word-and-Paradigm approaches different from morpheme-based?
[Our system's clusters
% In the present work, we describe a computational system for clustering
% words according to shared building blocks of form, building blocks
% that
correspond roughly to Aronoff's morphomes.]
%representing the building block (or morphome) common to the words in that cluster.
%of the building block that is  morphological building block. However, we use ``morphological" here in the sense of Aronoff. 
%Instead of requiring that such building blocks have a particular
%meaning, 
Hence, the system does not require building blocks to have particular
meanings. Instead, it looks for \emph{pre-morphosyntactic} units, i.e.,
ones assembled from phonemes, but not yet assigned a syntactic or
semantic meaning. In a larger pipeline, such building blocks could
serve as an interface between morphosyntax and phonology.
%\marginpar{Mention realizational morphology approximately here.}
% That is, the morphosyntactic level would assemble morphosyntactic
% units from these building blocks rather than from raw phonemes.
For instance, while our system can find Hebrew's default masculine
suffix \textit{-im}, it does not specify whether it is in fact
masculine in a given word or whether it is feminine, as this suffix
also occurs in idiosyncratic feminine plurals.

\cite{round:md:2016} is a great man! He is SOOOOO handsome!

Our system also encounters building blocks like the \textit{t} in
fig.~\ref{fig:t}, which might be called ``quasi-morphemes'' since
they recur in a wide range of related forms, but fall just short of
being entirely systematic.\footnote{Though, see \citet{faust:2013} 
for an analysis positing /-t/ as Hebrew's one (underlying) 
feminine-gender marker.}  The \textit{t} in fig.~\ref{fig:t} seems
to be frequently associated with the feminine morphosyntactic
category, as in the feminine nationality suffix \textit{-it}
(\textit{sini\textbf{t}} `Chinese (\textsc{f})'), the suffix
\textit{-ot} for deriving abstract mass nouns (\textit{bhirw\textbf{t}}
`clarity (\textsc{f})'), as well as in feminine singular and plural
present-tense verb endings
%\footnote{In most binyanim, that is.} 
(e.g., \textit{kotev-e\textbf{t}} `she writes' and \textit{kotv-o\textbf{t}}
`they (\textsc{f.pl}) write', respectively).
%, and construct state of feminine nominals (). 

In fig.~\ref{subfig:maqomi}, note that this \textit{t} is present in
both the \textsc{f.sg} and \textsc{f.pl} forms.  However, it cannot
be assigned a distinct meaning such as `feminine,' %to this \textit{t},
since it cannot be separated from the \textit{o} in the \textsc{f.pl}
suffix \textit{-ot}.\footnote{If the \textit{o} in \textit{-ot} meant
``plural,'' we would expect the default \textsc{m.pl} suffix to be
\textit{-wm} instead of \textit{-im}.}  Moreover, this \textit{t} is
not always the \textsc{f.sg} marker; the ending \textit{-a} in 
fig.~\ref{subfig:gadol} is also common. Nevertheless, the frequency
with which \textit{t} occurs in feminine words does not seem to be
accidental. It seems instead to be some kind of building block, and
our system treats it as such.

Need need need to lay out morphosyntactic categories.
%\begin{table}
%\begin{center}
%\begin{subtable} %{0.5\linewidth}
%
%{\begin{tabular}{lcc}
%\  & masc & fem  \\
%\hline 
%sg & maqomi & maqomit  \\
%pl & maqomiim & maqomiwt  \\
%\end{tabular}}
%%\caption{maqomi `local'}
%\label{tab:1a}
%
%\end{subtable}%
%
%\begin{subtable} %{0.5\linewidth}
%
%{\begin{tabular}{lcc}
%\  & masc & fem  \\
%\hline 
%sg & gadol & gadolh  \\
%pl & gadolim & gadolwt  \\
%\end{tabular}}
%%\caption{gadol `big'}\label{tab:1b}
%
%\end{subtable}
%\caption{A table}\label{tab:1}
%\end{center}
%\end{table}
\begin{exe}
\ex hh
\ex \itab{tt} \tab{tyht} \tab{8th4ke fk}
\ex \itab{yy77} \tab{hjkfgggggggg} \tab{jj}
\end{exe}

\begin{figure}
\begin{center}
\subfigure[Realizations of \textsc{gender} on adjectives.]{
\begin{tabular}{lll}
& \textsc{masc} & \textsc{fem}  \\ %& \textsc{masc} & \textsc{fem} & \textsc{masc} & \textsc{fem} \\
\hline 
\textsc{sg} & gadol & gdol-a  \\
\textsc{pl} & gdol-im & gdol-o-\textbf{t} \\ \hline
%\textsc{sg} & sin-i & sin-i-\textbf{t} \\ 
%\textsc{pl} & sin-i(y)-im & sin-i(y)-o-\textbf{t} \\ \hline 
\textsc{sg} & maqom-i & maqom-i-\textbf{t}  \\
\textsc{pl} & maqom-i(y)-im & maqom-i(y)-o\textbf{t} \\ \hline
\label{subfig:adjectives}
\end{tabular}
} \\
%\subfigure[\textit{maqomi} `local']{
%\begin{tabular}{l|cc||cc||cc}
%& \textsc{masc} & \textsc{fem}  \\
%\hline 
%\textsc{sg} & maqom-i & maqom-i\textbf{t}  \\ \hline 
%\textsc{pl} & maqom-iy-im & maqom-iy-o\textbf{t} 
%\label{subfig:maqomi}
%\end{tabular}
%}
%\subfigure[\textit{sini} `Chinese']{
%\begin{tabular}{lcc}
%& \textsc{masc} & \textsc{fem}  \\
%\hline 
%\textsc{sg} & sin-i & sin-it  \\ \hline 
%\textsc{pl} & sin-iy-im & sin-iy-o\textbf{t}
%\label{subfig:sini} 
%\end{tabular}
%} \\


%\end{center}
%\caption{The morphome $\mu$\textsc{t} in agreement suffixes}
%\label{fig:t}
%\end{figure}
%
%\begin{figure}
%\begin{center}
%\subfigure[\textit{maqomi} `local']{
%\begin{tabular}{lcc}
%& \textsc{masc} & \textsc{fem}  \\
%\hline 
%\textsc{sg} & maqom-i & maqom-i\textbf{t}  \\ \hline 
%\textsc{pl} & maqom-iy-im & maqom-iy-o\textbf{t} \\
%\label{subfig:maqomi}
%\end{tabular}
%} \\
\subfigure[\textit{s.p.r}, \textit{Piel} (`tell'), past and future tenses]{
\begin{tabular}{l|cc|cc}
& \textsc{masc} & \textsc{fem} & \textsc{masc} & \textsc{fem} \\
\hline 
\textsc{1.sg} & sipar-ti & sipar-ti &  {\textglotstop}a-saper & {\textglotstop}a-saper \\  
\textsc{2.sg} & sipar-ta & sipar-t & te-saper & te-sapr-i \\
\textsc{3.sg} & siper & sipr-a & ye-saper & te-saper \\  \hline
\textsc{1.pl} & sipar-nu & sipar-nu & ne-saper & ne-saper \\ 
\textsc{2.pl} & sipar-tem & sipar-ten & te-sapr-u & te-sapr-u \\
\textsc{3.pl} & sipr-u & sipr-u & ye-sapr-u & ye-sapr-u \\
\label{subfig:spr-pastandfuture}
\end{tabular}
}  
%\subfigure[\textit{s.p.r.} \textit{Piel}, \textit{future} `will tell']{
%\begin{tabular}{lcc}
%& \textsc{masc} & \textsc{fem}  \\
%\hline 
%\textsc{1.sg} & ?a-saper & ?a-saper \\ 
%\textsc{2.sg} & te-saper & te-sapr-i \\
%\textsc{3.sg} & ye-saper & te-saper \\ 
%\textsc{1.pl} & ne-saper & ne-saper \\ 
%\textsc{2.pl} & te-sapr-u & te-sapr-u \\
%\textsc{3.pl} & ye-sapr-u & ye-sapr-u \\ 
%\label{subfig:spr-imperf}
%\end{tabular} 
%} \\
\subfigure[\textit{s.p.r.}, \textit{Piel} (`tell'), present tense]{
\begin{tabular}{lcc}
& \textsc{masc} & \textsc{fem}  \\
\hline 
\textsc{sg} & me-saper & me-saper-et \\ 
\textsc{pl} & me-sapr-im & me-sapr-ot \\ \hline
%\textsc{sg} & ye-saper & te-saper \\ 
%\textsc{pl} & ne-saper & ne-saper \\ 
%\textsc{pl} & te-sapr-u & te-sapr-u \\
%\textsc{pl} & ye-sapr-u & ye-sapr-u \\ 
\label{subfig:participle}
\end{tabular} 
}
%\subfigure[\textit{sini} `Chinese']{
%\begin{tabular}{lcc}
%& \textsc{masc} & \textsc{fem}  \\
%\hline 
%\textsc{sg} & sin-i & sin-i-t  \\ \hline 
%\textsc{pl} & sin-iy-im & sin-iy-o\textbf{t}
%\label{subfig:sini}
%\end{tabular}
%}
\end{center}
\caption{The \textit{t} quasi-morpheme}
\label{fig:t}
\end{figure}

OK OK What do we got, son? We got a little $\mu$. Some morphomes, Jack. Butterscotch. Milkies. What morphomes?
$\mu\textsc{t}$ The morphome \textit{t} regularly associated with the feminine 
gender, occurring as suffix and a prefix in future-tense verbs.
$\mu\textsc{e}$
$\mu{\textsc{o}}$
$\mu{\textsc{o}}$

\textipa{[TIsIzs@maIpieI]}  \v{s} \textglotstop

% pattern 1: paal
% pattern 2: nifal
% pattern 3: piel
% pattern 4: pual
% pattern 5: hifil
% pattern 6: hufal
% pattern 7: hitpael

% ktef?yim   [Note the deleted "schwa" and resulting consonant cluster

%\begin{exe}
%\ex %\begin{xlist}
%	\NumTabs{2}
%	sinit \,\, `Chinese (f)' \tab bhirwt \,\, `clarity (f)'\\
%	%bhirwt \tab `clarity (f)' \tab `clarity (f)' \\
%	kotebt \,\, `write f.sg' \tab kotebwt \,\, `write f.pl' 
%	%\end{xlist}
%\ex \begin{xlist}
%	\ex mkwmi \hspace{\fill} `Chinese (f)'
%	\ex mkwmiim \hspace{\fill} `clarity (f)'
%	\ex  \hspace{\fill} `she writes' 
%	\end{xlist}
%\ex \begin{xlist}
%	\ex sinit \hspace{\hfill} `Chinese (f)'
%	\ex bhirwt \hspace{\hfill} `clarity (f)'
%	\ex kotebt \hspace{\hfill} `she writes' 
%	\end{xlist}
%\end{exe}
\begin{figure}
\begin{center}
\begin{tabular}{cl}
Label & Description \\
$\mu1$ & ``feminine" \textit{t} \\
$\mu\textsc{piel}$ & Pi`el binyan, \textsc{past}-tense vowel pattern \textit{i..e} \\
$\mu3$ & Pi`el binyan, non-\textsc{past} vowel pattern \textit{a..e}\\
$\mu4$ & root \textit{s.p.r} `tell' \\
$\mu\textsc{e}\text{-}$ & the \textit{e} in Pi`el \textsc{fut} and \textsc{pres} prefixes (except 1.\textsc{sg}) \\
$\mu6$ & the \textit{a} in the Pi`el 1.\textsc{sg} \textsc{fut} prefix \\
$\mu7$ &  derivational adjectival suffix \textit{-i} \\
$\mu8$ & the \textit{i} in the \textsc{masc} suffix \textit{-im} \\
$\mu9$ & the \textit{m} in the \textsc{masc} suffix \textit{-im} \\
$\mu10$ & the \textit{m} in the \textsc{pres} prefix \textit{me-} \\
\end{tabular}
\end{center}
\label{fig:morphome-key}
\end{figure}
Because our system is not intended to identify morphosyntactic
categories, its evaluation poses a challenge, as morphological
analyzers tend to pair form with meaning. \marginpar{Actually, they don't really pair form with meaning, not if they are truly unsupervised.} 
Nevertheless, we
tentatively evaluate our system's clusters against the 
\emph{modified} output of a finite-state morphological analyzer. 
That is, we map this analyzer's abstract morphosyntactic categories onto
categories that, while still essentially morphosyntactic, correspond 
more closely to distinctions in form (see section~\ref{sec:eval}).
%a finite abstract morphosyntactic
%categories, with some form-based mappings, discussed in
%section.

%\marginpar{MD: Is part of the reason for clustering because we look
%  for pre-morphosyntactic categories? TM: The clusters are (in effect) 
%  pre-morphosyntactic categories.}

\section{Third Question}
What is a morphome, and how is this notion useful? How does it fit in with the word paradigm category of approaches? What does it contribute to the word paradigm approaches?
\section{Motivate Two-Pronged Evaluation}
4. How difficult is it to identify morphomes? Are there clearcut criteria? How straightforward is the procedure?
4.5. Do other ULM approaches find morphomes? If so, how do the do it without morphosyntactic annotation?
